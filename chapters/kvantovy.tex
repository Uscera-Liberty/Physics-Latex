\documentclass[../main.tex]{subfiles}
\graphicspath{{\subfix{../images/}}}

\begin{document}
\chapter{Физические основы ЭВМ}

\section{Краткие сведения из квантовой механики}
В 1923 Де Блойль вывел формулу длины волны, после было доказан наличие волновых свойств у электорна
\[\lambda = \frac{h}{p}\]
\begin{center}
    где h это постоянная Планка равная $6,626 * 10 ^-34$
\end{center}
\subsection{Корпускулярно-волновой дуализм}
Гейзенберг пришел к следующим соотношениям называемыми Неравенство Гейзенберга(соотношения неопределенности):
\[\Delta x \cdot \Delta p_x >= h\]
\[\Delta y \cdot \Delta p_y >= h\]
\[\Delta z \cdot \Delta p_z >= h\]
\begin{center}
    где $\Delta x, \Delta y, \Delta z$ -- положение, $\Delta p_x, \Delta p_y, \Delta p_z$
\end{center}

Волновая функция $\Psi(x,y,z)$ - физического не имеет.
\[ |\psi(x,y,z,t)| ^2 \] - имеет физический смысл

Если $\psi_1, \psi_2, \ldots \psi_n$ - волновые частицы в разных положениях, то
$\psi = \psi_1 + \psi_2 + \ldots + \psi_n$ -  общая волновая функция.

\subsection{Спектр электронных состояний в атомах молекулах и кристаллах}
Рассмотрим абстрактный пример: пусть частица в потенциальном яме:

\[A \cdot e^{i(kx+ \omega t)} = \psi_1\]
\[A \cdot e^{i(-kx - \omega t)} = \psi_2\]

Поскольку $e^{ikx} - e^{-ikx} = 2i \cdot \sin{kx}$

\[\Psi = \psi_1 + \psi_2 = A(e^{ikx} \cdot e^{-i \omega t} = A \cdot e^{-i \omega t} \cdot 2i \cdot \sin{kx})\]

Так как $\sin{kL} \Rightarrow kL = \pi n$ выведем формулу волного числа:
\[k = \frac{\pi n}{L} , n = 1, 2, \ldots\]
\begin{center}
    где n - количество длин волн
\end{center}

Формула, определяющая длину волны:
\[k = \frac{2 \pi}{\lambda}\]

Используя формулу Де-Бройля 
\[\lambda = \frac{h}{mv} = \frac{h}{p} \Rightarrow p = \frac{h}{\lambda} = \frac{2 \pi}{\lambda} \cdot \frac{h}{2 \pi} = k\cdot \frac{h}{2 \pi}\]

\[\frac{h}{2 \pi} = h\]
\begin{center}
    - приведенная постоянная Планка
\end{center}

Тогда $p = \frac{n h}{2 L}$
Рассмотрим потенциальную энергию:
\[W = \frac{p^2}{2 m} = \frac{n^2 h^2}{8L^2 m}\]
В силу того что n есть целое число, то W принимает фиксированные значения - квантуется.

Рассмотрим атом водорода: пусть на электрон действует сила Кулона, а именно
\[F = \frac{1}{4 \pi \epsilon_0} \cdot \frac{q_e \cdot q_p}{r^2} = \frac{1}{4 \pi \epsilon_0} \cdot \frac{e^2}{r^2}\]
Рассмотрим второй закон Ньютона:
\[F = ma \Rightarrow a = \frac{v^2}{r} \Rightarrow\]
\[\frac{1}{4 \pi \epsilon_0} \cdot \frac{e^2}{r^2} = m \cdot \frac{v^2}{r} \Rightarrow\]
\[r  = \frac{e^2}{4 \pi \epsilon_0 m v^2}\]

На орбиту электрона укладывается целое число волн, то есть число квантуется: 
\[2 \pi r = \lambda \cdot n\]
\[2 \pi r = \frac{h n}{mv}\]
Решим систему (1,2) с неизвестными r и v:
\[r = \frac{e^2}{4 \pi \epsilon_0 m v^2}\]
\[r = \frac{hn}{2 \pi mv}\]

\[e^2 \cdot 2 \pi mv = hn \cdot 4 \pi \epsilon_0 mv^2\]
\[r = \frac{hn}{2 \pi mv}\]

\[e^2 \cdot 2 \epsilon_0 hn = e^2\]
\[r = \frac{hn}{2 \pi mv}\]

\[v = \frac{e^2}{2 \epsilon_0 hn}\]
\[r = \frac{h^2 n^2}{\pi me^2}\]

Уравнение Шредингера:

\[\psi = A \cdot e^{i(kx - \omega t)}\] - для одномерного случая

\begin{center}
    где k - волновое число, $\omega$ - усгловая скорость(волновой вектор)
\end{center}
Если $\psi_1, \psi_2 \ldots \psi_n$ - 

\[r = \frac{n^2 h^2 \epsilon_0}{m e^2 \pi}\]
\[v = \frac{e^2}{2h \epsilon_0 n}\]

Получаем, что чем дальше электрон тем медленнее он вращается
Тогда
\[E_k = \frac{mv^2}{2} = \frac{me^4}{8h^2 \epsilon_0^2 n^2}\]
\[E_p = \frac{1}{4 \pi \epsilon_0} \cdot \frac{e^2}{r} = - \frac{m e^4}{4 h^2 \epsilon_0^2 n^2}\]

Поскольку $\vec F = - grad E_n$ получаем полную энергию электрона на орбите при $E_n \to 0$ и $r \to \infty$:

\[E = E_k + E_p = - \frac{me^4}{8 h^2 \epsilon_0 ^2 n^2}\]

Электрон как и всякое тело стремится в минимуму потенциальной энергии, для электорна это достигается на нижней орбите - самой стабильной орбите. 
Следовательно электрон вседга стремится к нижней орбите, при этом он должен избавится от избытка энергии. 
Поскольку энергия электрона квантована, при переходе он испускает волны определенной частоты.
Причем набор этот набор у каждого элемента фиксирован. 

Свободный электрон может иметь любую энергию, а несвободный только фиксированную.
Будем называть валентным тот электрон который прикрепленным к некоторому ядру, а свободным соответсвенно неприкрепленным.

\subsection{Энергетические состояния электронов в многоэлектронных атомах}

\subsubsection{Принцип Паули}
\begin{center}
    В пределах одной квантовой системы в данном квантовом состоянии может находится только один фермион(электрон)
\end{center}
Квантовое состояние описывает всевозможное состояния квантовой системы, характеризуется квантовыми числами.
\begin {enumerate}
    \item n определеляет радус круговой орбиты или величину большой полуоси эллиптической орбиты. (1,2,3...)
Состояние электрона определяемое главным квантовым числом называют главным энергетическим уровне.

    \item l главное орбитальное квантовое число определяет величину малой полуоси эллиптической орбиты. (0,1,2, \ldots n-1) 

    \item m магнитное квантовое число (0 +- 1... +- l) определяет ориентацию орбит, орбитальный магнитный момент электрона

    \item s спиновое квантовое число s = +-1/2
\end{enumerate}

Из принципа Паули вытекает что на орбите не может находится не более двух электоронов.

\subsubsection{Принцип Хунда}
\begin{center}
    Суммарное значение спина электронов данного подуровня должно быть максимальным.
\end{center}

\section{Виды химических связей}

\subsection{Ковалентная связь}

При такой связи возникает обоществленная пара электронов  по одному от каждого атома.Различают два вида связи, связь смещенная и не смещенная.
\begin{enumerate}
    \item Несмещенная связь - возникает между атоматими одинаковой массы($O_2, H_2, N_2$)
    \item Смещенная связь - обычно возникают между атомами с различным числом протонов в ядре, что вызывает поляризацию.
\end{enumerate}

\subsection{Металлическая связь}
Будем рассматривать решетку образованную электронами и ионами соотвесвтенно, такая решетка будет сохраняться за счет электронного облака.

\subsection{Ионная связь}
Характерно для связи металлов с неметаллами.
\subsection{Молекулярная связь}
Возникает в газах при очень низких температурах.

\subsection{Водородная связь}

Для связь электрон может принимать фиксированное значение энергии, их называют энергетическими уровнями.

\section{Электопроводимость твердых тел}
Модель электронного газа - в металлах большое количество свободных электронов. Эти электроны внутри металла двигаются непрывно и хаотично подобно молекулам газа.
При наличии электрического поля, эти свободные электроны приходят в упорядоченное движение, создавая электрический ток.

Напишем среднюю хаотическую энергию электрона
\[E_t = \frac{3}{2} kT\]

В результате применения закономв термодинамики приходим к этой формуле:
\[j = \frac{1}{\rho E}\]
Модель верна потому что приходим потом к закону Ома в дифферернцированной форме.

При своем упорядоченном движении электроны сталкиваются с ионами кристаллической решетки чем объясняется электрическое сопротивление.С ростом
температуры проводника увеличивается сопротивление проводника, поскольку само движение атомов становится активнее и электроны ольше сталкиваются.

В полупроводниках наоборот: при уменьшении температуры увеличивается сопротивление, то есть теория электронного газа не подходит для полупроводников.

Сверхпроводимость при ультра низких температурах, на таком явлении построены квантовые компьютеры.
\section{Квантовая модель электропроводимости}
Рассмотрим энергию атомов, можно разделить на три уровня:
\begin{itemize}
    \item Валентная зона($E_v$ - граница валентной зоны)
    \item Запретная зона- те уровни энергии которую не может принимать электрон в этом материале
    \item Зона проводимости - электрон свободный  иможет перемещаться по материалу ($E_c$ - граница нижняя зоны)
\end{itemize}

Наивысшая плотность уровней в одной зона вблизи границ зон, верхней для валетной и нижняя для зоны проовдимости.

У металлов запретная зона практически отсутсутвует, $\Delta E < 0,1 $еВ - ширина запретной зоны, измеряется в энергиях эВ.

У полупроводников есть запретная зона, но $ 0,1 \leq \Delta E \geq 0,3$
У диэлектриков $\Delta E > 3 $
Чтобы перейти в другую зону, электрону нужно сообщить энергию, можно тепловую энергию(фотоэлементы).Для полупроводников 
достаточно сообщить тепловую энергию чтобы перевести в проводимое состояние.Для диэлектриков слишком широкая полоса для преодоления электронами следовательно
не может идти ток, етсь свободные электроны но они не достигают зону.

Для металлов, поскольку по сути запретной зоны нет то прееходя в другую зону они не вызывают аткого влияния как колебания кристаллической решетки следовательно 
сопротивление ухудшается.

Рассмотрим трехмерный потенциальный ящик, для одномерного: $P_p = \frac{h}{L} \cdot n \Rightarrow n = \frac{P_p}{h} \cdot L$
разложим квантовое число по каждому измерению
\[n_x = \frac{2L}{h} \cdot P_x\]
\[n_y = \frac{2L}{h} \cdot P_y\]
\[n_z = \frac{2L}{h} \cdot P_z\]

\[P^2 = P^2_x + P^2_y + P^2_z = \frac{h^2}{4L^2}(n_x^2 + n_y^2+n_z^2)\]

\[E = \frac{p^2}{2m} = \frac{h^2}{8mL^2}(n_x^2 + n_y^2+n_z^2)\]

Ферми - уровень энергии для которого заполненны электронные состояния  при T стремящемся к нулю называется уровнем энергии Ферми.
Для металлов Ec = Ef = Ev это одна и та же линия.Вероятность обнаружить электрон на уровне Ферми равен одной второй 

Уровень ферми определяет значение главного квантового числа.
Тогда можем считать что 
\[n_zmax = n_ymax = n_xmax = \frac{2L}{h} \cdot P_f\]
До сих пор мы рассматривали ящик в рассматриваемом пространстве, есйчас переходим к пространсву квантовых чисел, то есть 
по каждой оси число n ограниченно $n_max$

\[n = \sqrt{n^2_x+n^2_y+n^2_z} \leq \frac{2L}{h} \cdot P_F\]
Тем самым любое значение 
\[0 < n_x, n_y,n_z \leq \frac{2L}{h} \cdot P_F\]
четвертая часть сферы, найдем площадь поверхности сферы
\[S = 2 \frac{1}{8} \frac{4}{3} \pi \cdot (\frac{2L}{h} P_F)^3 = \frac{8 \pi}{3} (\frac{P_F}{h})^3 \cdot L^3\]
где $L^3$ - объем потенциального ящика, обычно рассматривают в единице объема, тогда 
\[N = \frac{S}{V} = \frac{8 \pi}{3} \frac{P_F^3}{h^3}\]

\[P_F = \frac{h}{2} \cdot (\frac{3N}{\pi})^{\frac{1}{3}}\]
\[E_F = \frac{p^2}{2m} = \frac{h^2}{8m}(\frac{3N}{\pi})^{\frac{2}{3}}\]

\[N = \frac{\pi}{3} \frac{(8mE_f)^{3/2}}{h^3}\]
Найдем плотность энергетических состояний электрона на единицу энергии(количество состояний)
\[\frac{dN}{dE} = \rho_N(E) = \frac{\pi (8m)^{\frac{3}{2}}}{2 h^3} \cdot E^{\frac{1}{2}}\]

\subsubsection{Распределение Ферми}
Это функция f(E) представляет из себя вероятность того что электрон будет находиться находится на данном уровне с энергией Е(вероятность заполнения состояния).
\[f(E) = \frac{1}{1 + \exp{\frac{E-E_F}{kT}}}\]

(функция)

Концентрация электронов в стоттоянии

\[h(E) = \rho_N(E) \cdot f(E)\]
Общая концентрация электронов
\[n = \int_{E_F}^{\infty} f(E) \cdot \rho_N(E) dE\]

\subsection{Полупроводники}
Какие полупровдники бывают

Ширина запрещенной зоны: Германий $\Delta E \approx 0,67$ Кремний $\Delta E \approx 1,12$

Носители зарядов в полупроводнике

Пусть электрон в решетке получает жнергию чтобы выйти из ковалентной связи следовательно из решетки - появление свободных электронов.
Когда есть свободное метсто в связи(там меньше потенциальной энергии) он заходит туда.
Рассмотрим прошлой место теперь оно свободно, там образуется положительный заряд, и так электрон может перейти на другую орбиталь с вакантным местом.
Так пустое место заряженное место перемещается до кристаллу, это место находится всегда в валентной зоне - поскольку меньше потенциальной энергии.

В полупроводнике носители заряда являются как жлектроны так и вакантные места.(n - электрон, p - пустое место)
Так образуется пара - пустое место и электрон. В идеальном кристалле количество дырок и электронов совпадает.

Причем свободный и электрон в решетке - разные жлектроны - массы не совпадают, у пустого места тоже есть масса - эффективная масса.
\[m^{*}_n = \frac{p_E}{v_n}\] - масса электрона
\[m^{*}_p = \frac{p_F}{v_p}\] - эффективная масса

Образование электронно-дырочной пары называется генерации, обратный процесс - рекомбинации. 
При генерации электрон из валентной зоны в зону проводимости, и при рекомбинации наоборот.

Количество электронов в зоне проводимости

\[R = \frac{2L}{h}p \Rightarrow dR = \frac{2L}{h}\cdot dp  \]
\[S = \frac{2}{8} \cdot 4 \pi R^2\]
\[V = \frac{2}{8} \cdot 4 \pi R^2 dR \Rightarrow = \frac{8 \pi L^3}{h^3} p^2 dp\] - площать узкой области между dE 
\[N = \frac{8 \pi}{h^3} p^2 dp\]

Общая энергия в полупроводнике
\[E = E_c + \frac{p^2}{2m_n}\]

Получим зависимость p от энергии и потом зависимость n

\[p =  (2m_n(E-E_c))^\frac{1}{2}\]
\[dp  = \frac{\sqrt{2m_n \cdot dE}}{2(E - E_c)^{\frac{1}{2}}}\]
Тогда получим
\[N = \frac{8 \pi}{h^3} \cdot 2m_n(E-E_c) \cdot \frac{\sqrt{2m}}{2} \cdot (E-E_c)^1/2 dE = \frac{4 \pi}{h^3}(2m_n)^{3/2} (E-E_c)^1/2 dE\]

\[ \frac{dN}{dE} = \rho\]
\[\rho_n(E) = \frac{4 \pi}{h^3}(2m_n)^3/2 (E-E_c)^1/2\]
\[\rho_p(E) = \frac{4 \pi}{h^3}(2m_p)^3/2 (E_v-E)^1/2\]

\[n = \int_{E_c}^{\infty} \rho_n(E) f(E) dE = \int_{E_c}^{\infty} \frac{4 \pi}{h^3}(2m_n)^3/2 (E-E_c)^1/2 \cdot \exp{\frac{- E - E_F}{kT}} dE\]
\[\frac{4 \pi}{h^3}(2m_n)^3/2 \int_{E_c}^{\infty} (E-E_c)^1/2 \cdot \exp{\frac{- E_c - E_F}{kT}} dE = \frac{4 \pi}{h^3}(2m_n)^3/2 \cdot\exp{\frac{- E_c - E_F}{kT}} \int_{0}^{\infty} \sqrt{U}exp(-u) \cdot (kT)^3/2 = \frac{\sqrt{\pi}}{2} \]
Для полупроводников
\[E_c - E_f >> kT\]
\[E_F - E_v >> kT\]

$N_v$ эффективная плотность состояний в валентной зоне

Введем параметр n - температурную концентрацию носителей заряда:
\[n_i^2 = n \cdot p \Rightarrow n_i = \sqrt{N_c \cdot N_v} \cdot \exp{\frac{- \Delta E}{2 k T}}\]
Рассматриваем температуру близкой к нулю градусов кельвина.

\subsection{Уровень Ферми в собственном полупроводнике}

Рассмотрим температуру стремящуюся к нулю градусов кельвина, скажем что веротяность появления электрона в зоне проводимости равна вероятности его отсутсвия в валентой зоне:
\[f(E_c) = 1 - F(E_v) \Rightarrow \frac{1}{1+ \exp{\frac{E_c - E_v}{kT}}} = 1 - \frac{1}{1+ \exp{\frac{E_v - E_f}{kT}}} \]
\[ = \frac{\exp{\frac{E_v- E_f}{kT}}}{1 + \exp{\frac{E_v - E_f}{kT}}} = \frac{1}{\exp{- \frac{E_v - E_f}{kT}} +1 } = \frac{1}{1+\exp{\frac{E_v - E_f}{kT}}}\]

Из монотонности функции F должны быть равны и агрументы функции то есть $\frac{E_c - E_v}{kT} = \frac{E_v - E_f}{kT}$, тогда получим:
\[E_f = \frac{E_c + E_v}{2}\]
Так уровень ферми располагается по центру и не зависит от температуры, он зафиксирован.

Представим что в решетке можно подставить в решетку другим атомов напрмер атомом кремния, для того чтобы увеличить проводимость поулпроовдника.
Например возьмем полупроводники Si, Ge. При легированнии проводников вносятся элементы третьей группы например In Ga либо элементы пятой группы P As Sb.
Соотвественно получаем по свойствам разные полупроводники, так элементы 3 группы - \textbf{акцепторы} 5 группы донорами.
\begin{itemize}
    \item Рассмотрим легирование элементами 5 группы
    электрон не участвующий  всвязях разрывается и перехоидт в зону проводимости, получается от каждого атома донора внедренного в решетку мы получаем свободный электрон - получается мы существенно 
увеличиваем количество свободных электронов. - количество дырок резко уменьшается  тогда $n >> p n * p = n^2$ 
 тогда этот полупроводник называется пп с жлектронноый проводимостью или пп n-типа.

 При этом никто процесс генерации и рекомбинации 

 $N_D$ - концентрация доноров в примеис, каждый атом вносит в полупроовдник свободный электрон.

 $n_n = N_D +p_n$ - соотношения связи между электронами и дырками в полупроводника n типа
 Таким образом концентрация электронов примерно равняяется концетрации доноров.

    \item Рассмотрим легирование акцепторами, помещаем индий (3 группа)
    Так на одной из связей будет вращаться один жлектрон то есть образуется дырка или вакантное место. Так будут поглащаться свободные электроны в эти дыркы.
    Так $p >> n$, концетрация акцепторов $N_A$. Эти полупроводники называются полупроводниками с дырочной проводимосттью или пп p типа $p_p = N_A + n_p$

\end{itemize}

\subsection{Уровень Ферми в легированных проводниках.}

Уровень ферми не двигается двигаются только  $E_c$ и $E_v$.Рассмотрим изменения при легировании 
\[ n_n = N_c \cdot \exp{-\frac{E_c - E_f}{kT}} \]
\[ p_n = N_v \cdot \exp{-\frac{E_f - E_c}{kT}} \]

Возьмем отношение $\frac{n_n}{p_n} = \frac{N_c}{N_v} \cdot \exp{- \frac{E_c + E_v - 2E_F}{kT}}$

Берем $\frac{N_c}{N_v} \approx 1$ 

Известно из $n_i^2 = n_n \cdot p_n \Rightarrow p_n = \frac{n_i^2}{n_n}$ подставим 
\[\frac{n_n^2}{n_i^2} = \exp{- \frac{E_c + E_v - 2E_F}{kt}}\]
\[2 \ln{\frac{n_n}{n_i}} = - \frac{E_c + E_v - 2E_F}{kt} \Rightarrow kT \ln{\frac{n_n}{n_i} = - \frac{E_c + E_v}{2} +E_f}\]

\[E_f = \frac{E_v + E_v}{2} + kT \ln{\frac{n_n}{n_i}} = \frac{E_c + E_v}{2} + kT \ln{\frac{N_D}{n_i}}\]

Уровень ферми будет смещен ближе к зоне проводимости. Выше для n типа, для p типа: 
\[E_f = \frac{E_c + E_v}{2} - kTln \frac{n_n}{n_i} = \frac{E_c+E_v}{2} - kT \ln{\frac{N_A}{n_i}}\]
То есть уровень Ферми больше смещен к зоне проводимости.

Определение: Если уровень Ферми при легировании попадает в зону проводимости(для n типа) или в валентную зону (p - тип) такие полупроводники называются вырожденными.

По хорошему уровень Ферми должен распологаться в запретной зоне.
При легировании появляется уровень донора и для них как раз и для него $\Delta E$ доноров и там при таком уровне энергии много электронов - для n-типа. Аналогично для p-типа 
только появляется уровень акцептора.

Рисунки жнергетические лиаграммы

\subsection{Движение свободных носителей заряда в полупроводниках.}

В полупроводниках будем рассматривать движение не только из-за действия электрического поля, но из-за неравномерного распределения электронов из меньшей концентрации в большую.
Ток в полупроводнике модет быть вызван двумя причинами: при наличии электрического поля и такой ток называется дрейфовый и в следствии неравномерного распределения на сфере заряда - диффузионный.

Мы будем рассматривать именно плотность тока:
\[j = q k v\]
 - плотность тока в проводника, v скорость направленного движения тока, за направление тока мы принимает движение положительных частиц.

\subsubsection{Дрейфовый ток}

Рассмотрим движения дрейфовых электронов: электроны , будут идти в одну сторону, а ток в другую соотвесвтенно

\[j = -e n V\]

Вычислим подвижность электрона $\mu = \frac{v_n}{E} \Rightarrow v = - \mu E$. Эта подвижность характерихуется свойствами материалов. Тогда:

\[j = e n \mu_n E\] 
знака нет посокльку ток направлен по направлению жлектрического поля

для дырок

\[j = e p \mu_p e\]

Напомним: Закон Ома в диф форме $j = \frac{E}{\rho} \sigma = 1/\rho$

\subsubsection{Диффузионный ток}
Диффузия это стремления вещества распространится по всему объему равномерно. Возникает тогда когда электроны и дырки распространенны неравномерно
тогда возникает стремление заполнить объем равномерно, например с концентрацией k.
\[\Phi_k = -D_k grad k\]
D - поток частиц через поверхность единичной площади через площадку в деиницу времени.

Перейдем к полупроводникам и рассмотрим плотность тока.
\[\vec j_difn = e \cdot D_n \cdot grad n = -e \cdot n \cdot \vec v_difn \]

\[\vec j_difp = - e \cdot D_p \cdot grad p = e \cdot p \cdot \vec v_difp\]

\[\vec j_n = \vec j_difn + \vec j_drp = e(n \cdot \mu_n \cdot E + D_n \cdot grad n)\]

\[\vec j_n = \vec j_difp + \vec j_drp = e(n \cdot \mu_p \cdot E + D_p \cdot grad p)\]

\subsection{Уравнение неразрывности}
\[\frac{d \rho_p}{d t} = -div \vec j\]
- изменение плотности заряда за время T - уравнение неразрывности

Плотность заряда для электронов
\[\rho_{qn} = - e \cdot n\]
\[\frac{d -en}{d t} = -div(e \mu_n n \vec E + e D_n grad n)\]
\[-e \frac{d n}{d t} = -e \mu_n div(n \vec E) - eD_n \Delta n\]
\[\frac{d n}{d t} = \mu_n div(n \vec E) + D_n \Delta n\]

(переписать)

При рассмотрении легированных проводников n_n = N_D, поэтому уравнение неразрывности мы рассматриваем для не основных носителей зарядов.
Записываем с учетов генерации и рекомбинации:

\[\frac{d n_p}{d t} = \mu_n \frac{d n \vec E}{dx} + D_n \cdot \frac{d^2 n}{dx^2}+ G_n - \frac{n_p - n_p0}{\tau_n}\]

\[\frac{d p_n}{d t} = \mu_n \frac{d n \vec E}{dx} + D_n \cdot \frac{d^2 n}{dx^2}+ G_p - \frac{n_p - n_p0}{\tau_n}\]

G - скорости генрации электронов и дырок, p n - равновесная концентрация не основных носителей заряда. tau_n - характерное время рекомбинации(среднее время когда электрон остается свободным)
v. - подвижность электрона D - коэффициент диффузии

Соотношение Энштейна

\[D_n = \frac{kT}{e} \mu_n\]
\[D_n = \frac{kT}{e} \mu_n\]

\section{Электрические переходы}

Электрический переход - называется граничный слой между двумя областями электрические характеристики которых имеют существенные физические различия.В полупроводниках 4 перехода:
-- Электронно дырочный переход или p_n переход
Переход между двумя областями одного и того же полупроводника имеющими различный тип электропроводимости. Но полупроводник один и тот же
-- Переход металл-полупроводник
-- Переход между двумя областями проводника с одним типом электропроводимости но с различной степенью легирования
-- Гетеро переход переход между двумя различными полупроводниковыми материалами с различной шириной запрещенной зоны.

\subsection{P-N переход.Контактные явления на границе}

Рассмотрим полупроводник н-п типа, первым делом начинается диффузия электронов и дырок. После происходит рекомбинация, после за их счет происходит 
накопление заряда в граничной области поскольку рекомбинированные жлектроны уже никуда не уходят, а остаются на месте.
Возникает контанктное электрическое поле и разность потенциалов $\Delta \phi = \phi_p - \phi_n$ - потенциальным барьером.
Возникающее поле препятсвует дальнейшей диффузии носителей заряда.С большой энергей дырки и электроны могут проскакивать через переход, такое дфижение вызывано дифузией то есть это диффущионный ток.
Для основных носителей заряда это поля не является барьером- и они переносят дрейфовый ток. В конце концов в переходе устанавливается равновесие
Так диффузионный ток неосновных зарядов становится равным дрейфому току основных зарядов.

Образуется запирающий слой без носителей заряда, (контакное жлектрическое поле)

Если Na == Nd то такой переход называют симметричным пеерходом, иначе несимметричным. 
У полупродника сильно легированного обозначают n+.

Рассматриваем невырожденный проводник, картинки 


\end{document}
