\chapter{Динамика}

\section{Динамика материальной точки. Законы Ньютона}
\subsection{Первый закон Ньютона}
\textit{\textbf{Первый закон Ньютона :} Существует такая система отсчета в которой всякое тело покоится или прямолинейно и равномерно движется до тех пор пока воздействие со стороны других тел не изменяет этого состояния - инерциальные системы отсчета (ИСО).}

\vspace{5px}

В ИСО пространство однородно и изотропно, а время однородно.
\begin{itemize}
    \item \textit{Однородность} пространства означает, что во всех его точках все физические законы действуют одинаково.
    \item \textit{Изотропность} пространства означает, что по всем направлениям пространства все физические законы(на вектора) действуют одинаково
    \item О\textit{днородность времени} означает, что во все моменты времени все физические законы действуют одинаково.
\end{itemize}
\textit{Любая система отсчета, которая покоиться или движется прямолинейно и равномерно относительно инерциальной тоже будет ИСО, следовательно их бесконечное множество}
\subsection{Понятие силы и массы.Второй закон Ньютона}

\define  \textbf{\textit{Масса}} - мера инертности тела. Под инертностью понимают способность тела сопротивляться внешнему воздействию.

\vspace{5px}

\define  \textbf{\textit{Сила}} - мера взаимодействия тел. Она может проявляться либо в получении ускорения, либо в деформации.

\begin{enumerate}
    \item \textbf{\textit{Принцип независимого действия }}

          Действия силы на тело не зависит от того покоится это тело или движется, а также не зависит от количества и вида других сил, действующих на это тело.
          \newpage
    \item \textbf{\textit{Принцип суперпозиции тел}}

          \textbf{Определение:} Если на тело действует несколько сил, то их совместное действие можно заменить действию одной силы, равной векторной сумме сил. Такую силу называют равнодействующей силой.

          Если на тело действует система сил, равнодействующая который равна $\vec 0$, то тело не меняет своё состояние.
\end{enumerate}
\textit{\textbf{Второй закон Ньютона:} ускорение тела прямо пропорционально силе, действующей на тело и обратно пропорционально массе тела} \[ \vec w = k \cdot \frac{\vec F}{m} \] где $\vec F$ - равнодействующая сила, k - коэффициент пропорциональности из-за разных единиц измерения w, m, F
\[ \vec F = m \cdot \vec W\]

Выберем единицы измерения так, чтобы k = 1, получаем:
\[[F] = \frac{\text{кг}\cdot\text{м}}{c^2} = \text{Н}\]

\textit{\textbf{Третий закон Ньютона:} действие тел друг на друга носит характер взаимодействия. Силы, с которыми тела действуют друг на друга равны по величине, но противоположны по направлению.}
\[\vec F_{12} = - \vec F_{21}\]

$\vec F_{12}$ - сила, действующая на второе тело от первого, $\vec F_{21}$ -сила, действующая на первое тело от второго
\subsection{Принцип относительности Галилея}
\begin{center}
    \begin{tikzpicture}[line cap=round,line join=round,>=triangle 45,x=0.5921540164917828cm,y=0.507076355328508cm]
        \clip(11.89,17.64) rectangle (24.15,29.81);
        \draw [->] (16,20) -- (16,28);
        \draw [->] (16,20) -- (22,20);
        \draw [->] (16,20) -- (12,18);
        \draw [->] (18,26) -- (17.97,23.35);
        \draw [->] (18,26) -- (21.65,26.02);
        \draw [->] (18,26) -- (20.86,29.4);
        \draw [->] (18,26) -- (20.37,27.18);
        \draw [->] (16,20) -- (18,26);
        \draw [->] (18,26) -- (20,20);
        \draw (16.39,24.7) node[anchor=north west] {$\vec r_0$};
        \draw (19.31,23.84) node[anchor=north west] {$\vec r^\prime$};
        \draw (15.26,21.4) node[anchor=north west] {$O$};
        \draw (17.61,27.48) node[anchor=north west] {$O^\prime$};
        \draw (20.4,29.93) node[anchor=north west] {$z^\prime$};
        \draw (19.7,29.93) node[anchor=north west] {$z^\prime$};
        \draw (20.98,27.37) node[anchor=north west] {$x^\prime$};
        \draw (17.88,23.73) node[anchor=north west] {$y^\prime$};
        \draw (20.16,28.69) node[anchor=north west] {$\vec v_0$};
        \draw (20.22,19.78) node[anchor=north west] {$M$};
        \begin{scriptsize}
            \fill [color=black] (16,20) circle (1.5pt);
            \fill [color=black] (18,26) circle (1.5pt);
            \fill [color=black] (20,20) circle (1.5pt);
        \end{scriptsize}
    \end{tikzpicture}
\end{center}

Возьмем две инерциальные системы отсчета. Пусть $\vec v_0$ - постоянный вектор, $O x^\prime y^\prime z^\prime$ - подвижная система (движется прямолинейно и равномерно)
\[ \vec r = \vec r^\prime + \vec r_0 \Rightarrow \]
\[ \vec v = \vec v^\prime + \vec v_0(const) \Rightarrow \]
\[ \vec w = \vec w^\prime \]
\textit{\textbf{Принцип относительности Галилея:} Так как и масса и ускорение точки М равны в обеих системах отсчета, то во всех инерциальных системах отсчета силы действуют одинаково.}

\vspace{5px}

\textbf{Следствие:} Никакими опытами невозможно определить движется ли наша (инерциальная) система отсчета прямолинейно и равномерно или не движется.

\section{Виды сил}
\begin{enumerate}
    \item \underline{Сила тяжести.}
          \begin{center}
              \begin{tikzpicture}[line cap=round,line join=round,>=triangle 45,x=1.0cm,y=1.0cm]
                  \clip(5.33,10.9) rectangle (15.43,15.81);
                  \fill[fill=black,fill opacity=0.1] (6,13) -- (6,12) -- (11,12) -- (11,13) -- cycle;
                  \fill[fill=black,fill opacity=0.1] (12.46,12) -- (13.38,12) -- (13.38,12.98) -- (12.48,12.98) -- cycle;
                  \draw (8,13)-- (8,14);
                  \draw (8,14)-- (9,14);
                  \draw (9,14)-- (9,13);
                  \draw (9,13)-- (8,13);
                  \draw (6,13)-- (11,13);
                  \draw [->] (8.46,13.56) -- (8.46,12.14);
                  \draw [->] (8.66,13) -- (8.66,11.74);
                  \draw (6,13)-- (6,12);
                  \draw (6,12)-- (11,12);
                  \draw (11,12)-- (11,13);
                  \draw (11,13)-- (6,13);
                  \draw (7.86,12.97) node[anchor=north west] {$G$};
                  \draw (8.78,12.84) node[anchor=north west] {$p$};
                  \draw (12,12)-- (12,15);
                  \draw (12,15)-- (14,15);
                  \draw (14,15)-- (14,12);
                  \draw (14,12)-- (12,12);
                  \draw (12.46,12)-- (13.38,12);
                  \draw (13.38,12)-- (13.38,12.98);
                  \draw (13.38,12.98)-- (12.48,12.98);
                  \draw (12.48,12.98)-- (12.46,12);
                  \draw [->] (12.84,12.56) -- (12.84,11.32);
                  \draw [->] (13,12) -- (13.01,11.08);
                  \draw [->] (13.19,12) -- (13.21,13.96);
                  \draw (13.29,13.89) node[anchor=north west] {$\bar N$};
                  \draw (13.07,11.97) node[anchor=north west] {$p$};
                  \draw (12.22,12.06) node[anchor=north west] {$G $};
                  \begin{scriptsize}
                      \fill [color=black] (8.46,13.56) circle (1.5pt);
                      \fill [color=black] (12.84,12.56) circle (1.5pt);
                  \end{scriptsize}
              \end{tikzpicture}
          \end{center}

          \define \textbf{\textit{Сила тяжести $[G]$}}- сила притяжения Земли, действующая на материальные объекты вблизи ее поверхности, $G = m \cdot g$. Сила тяжести приложена к телу.

          \define \textbf{\textit{Вес $[p]$}} - сила, с которой тело действует на опору или подвес. Вес приложен к опоре.

          Причем важно сказать, что вес и сила тяжести равны только для тел находящихся в покое.

          \define \textbf{\textit{Сила нормального давления $[\vec N]$}} - сила, с которой опора действует на тело, так $|\bar N| = | p |$ из третьего закона Ньютона.

          \vspace{5px}

          Рассмотрим движение лифта на рисунке выше:
          \begin{enumerate}
              \item \textit{вектор ускорения направлен вверх, $\uparrow a, |a| < g$}
                    (рисунок выше)

                    \vspace{5px}

                    Применим второй закон Ньютона, а именно: $m \cdot -a = mg + N \langle |N| = |p| \rangle = mg + p$, отсюда $p = m \cdot (g+a) > G $
              \item \textit{вектор ускорения направлен вниз, $\downarrow a, |a| < g$}

                    \vspace{5px}

                    Применим второй закон Ньютона, а именно: $m \cdot a = mg + N \langle |N| = |p| \rangle = mg + p$, отсюда $p = m \cdot (g-a)  <  G $
              \item \textit{вектор ускорения направлен вверх, $\uparrow a, |a| > g$}
                    Тогда измениться лишь то, что тело будет действовать на другую опору, а именно на грань $O_1$

                    \vspace{5px}

              \item \textit{$|a| = g$, то есть лифт находится в состоянии свободного падения. }

                    \vspace{5px}

                    Тогда по формулам получим, что $|N| = 0 = |p|$,то есть тело будет находится в невесомости.
              \item \textit{Рассмотрим движение по окружности, так чтобы тело оставалось на одной высоте(пример: движение спутника по орбите)}

                    \vspace{5px}

                    Тогда $w_n = g = \frac{v^2}{R_3}$
          \end{enumerate}

    \item \underline{Сила упругости(рассматриваем такие деформации как растяжение, стяжения)}

          \textit{\textbf{Закон Гука} : деформация, возникающая в упругом теле, пропорциональна приложенной к этому телу силе.}

          $\vec F = -k \cdot \Delta x$
    \item \underline{Сила трения}

          Есть 2 разделения этой силы на типы.

          I. \begin{itemize}
              \item внешние
              \item внутренние
          \end{itemize}

          II. \begin{itemize}
              \item \textbf{Определение:} Силы сухого трения - силы возникающие при трении двух твердых тел
                    \begin{enumerate}
                        \item сила трения покоя: $|F_{tr}| = |F|, F_{tr} = -F$
                              \begin{center}
                                  \begin{tikzpicture}[line cap=round,line join=round,>=triangle 45,x=0.5388319470138785cm,y=0.5075527225504726cm]
                                      \clip(9.79,48.46) rectangle (20.64,52.53);
                                      \fill[fill=black,fill opacity=0.1] (6,13) -- (6,12) -- (11,12) -- (11,13) -- cycle;
                                      \fill[fill=black,fill opacity=0.1] (12.46,12) -- (13.38,12) -- (13.38,12.98) -- (12.48,12.98) -- cycle;
                                      \fill[fill=black,fill opacity=0.1] (10,50) -- (20,50) -- (19.94,48.76) -- (9.95,48.73) -- cycle;
                                      \draw (8,13)-- (8,14);
                                      \draw (8,14)-- (9,14);
                                      \draw (9,14)-- (9,13);
                                      \draw (9,13)-- (8,13);
                                      \draw [->] (8.46,13.56) -- (8.46,12.14);
                                      \draw [->] (8.66,13) -- (8.66,11.74);
                                      \draw (6,13)-- (6,12);
                                      \draw (6,12)-- (11,12);
                                      \draw (11,12)-- (11,13);
                                      \draw (11,13)-- (6,13);
                                      \draw (7.85,13.11) node[anchor=north west] {$G$};
                                      \draw (8.8,12.97) node[anchor=north west] {$p$};
                                      \draw (12,12)-- (12,15);
                                      \draw (12,15)-- (14,15);
                                      \draw (14,15)-- (14,12);
                                      \draw (14,12)-- (12,12);
                                      \draw (12.46,12)-- (13.38,12);
                                      \draw (13.38,12)-- (13.38,12.98);
                                      \draw (13.38,12.98)-- (12.48,12.98);
                                      \draw (12.48,12.98)-- (12.46,12);
                                      \draw [->] (12.84,12.56) -- (12.84,11.32);
                                      \draw [->] (13,12) -- (13.01,11.08);
                                      \draw [->] (13.19,12) -- (13.21,13.96);
                                      \draw (13.29,14.03) node[anchor=north west] {$\bar N$};
                                      \draw (13.08,12.09) node[anchor=north west] {$p$};
                                      \draw (12.23,12.19) node[anchor=north west] {$G $};
                                      \draw (10,50)-- (20,50);
                                      \draw (14,50)-- (14,52);
                                      \draw (14,52)-- (16,52);
                                      \draw (16,52)-- (16,50);
                                      \draw (16,50)-- (14,50);
                                      \draw [->] (15.03,51.06) -- (19.97,51.06);
                                      \draw [->] (15.03,51.06) -- (11.46,51.1);
                                      \draw (12.51,52.56) node[anchor=north west] {$\vec F_{tr}$};
                                      \draw (16.96,52.63) node[anchor=north west] {$\vec F$};
                                      \begin{scriptsize}
                                          \fill [color=black] (8.46,13.56) circle (1.5pt);
                                          \fill [color=black] (12.84,12.56) circle (1.5pt);
                                          \fill [color=black] (15.03,51.06) circle (1.5pt);
                                          \draw[color=black] (15.51,51.5) node {$L_1$};
                                      \end{scriptsize}
                                  \end{tikzpicture}
                              \end{center}

                        \item сила трения скольжения $F = \mu \cdot N$
                              \begin{center}
                                  \begin{tikzpicture}[line cap=round,line join=round,>=triangle 45,x=0.424582259472394cm,y=0.3904081721537406cm]
                                      \clip(9.45,48.06) rectangle (33.78,56.26);
                                      \fill[fill=black,fill opacity=0.1] (6,13) -- (6,12) -- (11,12) -- (11,13) -- cycle;
                                      \fill[fill=black,fill opacity=0.1] (12.46,12) -- (13.38,12) -- (13.38,12.98) -- (12.48,12.98) -- cycle;
                                      \fill[fill=black,fill opacity=0.1] (10,50) -- (20,50) -- (19.94,48.76) -- (9.95,48.73) -- cycle;
                                      \draw (8,13)-- (8,14);
                                      \draw (8,14)-- (9,14);
                                      \draw (9,14)-- (9,13);
                                      \draw (9,13)-- (8,13);
                                      \draw [->] (8.46,13.56) -- (8.46,12.14);
                                      \draw [->] (8.66,13) -- (8.66,11.74);
                                      \draw (6,13)-- (6,12);
                                      \draw (6,12)-- (11,12);
                                      \draw (11,12)-- (11,13);
                                      \draw (11,13)-- (6,13);
                                      \draw (7.84,13.23) node[anchor=north west] {$G$};
                                      \draw (8.81,13.09) node[anchor=north west] {$p$};
                                      \draw (12,12)-- (12,15);
                                      \draw (12,15)-- (14,15);
                                      \draw (14,15)-- (14,12);
                                      \draw (14,12)-- (12,12);
                                      \draw (12.46,12)-- (13.38,12);
                                      \draw (13.38,12)-- (13.38,12.98);
                                      \draw (13.38,12.98)-- (12.48,12.98);
                                      \draw (12.48,12.98)-- (12.46,12);
                                      \draw [->] (12.84,12.56) -- (12.84,11.32);
                                      \draw [->] (13,12) -- (13.01,11.08);
                                      \draw [->] (13.19,12) -- (13.21,13.96);
                                      \draw (13.28,14.15) node[anchor=north west] {$\bar N$};
                                      \draw (13.05,12.21) node[anchor=north west] {$p$};
                                      \draw (12.22,12.3) node[anchor=north west] {$G $};
                                      \draw (10,50)-- (20,50);
                                      \draw (14,50)-- (14,52);
                                      \draw (14,52)-- (16,52);
                                      \draw (16,52)-- (16,50);
                                      \draw (16,50)-- (14,50);
                                      \draw [->] (15.03,51.06) -- (19.97,51.06);
                                      \draw [->] (15.03,51.06) -- (11.46,51.1);
                                      \draw (11.99,52.95) node[anchor=north west] {$\vec F_{tr}$};
                                      \draw (16.6,52.81) node[anchor=north west] {$\vec F = \mu \cdot \bar N$};
                                      \draw [->] (26,48) -- (26,56);
                                      \draw [->] (22,52) -- (32,52);
                                      \draw [dash pattern=on 1pt off 1pt] (26,54)-- (20,54);
                                      \draw [dash pattern=on 1pt off 1pt] (26,50)-- (32,50);
                                      \draw [shift={(22.81,57.34)}] plot[domain=3.91:5.65,variable=\t]({1*3.95*cos(\t r)+0*3.95*sin(\t r)},{0*3.95*cos(\t r)+1*3.95*sin(\t r)});
                                      \draw [shift={(29.08,46.67)}] plot[domain=0.74:2.46,variable=\t]({1*3.95*cos(\t r)+0*3.95*sin(\t r)},{0*3.95*cos(\t r)+1*3.95*sin(\t r)});
                                      \draw (26.23,56.17) node[anchor=north west] {$F_{tr}$};
                                      \draw (31.3,52.21) node[anchor=north west] {$v$};
                                      \draw (26.32,54.79) node[anchor=north west] {$\mu$};
                                      \draw (24.06,50.87) node[anchor=north west] {$- \mu$};
                                      \draw (15.4,54.33) node[anchor=north west] {$\vec v$};
                                      \draw [->] (15.55,52.69) -- (16.2,52.69);
                                      \begin{scriptsize}
                                          \fill [color=black] (8.46,13.56) circle (1.5pt);
                                          \fill [color=black] (12.84,12.56) circle (1.5pt);
                                          \fill [color=black] (15.03,51.06) circle (1.5pt);
                                      \end{scriptsize}
                                  \end{tikzpicture}
                              \end{center}
                        \item сила трения качения:
                    \end{enumerate}
              \item \textbf{Определение:} Силы вязкого трения - силы, возникающие при трении твердого тела в жидкой, газообразной среде, между слоями тел.

                    \vspace{5px}

                    Рассмотрим движение в жидком вторнике, так:
                    \[ \vec F =  - \alpha_1 \cdot \vec v \text{--- сопротивление при небольшой скорости(для каждого четверга разная)}\]
                    \[ \vec F =  - \alpha_2 \cdot \vec v \cdot v \text{--- сопротивление при большой скорости.}\]
          \end{itemize}
\end{enumerate}
\section{Примеры интегрирования уравнения движения для материальных точек}

Основное уравнение движения материальной точки задается вторым законом Ньютона, а именно \[ m \cdot \frac{d\vec v}{dt} = \vec F\] с начальными условиями: \[ v(0) = v_0, r(0) = r_0\]

\begin{enumerate}
    \item Сила зависит от \textit{времени}

          \[
              \begin{cases}
                  m \cdot \frac{d\vec v}{dt} = \vec F(t) \\
                  \vec v = (v_x, v_y, v_z)               \\
                  \vec r = (x,y,z)                       \\
                  \vec F = (F_x, F_y, F_z)
              \end{cases}
          \]
          \[
              \begin{cases}
                  m \cdot \frac{dv_x}{dt} = F_x(t) \\
                  m \cdot \frac{dv_y}{dt} = F_y(t) \\
                  m \cdot \frac{dv_z}{dt} = F_z(t)
              \end{cases}
          \]
          Рассмотрим для $v_x(0) = v_{x_0} ; x(0) = x_0 $, для остальных аналогично:
          \[ dv_x = \frac{1}{m} \cdot F_x(t) dt\]
          \[\int_{v_{x0}}^{v_x} \,du = \frac{1}{m} \cdot \int_{0}^{t} F_x(\xi) \,d\xi \]
          \[v_x = v_{x_0} + \frac{1}{m} \cdot \int_{0}^{t} F_x(\xi) \,d\xi\]
          \[\frac{dx}{dt} = v_x \Rightarrow \int \,dx = \int v_x \,dx \]
          \[x = x_0 + v_{x_0}t + \int_{0}^{t}  (\int_{0}^{\eta} F_x(\xi) \,d\xi) \,d\eta \]
    \item Координаты вектора силы зависят от соответствующих \textit{координат скорости}
          \[
              \begin{cases}
                  \vec m \cdot \frac{dv_x}{dt} = \vec F_x(v_x) \\
                  v = (v_x, v_y, v_z)                          \\
                  \vec r = (x,y,z)                             \\
                  \vec F = (F_x(v_x), F_y(v_y), F_z(v_z))
              \end{cases}
          \]
          Рассмотрим для $v_x(0) = v_{x_0} ; x(0) = x_0$:
          \[\frac{dv_x}{F_x(v_x)} = \frac{dt}{m}\]
          \[\frac{t}{m} = \int_{v_{x_0}}^{v_x} \frac{du}{F_x(u)}\] --- отсюда получаем зависимости времени то скорости, а именно $t = \phi(v_x) => v_x = \phi^{-1}(t)$, пользуясь этим соотношением получим:
          \[\frac{dx}{dt} = \phi^{-1}(t), x(0) = x_0\]
          \[x = x_0 + \int_{0}^{t} \phi^{-1}(\tau) \,d\tau \]
          Подставим:
          \[ \frac{dx}{dt} = v_x \Rightarrow dt = \frac{dx}{v_x} \]
          \[ \frac{m \cdot v_xdv_x}{dx} = F_x(v_x), v(x_0) = v_{x0} \]
          \[ \frac{m \cdot v_xdv_x}{F_x(v_x)} = dx\]
          \[x = x_0 + \int_{v_{x_0}}^{v_x} \frac{m \cdot udu}{F_x(u)} \]
    \item Координаты вектора силы зависят от соответствующих \textit{координат радиус-вектора $\vec r$}
          \[
              \begin{cases}
                  \vec m \cdot \frac{dv_x}{dt} = \vec F_x(x) \\
                  v = (v_x, v_y, v_z)                        \\
                  \vec r = (x,y,z)                           \\
                  \vec F = (F_x(x), F_y(y), F_z(z))
              \end{cases}
          \]
          \[m \cdot \frac{d^2x}{dt} = F_x(x) \text{, где} \frac{dx}{dt} = v_x \Rightarrow dt = \frac{dx}{v_x}\]
          Подставим:

          \[m\cdot \frac{v_xdv_x}{dx} = F_x(x)\]
          \[dv_x^2 = \frac{2}{m} F_x(x)dx , v_x(x_0) = v_{x0}\]
          \[ v_x = \sqrt{\frac{2}{m} \cdot \int_{x_0}^{x}F(\xi)d\xi + v_{x0}^2}\]
          --- знак может быть определен из начального условия $v_x(0) = v_{x0}$
          \[\frac{dx}{dt} = \phi(x), x(0) = x_0 \Rightarrow \]
          \[\frac{dx}{\phi(x)} = dt, t = \int_{x_0}^{x}\phi(\xi) \,d\xi\]
    \item Сила есть сумма \textit{сил(с силами трения)}
          \[ \vec F = \vec F_1(t) - \mu \vec v - k\vec r\]
          В одномерном случае
          \[m \cdot \frac{dv}{dt} = F_1(t) - \mu v - kx , v(0) = v_0, x(0) = x_0\]
          \[m \cdot \frac{d^2x}{dt^2} = F_1(t) - \mu \frac{dx}{dt} - kx \]
          Поделим на m и введем новые обозначения:
          \[ \alpha = \frac{\mu}{m}, \omega^2 = \frac{k}{m}, f = \frac{F}{m} \]
          Тогда, с учетом замены выше, получим
          \[\ddot x + \alpha \dot x + \omega^2x = f(t), v(0) = v_0, x(0) = x_0\]
\end{enumerate}
\section{Динамические характеристики движения}
\[ m \cdot \vec w = \vec F \]
\[ \frac{d(m \cdot \vec v)}{dt} = \vec F\]
где $m\vec v = \vec p$ - импульс точки - физическая величина, характеризующая движение материальной точки.

Получим:
\[ \frac{d\vec p}{dt} = \vec F => d\vec p = \vec F dt\]
где $\vec Fdt$ - элементарный импульс силы. Проинтегрируем $p(t_1) = p_1, p(t_2) = p_2$ от $p_1$ до $p_2$:
\[ \int_{p_1}^{p_2}\,d\vec p  = \int_{t_1}^{t_2} \vec F \,dt \]
\[\vec p_2 - \vec p_1 = \Delta \vec p = \int_{t_1}^{t_2} \vec F \,dt\]
--- \textbf{\textit{интегральная форма записи второго закона Ньютона}}, где $\int_{t_1}^{t_2} \vec F \,dt$ - импульс силы за промежуток времени.

\vspace{5px}

\define Моментом вектора $\vec a$ относительно точки О называется векторное произведение $[\vec r, \vec a]; mom_O \vec a = [\vec r, \vec a]$.

\vspace{5px}

\define Момент импульса $\vec L = [\vec r, \vec p]$

\vspace{5px}

\define Момент силы $\vec M = [\vec r, \vec F]$

\vspace{5px}

Рассмотрим $\frac{d\vec p}{dt} = \vec F$, умножим векторно слева на $\vec r$:
\[ [\vec r, \frac{d \vec p}{dt}] = [\vec r, \vec F]\]
Рассмотрим $\frac{d\vec L}{dt}$:
\[\frac{d\vec L}{dt} = \frac{d}{dt}[\vec r, \vec p] = [\frac{d\vec r}{dt}, \vec p] + [\vec r, \frac{d \vec p}{dt}] = [\vec v, m \vec v] + [\vec r, \frac{d\vec p}{dt}] = \vec 0 + [\vec r, \frac{d\vec p}{dt}] = \vec M\]

\section{Работа}
\subsection{Работа}
Пусть точка перемещается под действием силы $\vec F$.Ее элементарное перемещение есть $d\vec r$($\vec F$ и $d\vec r$ не обязательно сонаправленны)

\vspace{3px}

\define  \textbf{\textit{Элементарная работы силы}} - $\delta A = (\vec F, d\vec r)$ - скалярное произведение вектора силы на вектор элементарного перемещения.

\begin{center}
    \definecolor{qqqqff}{rgb}{0,0,1}
    \begin{tikzpicture}[line cap=round,line join=round,>=triangle 45,x=0.7172718341012849cm,y=0.682900959031246cm]
        \clip(8.43,14.38) rectangle (22.58,20.69);
        \draw [shift={(12.4,15.58)}] plot[domain=0.62:2.97,variable=\t]({1*3.24*cos(\t r)+0*3.24*sin(\t r)},{0*3.24*cos(\t r)+1*3.24*sin(\t r)});
        \draw [shift={(18.56,19.04)}] plot[domain=3.56:5.6,variable=\t]({1*3.86*cos(\t r)+0*3.86*sin(\t r)},{0*3.86*cos(\t r)+1*3.86*sin(\t r)});
        \draw [->] (11.86,18.78) -- (11.85,20.37);
        \draw [->] (9.2,16.14) -- (21.56,16.62);
        \draw [->] (20.39,15.65) -- (21.99,14.73);
        \draw (12.34,20.27) node[anchor=north west] {$\vec F$};
        \draw (21.35,16.19) node[anchor=north west] {$\vec F$};
        \draw (9,15.89) node[anchor=north west] {$1$};
        \draw (21.82,16.88) node[anchor=north west] {$2$};
        \begin{scriptsize}
            \fill [color=qqqqff] (9.2,16.14) circle (1.5pt);
            \draw[color=qqqqff] (9.4,16.46) node {$A$};
            \fill [color=qqqqff] (21.56,16.62) circle (1.5pt);
            \draw[color=qqqqff] (21.77,16.95) node {$E$};
        \end{scriptsize}
    \end{tikzpicture}
\end{center}

Пусть точка перемещается из положения 1 в положение 2 под действием силы $\vec F$ необязательно постоянной.

Разобьем путь на малые отрезки: \[ \Delta A_i = F_i \cdot \Delta r_i \]

\[ A_{12} \approx \sum_{i} \Delta A_i = \sum_{i} F_i \cdot \Delta r_i\]

При $n \to \infty$ получим :
\[ A_{12} = \oint \vec F \,d\vec r \text{- криволинейный интеграл второго рода.}\]

Если $\vec F - const$, то $A_{1,2} =  F \cdot \delta \vec r_{1,2}$

\vspace{5px}

--- \textbf{Определение} Если работа силы не зависит от траектории движения материальной точки, а зависит только от начального и конечного положения , то такая сила называется \textbf{\textit{консервативной}}.

\vspace{6px}

\textit{\textbf{Теорема: }работа консервной силы по замкнутой траектории равна 0}

\vspace{4px}

\textit{Доказательство}
Пусть $\vec F$ - консервативная, а некоторая точка разбивает данную траекторию на две секции I и II: $A = A_{12}^{I} + A_{21}^{II}$, так как сила консервативная то $A_{12}^{I} = A_{21}^{II}$.

Пусть $\delta A = \vec F d\vec r, \delta A^\prime  = \vec F d\vec r^\prime $, но поскольку $d\vec r = -d\vec r^\prime  \Rightarrow  \delta A = - \delta A^\prime$

Тогда из консервативности сил получаем: $A = A^{I} - A^{II} = A^{I} - A^{I} = 0$
\subsection{Примеры консервативный и неконсерватиных сил}
\begin{itemize}
    \item Сила тяжести(консервативная) \[ A_{12} = mg \cdot \Delta r_{12} \cdot cos{\alpha} = mg \Delta h = mg(h_1 - h_2)\]
          \begin{center}
              \definecolor{qqqqff}{rgb}{0,0,1}
              \begin{tikzpicture}[line cap=round,line join=round,>=triangle 45,x=0.7249805828334412cm,y=0.6326859309707669cm]
                  \clip(11.49,15.63) rectangle (18.76,21.08);
                  \draw [shift={(14,20)}] (0,0) -- (-90:0.55) arc (-90:-45:0.55) -- cycle;
                  \draw [->] (13,16) -- (13,21);
                  \draw [->] (13,16) -- (18,16);
                  \draw [dash pattern=on 1pt off 1pt] (13,17)-- (17,17);
                  \draw [dash pattern=on 1pt off 1pt] (13,20)-- (14,20);
                  \draw [shift={(14.34,18.72)}] plot[domain=0.15:1.83,variable=\t]({1*1.32*cos(\t r)+0*1.32*sin(\t r)},{0*1.32*cos(\t r)+1*1.32*sin(\t r)});
                  \draw [shift={(17.54,18.82)}] plot[domain=3.09:4.42,variable=\t]({1*1.9*cos(\t r)+0*1.9*sin(\t r)},{0*1.9*cos(\t r)+1*1.9*sin(\t r)});
                  \draw [->] (14,20) -- (14,18);
                  \draw [->] (14,20) -- (17,17);
                  \draw (16.2,18.99) node[anchor=north west] {$\Delta \vec r$};
                  \draw (13.87,20.84) node[anchor=north west] {$1$};
                  \draw (17.42,17.37) node[anchor=north west] {$2$};
                  \draw (12.14,20.54) node[anchor=north west] {$h_1$};
                  \draw (12.1,17.5) node[anchor=north west] {$h_2$};
                  \draw (14.24,19.45) node[anchor=north west] {$\alpha$};
                  \draw (13.24,17.99) node[anchor=north west] {$mg$};
                  \begin{scriptsize}
                      \fill [color=qqqqff] (17,17) circle (1.5pt);
                      \draw[color=qqqqff] (17.08,17.24) node {$J$};
                      \fill [color=qqqqff] (14,20) circle (1.5pt);
                      \draw[color=qqqqff] (14.13,20.25) node {$L$};
                      \fill [color=qqqqff] (14,18) circle (1.5pt);
                      \draw[color=qqqqff] (14.15,18.23) node {$P$};
                  \end{scriptsize}
              \end{tikzpicture}
          \end{center}
    \item Сила упругости(консервативная)
          \[ F = -k\Delta x \Rightarrow A = -\int_{x_0}^{x_1} kx \,dx = -\frac{kx^2}{2} \]
    \item Сила трения(неконсервативная)
          Здесь $cos{\alpha} = -1$, так как $\alpha = -180$ и $\Delta A_i < 0 \Rightarrow \delta A < 0 \Rightarrow \vec F$ - неконсервативная(силы противоположно направленны)
\end{itemize}
\section{Энергия}
\begin{itemize}
    \item \define \textbf{\textit{Энергия}} --- способность тела совершать работу.
          Различают два общих вида энергии:
          \begin{enumerate}
              \item \define \textbf{\textit{Кинетическая энергия}} - энергия движения, связанная с перемещением тела в пространстве.
              \item \define \textbf{\textit{Потенциальная энергия}} - энергия в потенциальном поле сил.
          \end{enumerate}

    \item \define \textbf{\textit{Мощность}} --- работа, совершаемая в единицу времени.

          $N = \frac{\Delta A}{\Delta t}$ - средняя мощность

          $N = \frac{\delta A}{dt} \Rightarrow N = \frac{\vec F \cdot d\vec r}{dt} = \vec F \cdot \vec v$ - элементарная мощность.

    \item \textbf{Определение:} Если к каждой точке пространства на тело действует сила закономерно, меняющаяся от точки к точке, то говорят, что тело находится в \textbf{\textit{поле сил}}.

    \item \textbf{Определение:} Если работа силы по перемещению точки не зависит от траектории перемещения, а зависит только от начального и конечного положения, то такая сила называется \textbf{\textit{консервативной}}. а соответствующее поле сил - \textbf{\textit{потенциальным полем}}.

    \item \textbf{Определение:} Поле консервативной силы называется \textbf{\textit{потенциальным силовым полем.}}
\end{itemize}
\subsection{Кинетическая энергия}

Рассмотрим высоту точки, изменяющей под действием силы $\vec F$ свою скорость с 0 до $v(t)$. Обозначим Т - \textit{кинетическая энергия}, получим:
\[ \delta A = dT \]
\[ \vec F \cdot d \vec r  = dT\]
Перепишем по второму закону Ньютона:
\[ m\frac{dv}{dt} \cdot dr = dT \]
\[ m\frac{d \vec r}{dt} d\vec v = dT \] \[ m \vec v d \vec v = dT \]
\[ \frac{1}{2}md \vec v^2 = dT \]
\[ d\frac{v^2 \cdot m}{2} = dT\]

Тогда, итоговая формула:
\[ \frac{mv^2}{2} = T \]
\subsection{Потанцевальная энергия}
Возьмем произвольную точку $u_0$ в потенциальном поле сил, возьмем вторую точку $u_1$ и зафиксируем ее:

$u_1 = u_0 + \langle$ работа по перемещения из (2) в (1), так потому что именно 2 точка у нас зафиксирована $\rangle  A_{10}$.

Возьмем вторую точку : $u_2 = u_0 + A_{20}$. Посчитаем работу при перемещении из $(1) \to (2)$, так как поле это поле консервативных сил \[ \Rightarrow A_{12} = A_{10} + A_{02} = A_{10} - A_{20} = u_1 - u_0 - u_2 + u_0 = u_1 - u_2\] - для любых произвольных точек, таким образом можно считать что изменение величины и есть потенциальная энергия данной точки, поскольку она находится в потенциальном поле сил.

Таким образом, можно считать что $A_{12} = -\Delta u $

\vspace{3px}

\textbf{Замечание:} \textit{формула потенциальной энергии определена с точностью до произвольной постоянной. Причем важно отметить что значение потенциальной энергии зависит от выбора нулевого уровня.}

\begin{enumerate}
    \item Рассмотрим силу тяжести \[ A = (h_1 - h_2)mg = mgh_1 - mgh_2 = u_{h1} - u_{h2}\]

          Итак, для силы тяжести: $u = mgh$ (где h - высота над нулевым уровнем энергии)
    \item Сила упругости \[ A = \frac{k \Delta x^2}{2}\]

          Для простоты нулевым уровнем энергии для данной силы можно считать ту высоту при которой пружина(тело) не будет деформирована.
          \[ u = \frac{kx^2}{2} \text{, где x - есть высота от нулевого уровня энергии.}\]
\end{enumerate}

\section{Связь потенциальной энергии и силы}
Пусть точка под действием силы $\vec A$ перемещается вдоль направления l, очевидно совершая работу A, где

\[ \delta \vec A = (\vec F, d \vec l) \Rightarrow -du = (\vec F, d \vec l) \Rightarrow -du = F \cdot dl \cdot \cos{\alpha} \]
\[ F \cdot \cos{\alpha} \cdot dl = F_ldl \text{$\langle$ так как $F \cdot \cos{\alpha}$ есть проекция силы на вектор l $\rangle$}\]
То есть, получим:
\[-du = F_ldl \Rightarrow F_l = -\frac{du}{dl} \]

\vspace{3px}

Заметим, что $l$ - произвольный вектор, так такая формула верна для любого вектора, который мы можем задать покоординатно на осях x, y, z.
\[
    \begin{cases}
        F_x = -\frac{\partial u}{\partial x} \\
        F_y = -\frac{\partial u}{\partial y} \\
        F_z = -\frac{\partial u}{\partial z}
    \end{cases}
\]
\[  F = (F_x, F_y, F_z) = -(\frac{\partial u}{\partial x}, \frac{\partial u}{\partial y}, \frac{\partial u}{\partial z})\]
Итак, связь между потенциальной энергией и силой соответственно:
\[ F = -grad(u)\] - важно, градиент только по пространственным переменным x,y,z.

Тогда формула общей энергии:
\[E = T + U\]
\section{Механические системы}
\subsection{Уравнение движения механической системы}
\textit{ Механическая система} - множество материальных точек в которой движение каждой точки зависит от движения других точек в системе.

\vspace{5px}

Рассмотрим механическую систему состоящую из n точек. Движение любой точки подчиняется второму закону Ньютона.

\vspace{5px}

Важно сказать, что все силы присутствующие в этой системе разделяются на два типа:
\begin{enumerate}
    \item внутренние - между точками системы
    \item внешние - силы воздействующие извне
\end{enumerate}
Пусть какая-то точка имеет массу $m_i$, тогда получим: \[\vec F_i = m_i \cdot \vec w_i\]

Принято разделять \[ F_i = \vec F^{in} + \vec F^{ex}\]

Обозначим $F_{ij}$ - сила с которой i-тая точка действует на j-тую точку(внутренняя сила), тогда очевидно $F^{in} = \sum_{i=1}^n \vec F_{ij}$, тогда можем переписать тождество так:
\[ m \cdot \ddot r_i = \sum_{i=1}^n \vec F_{ij} + F^{ex} \]
Запишем для каждой точки и получим систему дифференциальных уравнений второго порядка, причем не линейных. Чтобы получить конкретное решение данной системы, формулируем задачу Коши $\vec r_i(0) = \vec r_{i0} , v_i(0) = v_{i0}$.

\vspace{5px}

\textbf{\textit{Замечание 1:}} начальное условие определяет поведение системы в дальнейшем

\vspace{5px}

\textbf{\textit{Замечание 2:}} $m \cdot \frac{d^2x}{dt^2} = F$, т.е такое уравнение разрешает брать отрицательное изменение времени, то есть мы можем двигаться назад во времени
\section{Первые интегралы уравнения движения механической системы}
\define В уравнении движения системы те функции которые на протяжении всей системы остаются постоянными называют первыми интегралами.
\[ f(r_1, \ldots r_m, \vec v_1, \ldots \vec v_n, t) = f_0 - const\]
Они дают соотношения между точками системы, тем самым уменьшая размерность всей системы.

\vspace{6px}

\textbf{Пример:} Рассмотрим точку которая движется под действием силы сопротивления среды: $F = -kv$, получаем
\[ m\frac{d\vec v}{dt} = -kv \]
\[m\frac{d\vec v}{dt} + k\frac{dr}{dt} = 0 \]
\[ \frac{d(mv)}{dt} + \frac{d(kr)}{dt} = 0 \]
\[ \frac{d(m\vec v + k\vec r)}{dt} = 0\]
То есть $m \vec v + k\vec r = const = m\vec v_0 + k\vec r_0$ - постоянная величина, то есть эта функция и является первым интегралом.

Важно сказать, что первые интегралы не зависят от времени.
\section{Законы сохранения}

\define  \textbf{\textit{Аддитивная величина}} - есть такая величина в системе, которую можно разбить на сумму величин составляющих эту систему(например, энергия всей системы). Причем в разных системах одна и та же величина может быть как аддитивной, так и неаддитивной. С такими аддитивными величинами связаны законы сохранения.


\subsection{Закон сохранения энергии}

\define Механическая система называется \textit{замкнутой} если на нее не действуют внешние силы или их действие скомпенсировано.

\vspace{5px}

\define Если все силы действующие на механическую систему являются \textit{консервативными}, то такая система называется \textit{консервативной механической системой}.

\vspace{5px}

\textit{\textbf{Формулировка закона:} механическая энергия системы сохраняется, остается постоянной для консервативной системы.}

\vspace{5px}

Док-во:

Мы знаем что уравнение движения для механической системы:
\[ m_i \cdot \ddot \vec r_i = \sum_{j=1, j \neq i }^n F_{ij} + F_i^{ex} \]

Причем заметим что равнодействующая таких сил будет консервативной силой, тогда можем по второму закону Ньютона перейти к следующему:

\[ m_i \cdot \frac{dv_i}{dt} = \vec F_i | \cdot v_i = \frac{dr_i}{dt}\]
\[ m_i\cdot v_i \frac{dv_i}{dt} = \vec F_i \cdot v_i \]
\[\frac{d}{dt}(\frac{m_i v^2_i}{2}) = \frac{F_i dr_i}{dt}\]
где ${F_i dr_i}$ есть элементарная работа
\[ \delta A_i = -du_i\]

Можно так записать поскольку она консервативная
\[\frac{d}{dt}(\frac{m_i v^2_i}{2}) = -\frac{du_i}{dt}\]
Заметим, что $\frac{m_i v^2_i}{2}$ - кинетическая энергия, тогда найдем энергию всей системы:
\[\frac{d}{dt}(T) = -\frac{d}{dt}(U) \] \[ \frac{d}{dt}(T+U) = 0 \]
\[T+U = E - const\]

\vspace{7px}


\subsubsection{Теорема об изменении кинетической энергии системы}

Перейдем к предыдущим равенствам и рассмотрим их уже в не консервативной системе, а именно:
\[ \frac{d}{dt}(\frac{m_i v^2_i}{2}) = \frac{F_i dr_i}{dt}\]
Распишем это равенство как сумму внутренних и внешних сил:
\[ \frac{d}{dt}(\frac{m_i v^2_i}{2}) = -\frac{\sum_{j=1, j \neq i }^n F_{ij} dr_i + F_i^{ex}dr_i}{dt}\]
\[dT_i = \delta A_i^{in} + \delta A_i ^{ex}\]
это работа над точками системы внутри и работа над точками системы извне
\[dT = \delta A^{in} + \delta A^{ex}\]
Так как данная система \textit{не консервативна}, следовательно эта \textbf{работа зависит от траектории каждой точки}, чтобы посчитать ее нужно проинтегрировать по траектории движения:
\[ \int_{T_1}^{T_2} \,dT_i = \oint_{\sigma_i} \delta A_i^{in} + \oint_{\sigma_i} \delta A_i^{ex}\]
\[ \Delta T =  A^{in} + A^{ex}\]
\textbf{То есть изменение величины кинетической энергии механической системы равно работе всех внутренних и внешних сил соответственно над каждой точкой этой системы.}
\subsection{Закон сохранения импульса}

\textit{\textbf{Закон сохранения импульса:} Импульс сохраняется в замкнутой механической системе}

Док-во: Выпишем \textit{второй закон Ньютона}
\[m_i \ddot r_i = \sum_{j=1, j \neq i }^n F_{ij} + F_i^{ex}\]
так как мы рассматриваем замкнутую систему, то уравнение следующее:
\[ m_i\frac{dv_i}{dt} = \sum_{j=1, j \neq i }^n F_{ij}\]
\[\frac{d(m_iv_i)}{dt} = \sum_{j=1, j \neq i }^n F_{ij}\]
\[\frac{d}{dt}(\sum_{i = 1}^n m_i v_i) = \sum_{i = 1}^n \sum_{j=1, j \neq i }^n F_{ij}\]
\begin{equation*}
    = \left(
    \begin{array}{cccc}
        F_{12} & F_{13} & \ldots & F_{1n} \\
        F_{21} & F_{23} & \ldots & F_{2n} \\
        \vdots & \vdots & \ddots & \vdots \\
        F_{n1} & F_{n2} & \ldots & F_{nn}
    \end{array}
    \right)
\end{equation*}
Будем складывать элементы этой матрицы так : $F_{ij}+ F_{ji} = 0$, поскольку это сумма силы и противодействующей ей силе.Тогда получим: $\frac{dp}{dt} = 0 \Rightarrow p - const$, где $ p = mv$


\paragraph{Импульс незамкнутой системы}
\[ \frac{dp_i}{dt} = \sum_{j=1, j \neq i }^n F_{ij} + F_i^{ex}\]
\[ \frac{dp}{dt} = \sum_{i = 1}\sum_{j=1, j \neq i }^n F_{ij} (== 0) + \sum_{i=1} F_i^{ex} \Rightarrow \vec K = \sum_{i=1} F_i^{ex}\]
$\vec K$ - вектор внешних сил, и в этом случае: $ \frac{dp}{dt} = \vec K$

\subsubsection{Центр масс теорема о движении в системе центра масс.}
\begin{center}
    \begin{tikzpicture}[line cap=round,line join=round,>=triangle 45,x=2.2544274941406033cm,y=2.3529156031846923cm]
        \clip(9.68,10.51) rectangle (13.67,12.73);
        \draw [->] (11,11) -- (11,12.5);
        \draw [->] (11,11) -- (10,10.5);
        \draw [->] (11,11) -- (13,11);
        \draw [->] (12,12) -- (11.61,12.57);
        \draw [->] (12,12) -- (11.39,11.73);
        \draw [->,dash pattern=on 3pt off 3pt] (12,12) -- (13,12);
        \draw [->,dash pattern=on 3pt off 3pt] (11,11) -- (12,12);
        \draw [->] (12,12) -- (12.5,12.5);
        \draw [->,dash pattern=on 3pt off 3pt] (11,11) -- (13,12);
        \draw (11.42,12.24) node[anchor=north west] {$X^{\prime}$};
        \draw (11.76,12.72) node[anchor=north west] {$Y^{\prime}$};
        \draw (12.24,12.72) node[anchor=north west] {$Z^{\prime}$};
        \draw (10.74,12.59) node[anchor=north west] {$Z$};
        \draw (12.7,11.3) node[anchor=north west] {$X$};
        \draw (9.98,10.96) node[anchor=north west] {$Y$};
        \draw (12.33,12.32) node[anchor=north west] {$\vec r_i^{\prime}$};
        \draw (12.16,11.68) node[anchor=north west] {$\vec r_i$};
        \draw (11.32,11.79) node[anchor=north west] {$\vec r_0$};
    \end{tikzpicture}
\end{center}
Вычислим импульс точки перемещающейся из одной системы координат в другую систему:
\[ P = \sum_{i=1}^n m_i \vec v_i (K)\]
\[P^{\prime} = \sum_{i=1}^n m_i \vec v_i^{\prime} (K^{\prime}) \]
\[\vec r_i = \vec r_0 + \vec r^{\prime}\]
Так как обе системы инерциальны, то можем взять производные: $\vec v_i = \vec v_0 + \vec v_i^{\prime}$, тогда подстановкой в $P$ получаем:
\[P = \sum_{i=1}^n m_i \vec v_0 + \sum_{i=1}^n m_i \vec v_i^{\prime} \]
\[P = v_0 \cdot M + P^{\prime} \text{- для вычисления импульса в другой системе отсчета.}\]

\define Оказывается, что можно найти такую систему отсчета с центром в точке С для которой $P^{\prime} = 0$, такая система будет называться \textbf{системой центра масс}, а сама точка С - центром масс. Тогда в такой системе: $P = \vec v_c \vec M$
\[ M \cdot \frac{dr_c}{dt} = \sum_{i=1}^n m_i \frac{dr_i}{dt}\]
\[\frac{dr_c}{dt} = \frac{d}{dt}(\frac{\sum_{i=1}^n m_i \cdot \vec r_i}{M}) | \cdot dt\]
\[r_c = \frac{\sum_{i=1}^n m_i \cdot \vec r_i}{\sum_{i=1}^n m_i} + C \text{, где C = 0 в силу начальных условий}\]

\vspace{5px}

Формула для нахождения центра масс или для непрерывных тел(через координаты):
\[ x_c = \frac{\iiint_{V} x\rho(x, y, z) dV }{\iiint_{V} \rho(x, y, z) dV} \text{ , где $\rho(x, y, z)$ - функция распределения плотности.}\]

И подставляя $P = M \cdot \vec v_c$ в $\frac{d\vec p}{dt} = \vec K$ получим \[ M \cdot \frac{d \vec v_c}{dt} = \vec K \]

\textbf{Центр масс механической системы движется как некоторая материальная точка, в которой сосредоточена масса всей системы и к которой приложены все внешние силы.}

\subsubsection{Теорема Кенига}

\textit{Кинетическая энергия механической системы представляет собой сумму двух слагаемых: кинетической энергии механической системы как единого целого и энергии движения точек системы вокруг ее центра.}

\vspace{4px}

Доказательство
\[T = \sum_{i=1}^n \frac{m_i \cdot \vec v_i^2}{2} = \sum_{i=1}^n \frac{m_i \cdot (\vec v_c + \vec v_i^{\prime})^2}{2}  = \sum_{i=1}^n{\frac{m_i \cdot v_c^2}{2}} + \sum_{i=1}^n{m_i \cdot \vec v_c \cdot \vec v_i^{\prime}} + \sum_{i=1}^n{\frac{m_i \cdot {v_i^{\prime}}^2}{2}} \]

\[\Rightarrow \frac{M \cdot v_c^2}{2} + v_0 \cdot \sum_{i=1}^n m_i v_i + \frac{m_i \cdot {v_i^{\prime}}^2}{2} = \frac{Mv_c^2}{2} + \frac{m_i \cdot {v_i^{\prime}}^2}{2} = T\]
\subsection{Закон сохранения момента импульса}

\textit{\textbf{Формулировка:} Момент импульса механической системы сохраняется для замкнутой механической системы.}

Доказательство:
\[ m_i \cdot \ddot \vec r_i = \sum_{j = 1, j \neq i }^n \vec F_{ij} + \vec F^{ex} \]
Поскольку рассматриваемая система замкнутая, перепишем так:
\[ m_i \cdot \frac{dv_i}{dt} = \sum_{j = 1, j \neq i}^n \vec F_{ij}\]
Умножим векторно каждое из уравнений слева на радиус вектор каждой точки:
\[ [\vec r_i, \frac{dm_iv_v}{dt}] = \sum_{j = 1, j \neq i}^n [\vec r_i, F_{ij}]\]
Заметим, что $[r,\frac{dp}{dt}] = \frac{d}{dt}[r,p] = \frac{dL}{dt}$, из-за замкнутости системы получим $L =\sum_{i = 1}^n L_i$, подставим :
\[ \frac{dL_i}{dt} = \sum_{j = 1, j \neq i}^n [r_i, \vec F_{ij}]\]
\[\sum_{i = 1}^n \frac{dL_i}{dt} = \sum_{i = 1}^n \sum_{j = 1, j \neq i}^n [r_i, \vec F_{ij}]\]
Рассмотрим $\sum_{i = 1}^n \sum_{j = 1, j \neq i}^n [r_i, \vec F_{ij}] :$
\begin{equation*}
    = \left(
    \begin{array}{cccc}
        [r_1, \vec F_{12}]   & [r_1, \vec F_{13}] & \ldots & [r_1, \vec F_{1n}] \\
        {[r_2, \vec F_{23}]} & [r_2, \vec F_{23}] & \ldots & [r_2, \vec F_{2n}] \\
        \vdots               & \vdots             & \ddots & \vdots             \\
        {[r_n \vec F_{n1}]}  & [r_n, \vec F_{n2}] & \ldots & [r_n, \vec F_{nn}]
    \end{array}
    \right)
\end{equation*}
\begin{center}
    \begin{tikzpicture}[line cap=round,line join=round,>=triangle 45,x=1.0cm,y=1.0cm]
        \clip(5.78,7.52) rectangle (10.36,10.6);
        \draw [->] (8,8) -- (10,10);
        \draw [->] (8,8) -- (6,9);
        \draw [->] (10,10) -- (6,9);
        \draw [->] (7.12,9.56) -- (6.36,9.34);
        \draw [->] (8.68,9.92) -- (9.48,10.08);
        \draw (7.82,8.02) node[anchor=north west] {$O$};
        \draw (6.38,10.16) node[anchor=north west] {$F_{ij}$};
        \draw (8.68,10.72) node[anchor=north west] {$F_{ji}$};
        \draw (7.06,9.14) node[anchor=north west] {$r_i$};
        \draw (8.5,9.54) node[anchor=north west] {$r_j$};
        \draw (7.36,10.26) node[anchor=north west] {$r_i - r_j$};
    \end{tikzpicture}
\end{center}
Разобьем специальным образом по суммам:
\[ = [r_1, F_{12}] + [r_2, F_{21}] + \ldots + [r_i,F_{ij}] + [r_j, F_{ji}] + \ldots  = \ldots + [r_i,F_{ij}] - [r_j, F_{ij}] = [r_i - r_j, F_{ij}] = 0\]
Такое векторное произведение равно нулю из свойств векторного произведения(угол между слагаемыми-векторами = 0, см.рисунок выше)

Таким образом, возвращаясь к рассмотрению равенства выше получаем:
\[ \frac{dL}{dt} = 0 \Rightarrow L - const\]
\subsubsection{Момент импульса незамкнутых систем}
Возьмем полное уравнение второго закона Ньютона для незамкнутых систем: \[m_i \cdot \frac{dv_i}{dt} = \sum_{j = 1, j \neq i}^n \vec F_{ij} + \vec F_i ^{ex}\]
\[ [\vec r_i, \frac{dm_iv_i}{dt}] = \sum_{j = 1, j \neq i}^n [\vec r_i, F_{ij}] + [r_i,\vec F_i^{ex}]\]
\[\sum_{i = 1}^n \frac{dL_i}{dt} = \sum_{i = 1}^n \sum_{j = 1, j \neq i}^n [r_i, \vec F_{ij}] + [r_i,\vec F_i^{ex}]\]
Из доказанного выше $\sum_{i = 1}^n \sum_{j = 1, j \neq i}^n [r_i, \vec F_{ij}] = 0$
\[\sum_{i = 1}^n \frac{dL_i}{dt} = \sum_{i = 1}^n [r_i,\vec F_i^{ex}]\]
\[\frac{dL_i}{dt} = \vec N\]
$\vec N$ - главный момент внешних сил
\subsubsection{Момент импульса при изменении точки отсчета}
\begin{center}
    \begin{tikzpicture}[line cap=round,line join=round,>=triangle 45,x=1.0cm,y=1.0cm]
        \clip(5.94,6.48) rectangle (11.48,11.12);
        \draw [->] (8,8) -- (8,11);
        \draw [->] (8,8) -- (6,7);
        \draw [->] (8,8) -- (11,8);
        \draw [->] (8,8) -- (10,7);
        \draw [->] (8,8) -- (10,10);
        \draw [->] (10,7) -- (10,10);
        \draw (6.05,7.84) node[anchor=north west] {$X$};
        \draw (10.57,8.64) node[anchor=north west] {$Y$};
        \draw (7.47,11.19) node[anchor=north west] {$Z$};
        \draw (9.7,10.9) node[anchor=north west] {$M_i$};
        \draw (9.81,6.99) node[anchor=north west] {$A$};
        \draw (8.37,7.7) node[anchor=north west] {$a$};
        \draw (8.63,9.73) node[anchor=north west] {$r_i$};
        \draw (10.12,8.97) node[anchor=north west] {$r_i^{\prime}$};
    \end{tikzpicture}
\end{center}
\[ \sum_{i = 1}^n [\vec r_i, \vec P_i] = L \text{, где $\vec r_i = \vec r_i^{\prime} + \vec a$.}\]

Продифференцируем этот вектор по времени: $\vec v_i = \vec v_i^{\prime}$. Следовательно, при изменении точки отсчета скорость движения точек не изменится, а следовательно импульс так же не изменится $\vec p_i = \vec p_i^{\prime}$, посчитаем момент сил
\[ L = \sum_{i = 1}^n [r_i^{\prime}+a, p_i] = \sum_{i = 1}^n [r_i^{\prime}, p_i] + [\vec a,\sum_{i = 1}^n p_i] = L^{\prime} + [\vec a, \vec P] \]

\subsubsection{Момент импульса относительно центра масс}
\begin{center}
    \begin{tikzpicture}[line cap=round,line join=round,>=triangle 45,x=1.0cm,y=1.0cm]
        \clip(18.19,18.97) rectangle (23.45,23.7);
        \draw (9.81,6.99) node[anchor=north west] {$A$};
        \draw [->] (20,20) -- (20.03,23.55);
        \draw [->] (20,20) -- (18.68,18.92);
        \draw [->] (20,20) -- (23.14,20.01);
        \draw [->] (22,22) -- (22.37,23.13);
        \draw [->] (22,22) -- (22.32,20.97);
        \draw [->] (22,22) -- (21.05,21.66);
        \draw [->] (20,20) -- (22,22);
        \draw [->] (20,20) -- (20.56,23.01);
        \draw [->] (20.56,23.01) -- (22,22);
        \draw (18.48,19.9) node[anchor=north west] {$X$};
        \draw (22.68,20.77) node[anchor=north west] {$Y$};
        \draw (19.39,23.53) node[anchor=north west] {$Z$};
        \draw (19.47,20.39) node[anchor=north west] {$O$};
        \draw (21.05,21.17) node[anchor=north west] {$r_c$};
        \draw (20.37,21.97) node[anchor=north west] {$\vec r_i$};
        \draw (21.32,23.3) node[anchor=north west] {$\vec r_i^{\prime}$};
        \draw (22.5,21.48) node[anchor=north west] {$x^{\prime}$};
        \draw (20.81,22.28) node[anchor=north west] {$y^{\prime}$};
        \draw (22.48,23.28) node[anchor=north west] {$z^{\prime}$};
        \draw (22.08,22.37) node[anchor=north west] {$O^{\prime}$};
        \draw (20.27,23.75) node[anchor=north west] {$M_i$};
        \begin{scriptsize}
            \fill [color=black] (22,22) circle (1.5pt);
            \fill [color=black] (20.56,23.01) circle (1.5pt);
        \end{scriptsize}
    \end{tikzpicture}
\end{center}
Разница от прошлого случая в том, что система движется вместе с основной системой

\[ \vec r_i = \vec r_i^{\prime} + \vec r_c\]

Продифференцируем этот вектор по времени: $\vec v_i = \vec v_i^{\prime} + v_c$. Переходим к новой системе:
% \[ L = \sum_{i = 1}^n [r_i, P_i] = \sum_{i = 1}^n [r_i^{\prime} + r_c, m_i(v_i^{\prime} + v_с)] \Rightarrow \] %TODO!: FIX GOVNO


\[ \sum_{i = 1}^n [r_i^{\prime}, m_i \cdot v_i^{\prime}] + \sum_{i = 1}^n [r_i^{\prime}, m_i \cdot v_c] + \sum_{i = 1}^n [r_c, m_i \cdot v_i^{\prime}] + \sum_{i = 1}^n [r_c, m_i \cdot v_c]\]
\[ L_c^{\prime} + [r_c, v_c \cdot \sum_{i = 1}^n m_i]\]
Пояснение:
\begin{enumerate}
    \item $\sum_{i = 1}^n [r_i^{\prime}, m_i \cdot v_c]  = 0$ поскольку $r_c^{\prime} = \frac{r_i^{\prime}m_i}{M}$, но так как мы рассматриваем систему центра масс, то в ней $r_c^{\prime}  = 0$, а следовательно $\sum_{i = 1}^n [r_i^{\prime}, m_i \cdot v_c]  = 0$
    \item $\sum_{i = 1}^n [r_c, m_i \cdot v_i^{\prime}] = 0$, поскольку импульс в системе центра масс равен нулю
\end{enumerate}
\textbf{Вывод: момент импульса равен сумме момента импульса относительно центра масс и импульса всей системы как единого целого(аналог теоремы Кенига)}
\[ L = L_c^{\prime} + [r_c, v_c \cdot \sum_{i = 1}^n m_i]\]

\textit{Тем самым, для абсолютного твердого тела хватит лишь двух уравнений, чтобы полностью определить его движение:}
\[ \frac{d\vec p}{dt} = \vec K \]
\[ \frac{d\vec L}{dt} = \vec N\]
$\vec N$ - главный момент внешних сил, $\vec K$ - главный вектор внешних сил

\section{Секториальная скорость, теорема площадей}
\begin{center}

    \definecolor{qqwuqq}{rgb}{0,0.39,0}
    \definecolor{xdxdff}{rgb}{0.49,0.49,1}
    \definecolor{qqqqff}{rgb}{0,0,1}
    \begin{tikzpicture}[line cap=round,line join=round,>=triangle 45,x=1.3650335533039561cm,y=1.7832789948446495cm]
        \clip(12.05,33.64) rectangle (17.53,36.19);
        \draw [shift={(15.5,34)},color=qqwuqq,fill=qqwuqq,fill opacity=0.1] (0,0) -- (53.13:0.22) arc (53.13:161.57:0.22) -- cycle;
        \draw [->] (14,12.57) -- (14,15.5);
        \draw [->] (13.5,13) -- (17,13);
        \draw [shift={(16.15,15.15)}] plot[domain=3.23:4.62,variable=\t]({1*1.66*cos(\t r)+0*1.66*sin(\t r)},{0*1.66*cos(\t r)+1*1.66*sin(\t r)});
        \draw (14,15)-- (14.5,15);
        \draw (14.5,15)-- (14.5,13);
        \draw (16,14.5)-- (14,14.5);
        \draw (16,14.5)-- (16,13);
        \draw [rotate around={-54.68:(14.56,25.5)}] (14.56,25.5) ellipse (2.13cm and 2.06cm);
        \draw [->] (14,26) -- (15,26);
        \draw [->] (14.5,25) -- (15,26);
        \draw [->] (14,26) -- (14.5,25);
        \draw (14.34,26.43) node[anchor=north west] {$\vec a$};
        \draw (14.84,25.73) node[anchor=north west] {$\vec b$};
        \draw (13.96,25.64) node[anchor=north west] {$\vec r$};
        \draw (13.69,26.35) node[anchor=north west] {$A$};
        \draw (15.01,26.29) node[anchor=north west] {$M$};
        \draw (14.52,25.07) node[anchor=north west] {$B$};
        \draw [->] (12.5,35) -- (17,36);
        \draw [->] (12.5,35) -- (15.5,34);
        \draw [->] (15.5,34) -- (17,36);
        \draw (14.27,35.89) node[anchor=north west] {$\vec r+d \vec r$};
        \draw (15.77,35.23) node[anchor=north west] {$d\vec r$};
        \draw (14.22,34.86) node[anchor=north west] {$\vec r$};
        \begin{scriptsize}
            \fill [color=qqqqff] (14,12.57) circle (1.5pt);
            \draw[color=qqqqff] (11.87,38.77) node {$A$};
            \fill [color=qqqqff] (14,15.5) circle (1.5pt);
            \draw[color=qqqqff] (11.87,38.77) node {$B$};
            \fill [color=qqqqff] (13.5,13) circle (1.5pt);
            \draw[color=qqqqff] (11.87,38.77) node {$C$};
            \fill [color=qqqqff] (17,13) circle (1.5pt);
            \draw[color=qqqqff] (11.87,38.77) node {$D$};
            \fill [color=qqqqff] (14.5,15) circle (1.5pt);
            \draw[color=qqqqff] (11.87,38.77) node {$E$};
            \fill [color=qqqqff] (14.91,14.06) circle (1.5pt);
            \draw[color=qqqqff] (11.87,38.77) node {$F$};
            \fill [color=qqqqff] (16,13.5) circle (1.5pt);
            \draw[color=qqqqff] (11.87,38.77) node {$G$};
            \fill [color=xdxdff] (14,15) circle (1.5pt);
            \draw[color=xdxdff] (11.87,38.77) node {$H$};
            \fill [color=xdxdff] (14.5,13) circle (1.5pt);
            \draw[color=xdxdff] (11.86,38.77) node {$I$};
            \fill [color=qqqqff] (16,14.5) circle (1.5pt);
            \draw[color=qqqqff] (11.85,38.77) node {$J$};
            \fill [color=qqqqff] (14,14.5) circle (1.5pt);
            \draw[color=qqqqff] (11.87,38.77) node {$K$};
            \fill [color=xdxdff] (16,13) circle (1.5pt);
            \draw[color=xdxdff] (11.87,38.77) node {$L$};
            \fill [color=black] (12.5,35) circle (1.5pt);
            \fill [color=black] (17,36) circle (1.5pt);
            \fill [color=black] (15.5,34) circle (1.5pt);
        \end{scriptsize}
    \end{tikzpicture}
\end{center}
Пусть есть некоторая точка, которая движется в пространстве. Посчитаем площадь треугольника на рисунке за время $dt$( поскольку наш треугольник изменяется со временем) :
\[ ds  = \frac{1}{2}r \cdot dr \cdot \sin{(\hat{r;dr})} \]
\[ \langle  dr = vdt \rangle \]
\[ ds = \frac{1}{2} rvdt \sin{(\hat{r;dr})} \]
\[\frac{ds}{dt} = \frac{1}{2} rv \sin{(\hat{r;dr})}\]

\define \textit{Площадь, производная которой определяется радиус вектором при движении - \textbf{секториальная площадь.}} Причем вектор этой величины направлен перпендикулярно плоскости в которой лежит треугольник. Обозначение: $ \dot S = \frac{1}{2}[r,v]$

\vspace{5px}

Несложно заметить, что этот вектор похож на вектор момента импульса, учитывая что $L = [r, mv]$ следует $L = 2M \dot S$

\vspace{5px}

\define \textit{\textbf{Центральная сила} - сила, линии действия которой проходят через одну точку, которая в свою очередь называется центром силы.}

\vspace{5px}

\textbf{\textit{Если механическая система движется под действием центральной силы ее момент импульса относительно центра масс и любой другой неподвижной точки не изменяется.}}
\[L = [r, p] = const\]

\vspace{5px}

\textbf{\textit{Теорема площадей:}} \textit{Если материальная точка движется под действием центральной силы то ее траектория - плоская кривая и за равные промежутки времени радиус-вектор точки описывает равные по величине площади.}

\vspace{4px}

Справедливо и обратно: если траектория точки - плоская кривая и за равные промежутки времени радиус-вектор точки описывает равные по величине площади, то точка движется под действием центральной силы.

\section{Законы Кеплера. Закон всемирного тяготения}

\begin{enumerate}
    \item \textit{\textbf{I закон:} все планеты Солнечной системы двигаются по эллипсам, в одном из фокусов которых находится Солнце.}
    \item \textit{\textbf{II закон}: радиус-векторы планет за равные промежутки времени описывают равные по величине площади.}
    \item \textit{\textbf{III закон}: квадрат времени обращения планет вокруг Солнца относится как кубы больших полуосей эллиптических орбит: $ T_1^2 : T_2^2 : \ldots = r_1^3 :r_2^3 : \ldots$}
\end{enumerate}

Рассмотрим второй закон Ньютона: \[ F = m\frac{v^2}{r} = \langle v = \frac{2 \pi r}{T} \rangle = m \frac{4\pi^2 r^2}{T^2 r} = 4\pi^2 \frac{mr}{T^2}\].

\vspace{5px}

При этом по закону Кеплера: \[ T^2  = Kr^3 \text{, подставив получим }\]

\[ F = \frac{4\pi^2m}{Kr^2}\]

\vspace{5px}

Тогда, логично полагать что если Солнце притягивает к себе планеты солнечной системы, то очевидно что тело с такой же силой притягивает солнце, но тогда в этой формуле должна быть масса солнца, и она заключена в $\frac{4\pi^2}{K}$
\[ F = G_1\frac{Mm}{r^2} \text{, где $G_1 M = \frac{4\pi^2}{K}$.}\]

Однако очевидно, что существует сила с которой объекты на Земле притягиваются друг другу.

\vspace{5px}

\textit{\textbf{Закон всемирного тяготения: }все тела притягиваются друг другу с силой прямо пропорциональной произведению их масс и обратно пропорциональной квадрату расстояния между ними.}

\vspace{5px}

\textbf{Гравитационная постоянная: }$G = 6,67 \cdot 10^{-11} \text{Н} \cdot \frac{\text{м}^2}{\text{кг}^2}$

\vspace{5px}

Заметим, что во втором законе Ньютона и в законе всемирного тяготения отличается масса: в первом случае - масса инерциальная, во втором масса гравитационная. В практике было доказано что значения этих масс совпадают с точностью до 13 знака, однако в теоретически еще ничего не было доказано.
