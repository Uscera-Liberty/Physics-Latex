\documentclass[../main.tex]{subfiles}
\graphicspath{{\subfix{../images/}}}

\begin{document}

\chapter{Электричество и магнетизм}

\section{Электрическое поле в вакууме}
Наименьший заряд в природе обозначается как \texttt{e} и называется электроном. Из такого утверждения, следует что любой заряд в природе может быть определен как  $ q = \pm e \cdot N, N \in \mathbb{Z}$

\vspace{5px}

Заряд величина дискретная, может быть положительной и отрицательной, при соприкосновении разноименных зарядов они взаимоуничтожаются.
\subsection{Закон сохранения заряда}
\textit{\textbf{Формулировка:} Заряд сохраняется в электрически замкнутой системе.}

\define Электрически замкнутая система - система, через границу которой не проходят электрические заряды.

\subsection{Закон Кулона}
\textit{\textbf{Формулировка:} Все заряженные тела взаимодействуют между собой. При этом одноименные заряды отталкиваются, а разноименные притягиваются.}
Причем сила с которой они взаимодействуют равна:
\[ F = k\frac{q_1 q_2}{r^2}\]
\begin{center}
    где k - коэффициент пропорциональности,  $q_1,q_2$ - величины зарядов
\end{center}

\define \textbf{Точечным зарядом} называется заряженное тело, размером которого можно пренебречь по сравнению с расстоянием до других заряженных тел.

\vspace{5px}

В системе СИ заряд измеряют в Кулонах, при этом Кулон не основная единица измерения, а составная: $[q] = \text{Кл} = \text{Ам} \cdot \text{c}$.
При измерении в Кулонах коэффициент k будет равен $k = 9 \cdot 10^9 \frac{\text{Н} \cdot \text{м}^2}{\text{Кл}^2}$

На практике в опытах также возникал некоторый коэффициент, поэтому при решении задач используют уже нормализованную форму:
\[ k = \frac{1}{4 \pi \epsilon_0}\]
\begin{center}
    где $\epsilon_0$ электрическая постоянная ,$\epsilon_0 = 8,85 \cdot10^{-12} \frac{\text{Ф}}{\text{м}}$
\end{center}

\[F = \frac{1}{4 \pi \epsilon_0}\frac{q_1 q_2}{r^2}\]

\section{Электрическое поле}
\define Напряженность электрического поля
\[ \vec E = \frac{\vec F}{q} \text{(векторная)}\]
\begin{center}
    \definecolor{qqqqff}{rgb}{0,0,1}
    \definecolor{ffqqqq}{rgb}{1,0,0}
    \begin{tikzpicture}[line cap=round,line join=round,>=triangle 45,x=1.5755455765500463cm,y=1.4141895319101883cm]
        \clip(26.95,39.98) rectangle (37.37,42.39);
        \draw [line width=1.6pt,color=ffqqqq] (33,41) ellipse (0.45cm and 0.4cm);
        \draw [color=ffqqqq](32.83,41.34) node[anchor=north west] {$+$};
        \draw [line width=1.6pt,color=qqqqff] (35,41) ellipse (0.45cm and 0.4cm);
        \draw [color=qqqqff](34.82,41.34) node[anchor=north west] {$-$};
        \draw [->] (35.28,40.99) -- (36.4,40.97);
        \draw (35.57,41.77) node[anchor=north west] {$\vec E$};
        \draw [->] (34.71,40.99) -- (33.69,40.97);
        \draw (34.19,41.75) node[anchor=north west] {$\vec F$};
        \draw [line width=1.6pt,color=ffqqqq] (29.17,45.85) ellipse (0.45cm and 0.4cm);
        \draw [color=ffqqqq](29,46.2) node[anchor=north west] {$+$};
        \draw [line width=1.6pt,color=qqqqff] (31.87,45.96) ellipse (0.45cm and 0.4cm);
        \draw [color=qqqqff](31.75,46.34) node[anchor=north west] {$-$};
        \draw (32.91,41.74) node[anchor=north west] {q};
        \draw (34.82,41.94) node[anchor=north west] {$q_{\text{пр}}$};
        \draw [line width=1.6pt,color=ffqqqq] (27.96,41) ellipse (0.45cm and 0.4cm);
        \draw [color=ffqqqq](27.8,41.35) node[anchor=north west] {$+$};
        \draw [line width=1.6pt,color=ffqqqq] (29.68,41.02) ellipse (0.45cm and 0.4cm);
        \draw [color=ffqqqq](29.51,41.37) node[anchor=north west] {$+$};
        \draw [->] (29.96,41) -- (31.24,40.99);
        \draw (30.4,41.78) node[anchor=north west] {$\vec F$};
        \draw (30.34,41.03) node[anchor=north west] {$\vec E$};
        \draw (29.44,42.01) node[anchor=north west] {$q_{\text{пр}}$};
        \draw (27.89,41.82) node[anchor=north west] {q};
        \draw [line width=1.6pt,color=qqqqff] (28.02,39.04) ellipse (0.45cm and 0.4cm);
        \draw [color=qqqqff](27.83,39.37) node[anchor=north west] {$-$};
        \draw [line width=1.6pt,color=qqqqff] (29.69,39.02) ellipse (0.45cm and 0.4cm);
        \draw [color=qqqqff](29.51,39.35) node[anchor=north west] {$-$};
        \draw [->] (29.98,39) -- (31.22,39);
        \draw [line width=1.6pt,color=qqqqff] (33.03,39.02) ellipse (0.45cm and 0.4cm);
        \draw [color=qqqqff](32.85,39.35) node[anchor=north west] {$-$};
        \draw [line width=1.6pt,color=ffqqqq] (35,39.03) ellipse (0.45cm and 0.4cm);
        \draw [color=ffqqqq](34.83,39.37) node[anchor=north west] {$+$};
        \draw [->] (34.74,39.13) -- (33.63,39.15);
        \draw [->] (34.75,38.9) -- (33.59,38.88);
        \draw (30.31,39.83) node[anchor=north west] {$\vec F$};
        \draw (34.16,39.86) node[anchor=north west] {$\vec F$};
        \draw (30.28,38.92) node[anchor=north west] {$\vec E$};
        \draw (34.11,38.89) node[anchor=north west] {$\vec E$};
        \draw [line width=1.6pt,color=ffqqqq] (28.01,36.06) ellipse (0.45cm and 0.4cm);
        \draw [color=ffqqqq](27.85,36.4) node[anchor=north west] {$+$};
        \draw [line width=1.6pt,color=qqqqff] (30.96,36.05) ellipse (0.45cm and 0.4cm);
        \draw [color=qqqqff](30.84,36.44) node[anchor=north west] {$-$};
        \draw (28.29,36.02)-- (30.68,36.01);
        \draw [->] (30.68,36.01) -- (29,36.02);
        \draw (29.51,36.78) node[anchor=north west] {$\vec P$};
        \draw (29.56,36.09) node[anchor=north west] {$l$};
        \draw (27.96,39.7) node[anchor=north west] {q};
        \draw (27.96,36.83) node[anchor=north west] {q};
        \draw (31.04,36.78) node[anchor=north west] {q};
        \draw (32.98,39.7) node[anchor=north west] {q};
        \draw [line width=1.6pt,color=ffqqqq] (28.04,31.02) ellipse (0.45cm and 0.4cm);
        \draw [color=ffqqqq](27.86,31.37) node[anchor=north west] {$+$};
        \draw [->] (28.01,31.61) -- (27.99,32.55);
        \draw [->] (28.42,31.55) -- (29.28,32.33);
        \draw [->] (28.6,31.19) -- (29.66,31.25);
        \draw [->] (28.58,30.6) -- (29.54,30.15);
        \draw [->] (28.2,30.29) -- (28.71,29.58);
        \draw [->] (27.57,30.25) -- (26.96,29.66);
        \draw [->] (27.51,30.7) -- (26.45,30.35);
        \draw [->] (27.4,31.21) -- (26.51,31.73);
        \draw [line width=1.6pt,color=qqqqff] (31.99,31.14) ellipse (0.45cm and 0.4cm);
        \draw [color=qqqqff](31.81,31.47) node[anchor=north west] {$-$};
        \draw [->] (31.96,31.61) -- (31.88,32.49);
        \draw [->] (31.39,31.39) -- (30.7,31.82);
        \draw [->] (31.33,30.68) -- (30.48,30);
        \draw [->] (31.94,30.6) -- (32.21,29.7);
        \draw [->] (32.45,30.98) -- (33.28,30.58);
        \draw [->] (32.53,31.51) -- (33.32,32.02);
        \draw [line width=1.6pt,color=ffqqqq] (36,31) ellipse (0.45cm and 0.4cm);
        \draw [color=ffqqqq](35.86,31.33) node[anchor=north west] {$+$};
        \draw [line width=1.6pt,color=qqqqff] (38,31) ellipse (0.45cm and 0.4cm);
        \draw [color=qqqqff](37.85,31.36) node[anchor=north west] {$-$};
        \draw [->] (36.28,31.02) -- (37.72,31.02);
        \draw [->] (35.76,31.16) -- (35.5,31.5);
        \draw [->] (35.77,30.84) -- (35.47,30.61);
        \draw [->] (38.68,30.46) -- (38.21,30.81);
        \draw [->] (38.5,31.5) -- (38.13,31.25);
        \draw [shift={(37.02,30.3)}] plot[domain=0.83:2.32,variable=\t]({1*1.31*cos(\t r)+0*1.31*sin(\t r)},{0*1.31*cos(\t r)+1*1.31*sin(\t r)});
        \draw [shift={(36.98,31.65)}] plot[domain=3.95:5.49,variable=\t]({1*1.23*cos(\t r)+0*1.23*sin(\t r)},{0*1.23*cos(\t r)+1*1.23*sin(\t r)});
        \draw [shift={(36.97,29.37)}] plot[domain=1.16:1.96,variable=\t]({1*1.95*cos(\t r)+0*1.95*sin(\t r)},{0*1.95*cos(\t r)+1*1.95*sin(\t r)});
        \draw [shift={(36.76,33.3)}] plot[domain=4.47:5.1,variable=\t]({1*2.63*cos(\t r)+0*2.63*sin(\t r)},{0*2.63*cos(\t r)+1*2.63*sin(\t r)});
        \draw [rotate around={0:(31,53)},line width=1.2pt,dash pattern=on 1pt off 1pt] (31,53) ellipse (3.23cm and 0.63cm);
        \draw [dash pattern=on 1pt off 1pt] (31,53) ellipse (3.15cm and 2.83cm);
        \draw [rotate around={90:(31,53)},line width=1.2pt,dash pattern=on 1pt off 1pt] (31,53) ellipse (3.21cm and 0.56cm);
        \draw [line width=1.6pt,color=ffqqqq] (30.96,53.02) ellipse (0.45cm and 0.4cm);
        \draw [color=ffqqqq](30.79,53.36) node[anchor=north west] {$+$};
        \draw [->] (31.13,53.25) -- (32,54);
        \draw [->] (30.73,53.19) -- (29.99,53.63);
        \draw [->] (30.87,52.76) -- (30.36,52.04);
        \draw [->] (31.17,52.85) -- (32,52.31);
    \end{tikzpicture} %TODO: Фиксануть переполнение
\end{center}
\textbf{Получим, что линии напряженности электрического поля направлены от положительного заряда к отрицательному:}

\begin{center}

    \definecolor{qqqqff}{rgb}{0,0,1}
    \definecolor{ffqqqq}{rgb}{1,0,0}
    \begin{tikzpicture}[line cap=round,line join=round,>=triangle 45,x=1.3229395145530378cm,y=1.0cm]
        \clip(27.33,37.96) rectangle (36.62,40.02);
        \draw [line width=1.6pt,color=ffqqqq] (33,41) ellipse (0.37cm and 0.28cm);
        \draw [color=ffqqqq](32.83,41.34) node[anchor=north west] {$+$};
        \draw [line width=1.6pt,color=qqqqff] (35,41) ellipse (0.37cm and 0.28cm);
        \draw [color=qqqqff](34.82,41.34) node[anchor=north west] {$-$};
        \draw [->] (35.28,40.99) -- (36.4,40.97);
        \draw (35.57,41.77) node[anchor=north west] {$\vec E$};
        \draw [->] (34.71,40.99) -- (33.69,40.97);
        \draw (34.19,41.75) node[anchor=north west] {$\vec F$};
        \draw [line width=1.6pt,color=ffqqqq] (29.17,45.85) ellipse (0.37cm and 0.28cm);
        \draw [color=ffqqqq](29,46.2) node[anchor=north west] {$+$};
        \draw [line width=1.6pt,color=qqqqff] (31.87,45.96) ellipse (0.37cm and 0.28cm);
        \draw [color=qqqqff](31.75,46.34) node[anchor=north west] {$-$};
        \draw (32.91,41.74) node[anchor=north west] {q};
        \draw (34.82,41.94) node[anchor=north west] {$q_{\text{пр}}$};
        \draw [line width=1.6pt,color=ffqqqq] (27.96,41) ellipse (0.37cm and 0.28cm);
        \draw [color=ffqqqq](27.8,41.35) node[anchor=north west] {$+$};
        \draw [line width=1.6pt,color=ffqqqq] (29.68,41.02) ellipse (0.37cm and 0.28cm);
        \draw [color=ffqqqq](29.51,41.37) node[anchor=north west] {$+$};
        \draw [->] (29.96,41) -- (31.24,40.99);
        \draw (30.4,41.78) node[anchor=north west] {$\vec F$};
        \draw (30.34,41.03) node[anchor=north west] {$\vec E$};
        \draw (29.44,42.01) node[anchor=north west] {$q_{\text{пр}}$};
        \draw (27.89,41.82) node[anchor=north west] {q};
        \draw [line width=1.6pt,color=qqqqff] (28.02,39.04) ellipse (0.37cm and 0.28cm);
        \draw [color=qqqqff](27.83,39.37) node[anchor=north west] {$-$};
        \draw [line width=1.6pt,color=qqqqff] (29.69,39.02) ellipse (0.37cm and 0.28cm);
        \draw [color=qqqqff](29.51,39.35) node[anchor=north west] {$-$};
        \draw [->] (29.98,39) -- (31.22,39);
        \draw [line width=1.6pt,color=qqqqff] (33.03,39.02) ellipse (0.37cm and 0.28cm);
        \draw [color=qqqqff](32.85,39.35) node[anchor=north west] {$-$};
        \draw [line width=1.6pt,color=ffqqqq] (35,39.03) ellipse (0.37cm and 0.28cm);
        \draw [color=ffqqqq](34.83,39.37) node[anchor=north west] {$+$};
        \draw [->] (34.74,39.13) -- (33.63,39.15);
        \draw [->] (34.75,38.9) -- (33.59,38.88);
        \draw (30.31,39.83) node[anchor=north west] {$\vec F$};
        \draw (34.16,39.86) node[anchor=north west] {$\vec F$};
        \draw (30.28,38.92) node[anchor=north west] {$\vec E$};
        \draw (34.11,38.89) node[anchor=north west] {$\vec E$};
        \draw [line width=1.6pt,color=ffqqqq] (28.01,36.06) ellipse (0.37cm and 0.28cm);
        \draw [color=ffqqqq](27.85,36.4) node[anchor=north west] {$+$};
        \draw [line width=1.6pt,color=qqqqff] (30.96,36.05) ellipse (0.37cm and 0.28cm);
        \draw [color=qqqqff](30.84,36.44) node[anchor=north west] {$-$};
        \draw (28.29,36.02)-- (30.68,36.01);
        \draw [->] (30.68,36.01) -- (29,36.02);
        \draw (29.51,36.78) node[anchor=north west] {$\vec P$};
        \draw (29.56,36.09) node[anchor=north west] {$l$};
        \draw (27.96,39.7) node[anchor=north west] {q};
        \draw (27.96,36.83) node[anchor=north west] {q};
        \draw (31.04,36.78) node[anchor=north west] {q};
        \draw (32.98,39.7) node[anchor=north west] {q};
        \draw [line width=1.6pt,color=ffqqqq] (28.04,31.02) ellipse (0.37cm and 0.28cm);
        \draw [color=ffqqqq](27.86,31.37) node[anchor=north west] {$+$};
        \draw [->] (28.01,31.61) -- (27.99,32.55);
        \draw [->] (28.42,31.55) -- (29.28,32.33);
        \draw [->] (28.6,31.19) -- (29.66,31.25);
        \draw [->] (28.58,30.6) -- (29.54,30.15);
        \draw [->] (28.2,30.29) -- (28.71,29.58);
        \draw [->] (27.57,30.25) -- (26.96,29.66);
        \draw [->] (27.51,30.7) -- (26.45,30.35);
        \draw [->] (27.4,31.21) -- (26.51,31.73);
        \draw [line width=1.6pt,color=qqqqff] (31.99,31.14) ellipse (0.37cm and 0.28cm);
        \draw [color=qqqqff](31.81,31.47) node[anchor=north west] {$-$};
        \draw [->] (31.96,31.61) -- (31.88,32.49);
        \draw [->] (31.39,31.39) -- (30.7,31.82);
        \draw [->] (31.33,30.68) -- (30.48,30);
        \draw [->] (31.94,30.6) -- (32.21,29.7);
        \draw [->] (32.45,30.98) -- (33.28,30.58);
        \draw [->] (32.53,31.51) -- (33.32,32.02);
        \draw [line width=1.6pt,color=ffqqqq] (36,31) ellipse (0.37cm and 0.28cm);
        \draw [color=ffqqqq](35.86,31.33) node[anchor=north west] {$+$};
        \draw [line width=1.6pt,color=qqqqff] (38,31) ellipse (0.37cm and 0.28cm);
        \draw [color=qqqqff](37.85,31.36) node[anchor=north west] {$-$};
        \draw [->] (36.28,31.02) -- (37.72,31.02);
        \draw [->] (35.76,31.16) -- (35.5,31.5);
        \draw [->] (35.77,30.84) -- (35.47,30.61);
        \draw [->] (38.68,30.46) -- (38.21,30.81);
        \draw [->] (38.5,31.5) -- (38.13,31.25);
        \draw [shift={(37.02,30.3)}] plot[domain=0.83:2.32,variable=\t]({1*1.31*cos(\t r)+0*1.31*sin(\t r)},{0*1.31*cos(\t r)+1*1.31*sin(\t r)});
        \draw [shift={(36.98,31.65)}] plot[domain=3.95:5.49,variable=\t]({1*1.23*cos(\t r)+0*1.23*sin(\t r)},{0*1.23*cos(\t r)+1*1.23*sin(\t r)});
        \draw [shift={(36.97,29.37)}] plot[domain=1.16:1.96,variable=\t]({1*1.95*cos(\t r)+0*1.95*sin(\t r)},{0*1.95*cos(\t r)+1*1.95*sin(\t r)});
        \draw [shift={(36.76,33.3)}] plot[domain=4.47:5.1,variable=\t]({1*2.63*cos(\t r)+0*2.63*sin(\t r)},{0*2.63*cos(\t r)+1*2.63*sin(\t r)});
        \draw [rotate around={0:(31,53)},line width=1.2pt,dash pattern=on 1pt off 1pt] (31,53) ellipse (2.71cm and 0.44cm);
        \draw [dash pattern=on 1pt off 1pt] (31,53) ellipse (2.65cm and 2cm);
        \draw [rotate around={90:(31,53)},line width=1.2pt,dash pattern=on 1pt off 1pt] (31,53) ellipse (2.7cm and 0.39cm);
        \draw [line width=1.6pt,color=ffqqqq] (30.96,53.02) ellipse (0.37cm and 0.28cm);
        \draw [color=ffqqqq](30.79,53.36) node[anchor=north west] {$+$};
        \draw [->] (31.13,53.25) -- (32,54);
        \draw [->] (30.73,53.19) -- (29.99,53.63);
        \draw [->] (30.87,52.76) -- (30.36,52.04);
        \draw [->] (31.17,52.85) -- (32,52.31);
    \end{tikzpicture}
\end{center}

Величина напряженности точечного заряда:
\[ E = \frac{1}{4 \pi \epsilon_0}\frac{q}{r^2}\]

\textit{\textbf{Принцип суперпозиции полей:} Если имеется несколько электрических полей $\vec E_1, \vec E_2, \ldots , \vec E_n$, то
    \[ \vec E = \sum_{i = 1}^{n} \vec E_i\]
    Это вытекает из принципа суперпозции сил.}

\subsection{Электрический диполь}
\define \textbf{Электрический диполь} - это структура, состоящая из пары зарядов.
\begin{center}
    \definecolor{qqqqff}{rgb}{0,0,1}
    \definecolor{ffqqqq}{rgb}{1,0,0}
    \begin{tikzpicture}[line cap=round,line join=round,>=triangle 45,x=1.9772284268505984cm,y=1.6583433612467202cm]
        \clip(27.11,34.97) rectangle (32.23,37.1);
        \draw [line width=1.6pt,color=ffqqqq] (33,41) ellipse (0.56cm and 0.47cm);
        \draw [color=ffqqqq](32.83,41.34) node[anchor=north west] {$+$};
        \draw [line width=1.6pt,color=qqqqff] (35,41) ellipse (0.56cm and 0.47cm);
        \draw [color=qqqqff](34.82,41.34) node[anchor=north west] {$-$};
        \draw [->] (35.28,40.99) -- (36.4,40.97);
        \draw (35.57,41.77) node[anchor=north west] {$\vec E$};
        \draw [->] (34.71,40.99) -- (33.69,40.97);
        \draw (34.19,41.75) node[anchor=north west] {$\vec F$};
        \draw [line width=1.6pt,color=ffqqqq] (29.17,45.85) ellipse (0.56cm and 0.47cm);
        \draw [color=ffqqqq](29,46.2) node[anchor=north west] {$+$};
        \draw [line width=1.6pt,color=qqqqff] (31.87,45.96) ellipse (0.56cm and 0.47cm);
        \draw [color=qqqqff](31.75,46.34) node[anchor=north west] {$-$};
        \draw (32.91,41.74) node[anchor=north west] {q};
        \draw (34.82,41.94) node[anchor=north west] {$q_{\text{пр}}$};
        \draw [line width=1.6pt,color=ffqqqq] (27.96,41) ellipse (0.56cm and 0.47cm);
        \draw [color=ffqqqq](27.8,41.35) node[anchor=north west] {$+$};
        \draw [line width=1.6pt,color=ffqqqq] (29.68,41.02) ellipse (0.56cm and 0.47cm);
        \draw [color=ffqqqq](29.51,41.37) node[anchor=north west] {$+$};
        \draw [->] (29.96,41) -- (31.24,40.99);
        \draw (30.4,41.78) node[anchor=north west] {$\vec F$};
        \draw (30.34,41.03) node[anchor=north west] {$\vec E$};
        \draw (29.44,42.01) node[anchor=north west] {$q_{\text{пр}}$};
        \draw (27.89,41.82) node[anchor=north west] {q};
        \draw [line width=1.6pt,color=qqqqff] (28.02,39.04) ellipse (0.56cm and 0.47cm);
        \draw [color=qqqqff](27.83,39.37) node[anchor=north west] {$-$};
        \draw [line width=1.6pt,color=qqqqff] (29.69,39.02) ellipse (0.56cm and 0.47cm);
        \draw [color=qqqqff](29.51,39.35) node[anchor=north west] {$-$};
        \draw [->] (29.98,39) -- (31.22,39);
        \draw [line width=1.6pt,color=qqqqff] (33.03,39.02) ellipse (0.56cm and 0.47cm);
        \draw [color=qqqqff](32.85,39.35) node[anchor=north west] {$-$};
        \draw [line width=1.6pt,color=ffqqqq] (35,39.03) ellipse (0.56cm and 0.47cm);
        \draw [color=ffqqqq](34.83,39.37) node[anchor=north west] {$+$};
        \draw [->] (34.74,39.13) -- (33.63,39.15);
        \draw [->] (34.75,38.9) -- (33.59,38.88);
        \draw (30.31,39.83) node[anchor=north west] {$\vec F$};
        \draw (34.16,39.86) node[anchor=north west] {$\vec F$};
        \draw (30.28,38.92) node[anchor=north west] {$\vec E$};
        \draw (34.11,38.89) node[anchor=north west] {$\vec E$};
        \draw [line width=1.6pt,color=ffqqqq] (28.01,36.06) ellipse (0.56cm and 0.47cm);
        \draw [color=ffqqqq](27.85,36.4) node[anchor=north west] {$+$};
        \draw [line width=1.6pt,color=qqqqff] (30.96,36.05) ellipse (0.56cm and 0.47cm);
        \draw [color=qqqqff](30.84,36.44) node[anchor=north west] {$-$};
        \draw (28.29,36.02)-- (30.68,36.01);
        \draw [->] (30.68,36.01) -- (29,36.02);
        \draw (29.51,36.78) node[anchor=north west] {$\vec P$};
        \draw (29.56,36.09) node[anchor=north west] {$l$};
        \draw (27.96,39.7) node[anchor=north west] {q};
        \draw (27.96,36.83) node[anchor=north west] {q};
        \draw (31.04,36.78) node[anchor=north west] {q};
        \draw (32.98,39.7) node[anchor=north west] {q};
        \draw [line width=1.6pt,color=ffqqqq] (28.04,31.02) ellipse (0.56cm and 0.47cm);
        \draw [color=ffqqqq](27.86,31.37) node[anchor=north west] {$+$};
        \draw [->] (28.01,31.61) -- (27.99,32.55);
        \draw [->] (28.42,31.55) -- (29.28,32.33);
        \draw [->] (28.6,31.19) -- (29.66,31.25);
        \draw [->] (28.58,30.6) -- (29.54,30.15);
        \draw [->] (28.2,30.29) -- (28.71,29.58);
        \draw [->] (27.57,30.25) -- (26.96,29.66);
        \draw [->] (27.51,30.7) -- (26.45,30.35);
        \draw [->] (27.4,31.21) -- (26.51,31.73);
        \draw [line width=1.6pt,color=qqqqff] (31.99,31.14) ellipse (0.56cm and 0.47cm);
        \draw [color=qqqqff](31.81,31.47) node[anchor=north west] {$-$};
        \draw [->] (31.96,31.61) -- (31.88,32.49);
        \draw [->] (31.39,31.39) -- (30.7,31.82);
        \draw [->] (31.33,30.68) -- (30.48,30);
        \draw [->] (31.94,30.6) -- (32.21,29.7);
        \draw [->] (32.45,30.98) -- (33.28,30.58);
        \draw [->] (32.53,31.51) -- (33.32,32.02);
        \draw [line width=1.6pt,color=ffqqqq] (36,31) ellipse (0.56cm and 0.47cm);
        \draw [color=ffqqqq](35.86,31.33) node[anchor=north west] {$+$};
        \draw [line width=1.6pt,color=qqqqff] (38,31) ellipse (0.56cm and 0.47cm);
        \draw [color=qqqqff](37.85,31.36) node[anchor=north west] {$-$};
        \draw [->] (36.28,31.02) -- (37.72,31.02);
        \draw [->] (35.76,31.16) -- (35.5,31.5);
        \draw [->] (35.77,30.84) -- (35.47,30.61);
        \draw [->] (38.68,30.46) -- (38.21,30.81);
        \draw [->] (38.5,31.5) -- (38.13,31.25);
        \draw [shift={(37.02,30.3)}] plot[domain=0.83:2.32,variable=\t]({1*1.31*cos(\t r)+0*1.31*sin(\t r)},{0*1.31*cos(\t r)+1*1.31*sin(\t r)});
        \draw [shift={(36.98,31.65)}] plot[domain=3.95:5.49,variable=\t]({1*1.23*cos(\t r)+0*1.23*sin(\t r)},{0*1.23*cos(\t r)+1*1.23*sin(\t r)});
        \draw [shift={(36.97,29.37)}] plot[domain=1.16:1.96,variable=\t]({1*1.95*cos(\t r)+0*1.95*sin(\t r)},{0*1.95*cos(\t r)+1*1.95*sin(\t r)});
        \draw [shift={(36.76,33.3)}] plot[domain=4.47:5.1,variable=\t]({1*2.63*cos(\t r)+0*2.63*sin(\t r)},{0*2.63*cos(\t r)+1*2.63*sin(\t r)});
        \draw [rotate around={0:(31,53)},line width=1.2pt,dash pattern=on 1pt off 1pt] (31,53) ellipse (4.05cm and 0.73cm);
        \draw [dash pattern=on 1pt off 1pt] (31,53) ellipse (3.95cm and 3.32cm);
        \draw [rotate around={90:(31,53)},line width=1.2pt,dash pattern=on 1pt off 1pt] (31,53) ellipse (4.03cm and 0.65cm);
        \draw [line width=1.6pt,color=ffqqqq] (30.96,53.02) ellipse (0.56cm and 0.47cm);
        \draw [color=ffqqqq](30.79,53.36) node[anchor=north west] {$+$};
        \draw [->] (31.13,53.25) -- (32,54);
        \draw [->] (30.73,53.19) -- (29.99,53.63);
        \draw [->] (30.87,52.76) -- (30.36,52.04);
        \draw [->] (31.17,52.85) -- (32,52.31);
    \end{tikzpicture}
\end{center}

Для характеристики данной структуры используют \textbf{момент диполя}: $P = q \cdot l$ , причем направлен он от отрицательного заряда к положительному.
\subsection{Линии напряженности электрического поля}
\begin{center}
    \definecolor{qqqqff}{rgb}{0,0,1}
    \definecolor{ffqqqq}{rgb}{1,0,0}
    \begin{tikzpicture}[line cap=round,line join=round,>=triangle 45,x=1.2300461177083677cm,y=1.2479929779551238cm]
        \clip(26.1,28.86) rectangle (39.14,32.89);
        \draw [line width=1.6pt,color=ffqqqq] (33,41) ellipse (0.35cm and 0.35cm);
        \draw [color=ffqqqq](32.84,41.36) node[anchor=north west] {$+$};
        \draw [line width=1.6pt,color=qqqqff] (35,41) ellipse (0.35cm and 0.35cm);
        \draw [color=qqqqff](34.83,41.35) node[anchor=north west] {$-$};
        \draw [->] (35.28,40.99) -- (36.4,40.97);
        \draw (35.57,41.79) node[anchor=north west] {$\vec E$};
        \draw [->] (34.71,40.99) -- (33.69,40.97);
        \draw (34.19,41.76) node[anchor=north west] {$\vec F$};
        \draw [line width=1.6pt,color=ffqqqq] (29.17,45.85) ellipse (0.35cm and 0.35cm);
        \draw [color=ffqqqq](29,46.2) node[anchor=north west] {$+$};
        \draw [line width=1.6pt,color=qqqqff] (31.87,45.96) ellipse (0.35cm and 0.35cm);
        \draw [color=qqqqff](31.74,46.36) node[anchor=north west] {$-$};
        \draw (32.93,41.74) node[anchor=north west] {q};
        \draw (34.81,41.95) node[anchor=north west] {$q_{\text{пр}}$};
        \draw [line width=1.6pt,color=ffqqqq] (27.96,41) ellipse (0.35cm and 0.35cm);
        \draw [color=ffqqqq](27.79,41.36) node[anchor=north west] {$+$};
        \draw [line width=1.6pt,color=ffqqqq] (29.68,41.02) ellipse (0.35cm and 0.35cm);
        \draw [color=ffqqqq](29.51,41.38) node[anchor=north west] {$+$};
        \draw [->] (29.96,41) -- (31.24,40.99);
        \draw (30.4,41.81) node[anchor=north west] {$\vec F$};
        \draw (30.33,41.06) node[anchor=north west] {$\vec E$};
        \draw (29.44,42.02) node[anchor=north west] {$q_{\text{пр}}$};
        \draw (27.9,41.83) node[anchor=north west] {q};
        \draw [line width=1.6pt,color=qqqqff] (28.02,39.04) ellipse (0.35cm and 0.35cm);
        \draw [color=qqqqff](27.83,39.39) node[anchor=north west] {$-$};
        \draw [line width=1.6pt,color=qqqqff] (29.69,39.02) ellipse (0.35cm and 0.35cm);
        \draw [color=qqqqff](29.51,39.37) node[anchor=north west] {$-$};
        \draw [->] (29.98,39) -- (31.22,39);
        \draw [line width=1.6pt,color=qqqqff] (33.03,39.02) ellipse (0.35cm and 0.35cm);
        \draw [color=qqqqff](32.84,39.37) node[anchor=north west] {$-$};
        \draw [line width=1.6pt,color=ffqqqq] (35,39.03) ellipse (0.35cm and 0.35cm);
        \draw [color=ffqqqq](34.83,39.39) node[anchor=north west] {$+$};
        \draw [->] (34.74,39.13) -- (33.63,39.15);
        \draw [->] (34.75,38.9) -- (33.59,38.88);
        \draw (30.3,39.84) node[anchor=north west] {$\vec F$};
        \draw (34.16,39.89) node[anchor=north west] {$\vec F$};
        \draw (30.28,38.93) node[anchor=north west] {$\vec E$};
        \draw (34.11,38.91) node[anchor=north west] {$\vec E$};
        \draw [line width=1.6pt,color=ffqqqq] (28.01,36.06) ellipse (0.35cm and 0.35cm);
        \draw [color=ffqqqq](27.85,36.41) node[anchor=north west] {$+$};
        \draw [line width=1.6pt,color=qqqqff] (30.96,36.05) ellipse (0.35cm and 0.35cm);
        \draw [color=qqqqff](30.83,36.46) node[anchor=north west] {$-$};
        \draw (28.29,36.02)-- (30.68,36.01);
        \draw [->] (30.68,36.01) -- (29,36.02);
        \draw (29.51,36.78) node[anchor=north west] {$\vec P$};
        \draw (29.54,36.1) node[anchor=north west] {$l$};
        \draw (27.95,39.72) node[anchor=north west] {q};
        \draw (27.95,36.85) node[anchor=north west] {q};
        \draw (31.04,36.78) node[anchor=north west] {q};
        \draw (32.98,39.72) node[anchor=north west] {q};
        \draw [line width=1.6pt,color=ffqqqq] (28.04,31.02) ellipse (0.35cm and 0.35cm);
        \draw [color=ffqqqq](27.86,31.38) node[anchor=north west] {$+$};
        \draw [->] (28.01,31.61) -- (27.99,32.55);
        \draw [->] (28.42,31.55) -- (29.28,32.33);
        \draw [->] (28.6,31.19) -- (29.66,31.25);
        \draw [->] (28.58,30.6) -- (29.54,30.15);
        \draw [->] (28.2,30.29) -- (28.71,29.58);
        \draw [->] (27.57,30.25) -- (26.96,29.66);
        \draw [->] (27.51,30.7) -- (26.45,30.35);
        \draw [->] (27.4,31.21) -- (26.51,31.73);
        \draw [line width=1.6pt,color=qqqqff] (31.99,31.14) ellipse (0.35cm and 0.35cm);
        \draw [color=qqqqff](31.81,31.48) node[anchor=north west] {$-$};
        \draw [->] (31.96,31.61) -- (31.88,32.49);
        \draw [->] (31.39,31.39) -- (30.7,31.82);
        \draw [->] (31.33,30.68) -- (30.48,30);
        \draw [->] (31.94,30.6) -- (32.21,29.7);
        \draw [->] (32.45,30.98) -- (33.28,30.58);
        \draw [->] (32.53,31.51) -- (33.32,32.02);
        \draw [line width=1.6pt,color=ffqqqq] (36,31) ellipse (0.35cm and 0.35cm);
        \draw [color=ffqqqq](35.84,31.34) node[anchor=north west] {$+$};
        \draw [line width=1.6pt,color=qqqqff] (38,31) ellipse (0.35cm and 0.35cm);
        \draw [color=qqqqff](37.85,31.36) node[anchor=north west] {$-$};
        \draw [->] (36.28,31.02) -- (37.72,31.02);
        \draw [->] (35.76,31.16) -- (35.5,31.5);
        \draw [->] (35.77,30.84) -- (35.47,30.61);
        \draw [->] (38.68,30.46) -- (38.21,30.81);
        \draw [->] (38.5,31.5) -- (38.13,31.25);
        \draw [shift={(37.02,30.3)}] plot[domain=0.83:2.32,variable=\t]({1*1.31*cos(\t r)+0*1.31*sin(\t r)},{0*1.31*cos(\t r)+1*1.31*sin(\t r)});
        \draw [shift={(36.98,31.65)}] plot[domain=3.95:5.49,variable=\t]({1*1.23*cos(\t r)+0*1.23*sin(\t r)},{0*1.23*cos(\t r)+1*1.23*sin(\t r)});
        \draw [shift={(36.97,29.37)}] plot[domain=1.16:1.96,variable=\t]({1*1.95*cos(\t r)+0*1.95*sin(\t r)},{0*1.95*cos(\t r)+1*1.95*sin(\t r)});
        \draw [shift={(36.76,33.3)}] plot[domain=4.47:5.1,variable=\t]({1*2.63*cos(\t r)+0*2.63*sin(\t r)},{0*2.63*cos(\t r)+1*2.63*sin(\t r)});
        \draw [rotate around={0:(31,53)},line width=1.2pt,dash pattern=on 1pt off 1pt] (31,53) ellipse (2.52cm and 0.55cm);
        \draw [dash pattern=on 1pt off 1pt] (31,53) ellipse (2.46cm and 2.5cm);
        \draw [rotate around={90:(31,53)},line width=1.2pt,dash pattern=on 1pt off 1pt] (31,53) ellipse (2.51cm and 0.49cm);
        \draw [line width=1.6pt,color=ffqqqq] (30.96,53.02) ellipse (0.35cm and 0.35cm);
        \draw [color=ffqqqq](30.8,53.38) node[anchor=north west] {$+$};
        \draw [->] (31.13,53.25) -- (32,54);
        \draw [->] (30.73,53.19) -- (29.99,53.63);
        \draw [->] (30.87,52.76) -- (30.36,52.04);
        \draw [->] (31.17,52.85) -- (32,52.31);
    \end{tikzpicture} %TODO: Переполнение
\end{center}
Линии напряженности электрического поля проводятся так, \textit{чтобы касательная к ним совпадала по направлению с вектором напряженности электрического поля.}

\vspace{5px}

Количество линий пропорционально величине напряженности электрического поля. Для подсчета таких линий, возьмем $dS$ - элементарная площадка, $dN$ - количество линий на этой площадке,
тогда очевидно: $dN = EdS$, а для подсчета таких линий на всей поверхности возьмем поверхностный интеграл:

\[N = \iint\limits_S \vec E \cdot \vec n \cdot dS\]
\linebreak
\begin{center}
    где $\vec n$ - нормаль к поверхности S
\end{center}

\define Произвольный интеграл такого вида называется потоком, то есть для произвольного вектора A поток $\Phi_{A}$

\[\Phi_A = \iint\limits_S \vec A \cdot \vec n \cdot dS\]

\subsection{Теорема Гаусса}
\textit{\textbf{Формулировка:}} поток вектора электрического поля через замкнутую
поверхность равен алгебраической сумме зарядов, заключенной внутри поверхности и деленному
на электрическую постоянную $\epsilon_0$
\begin{center}
    \definecolor{qqqqff}{rgb}{0,0,1}
    \definecolor{ffqqqq}{rgb}{1,0,0}
    \begin{tikzpicture}[line cap=round,line join=round,>=triangle 45,x=1.0cm,y=1.0cm]
        \clip(28.02,50.44) rectangle (34.06,55.52);
        \draw [line width=1.6pt,color=ffqqqq] (33,41) circle (0.28cm);
        \draw [color=ffqqqq](32.84,41.36) node[anchor=north west] {$+$};
        \draw [line width=1.6pt,color=qqqqff] (35,41) circle (0.28cm);
        \draw [color=qqqqff](34.83,41.35) node[anchor=north west] {$-$};
        \draw [->] (35.28,40.99) -- (36.4,40.97);
        \draw (35.57,41.79) node[anchor=north west] {$\vec E$};
        \draw [->] (34.71,40.99) -- (33.69,40.97);
        \draw (34.19,41.76) node[anchor=north west] {$\vec F$};
        \draw [line width=1.6pt,color=ffqqqq] (29.17,45.85) circle (0.28cm);
        \draw [color=ffqqqq](29,46.2) node[anchor=north west] {$+$};
        \draw [line width=1.6pt,color=qqqqff] (31.87,45.96) circle (0.28cm);
        \draw [color=qqqqff](31.74,46.36) node[anchor=north west] {$-$};
        \draw (32.93,41.74) node[anchor=north west] {q};
        \draw (34.81,41.95) node[anchor=north west] {$q_{\text{пр}}$};
        \draw [line width=1.6pt,color=ffqqqq] (27.96,41) circle (0.28cm);
        \draw [color=ffqqqq](27.79,41.36) node[anchor=north west] {$+$};
        \draw [line width=1.6pt,color=ffqqqq] (29.68,41.02) circle (0.28cm);
        \draw [color=ffqqqq](29.51,41.38) node[anchor=north west] {$+$};
        \draw [->] (29.96,41) -- (31.24,40.99);
        \draw (30.4,41.81) node[anchor=north west] {$\vec F$};
        \draw (30.33,41.06) node[anchor=north west] {$\vec E$};
        \draw (29.44,42.02) node[anchor=north west] {$q_{\text{пр}}$};
        \draw (27.9,41.83) node[anchor=north west] {q};
        \draw [line width=1.6pt,color=qqqqff] (28.02,39.04) circle (0.28cm);
        \draw [color=qqqqff](27.83,39.39) node[anchor=north west] {$-$};
        \draw [line width=1.6pt,color=qqqqff] (29.69,39.02) circle (0.28cm);
        \draw [color=qqqqff](29.51,39.37) node[anchor=north west] {$-$};
        \draw [->] (29.98,39) -- (31.22,39);
        \draw [line width=1.6pt,color=qqqqff] (33.03,39.02) circle (0.28cm);
        \draw [color=qqqqff](32.84,39.37) node[anchor=north west] {$-$};
        \draw [line width=1.6pt,color=ffqqqq] (35,39.03) circle (0.28cm);
        \draw [color=ffqqqq](34.83,39.39) node[anchor=north west] {$+$};
        \draw [->] (34.74,39.13) -- (33.63,39.15);
        \draw [->] (34.75,38.9) -- (33.59,38.88);
        \draw (30.3,39.84) node[anchor=north west] {$\vec F$};
        \draw (34.16,39.89) node[anchor=north west] {$\vec F$};
        \draw (30.28,38.93) node[anchor=north west] {$\vec E$};
        \draw (34.11,38.91) node[anchor=north west] {$\vec E$};
        \draw [line width=1.6pt,color=ffqqqq] (28.01,36.06) circle (0.28cm);
        \draw [color=ffqqqq](27.85,36.41) node[anchor=north west] {$+$};
        \draw [line width=1.6pt,color=qqqqff] (30.96,36.05) circle (0.28cm);
        \draw [color=qqqqff](30.83,36.46) node[anchor=north west] {$-$};
        \draw (28.29,36.02)-- (30.68,36.01);
        \draw [->] (30.68,36.01) -- (29,36.02);
        \draw (29.51,36.78) node[anchor=north west] {$\vec P$};
        \draw (29.54,36.1) node[anchor=north west] {$l$};
        \draw (27.95,39.72) node[anchor=north west] {q};
        \draw (27.95,36.85) node[anchor=north west] {q};
        \draw (31.04,36.78) node[anchor=north west] {q};
        \draw (32.98,39.72) node[anchor=north west] {q};
        \draw [line width=1.6pt,color=ffqqqq] (28.04,31.02) circle (0.28cm);
        \draw [color=ffqqqq](27.86,31.38) node[anchor=north west] {$+$};
        \draw [->] (28.01,31.61) -- (27.99,32.55);
        \draw [->] (28.42,31.55) -- (29.28,32.33);
        \draw [->] (28.6,31.19) -- (29.66,31.25);
        \draw [->] (28.58,30.6) -- (29.54,30.15);
        \draw [->] (28.2,30.29) -- (28.71,29.58);
        \draw [->] (27.57,30.25) -- (26.96,29.66);
        \draw [->] (27.51,30.7) -- (26.45,30.35);
        \draw [->] (27.4,31.21) -- (26.51,31.73);
        \draw [line width=1.6pt,color=qqqqff] (31.99,31.14) circle (0.28cm);
        \draw [color=qqqqff](31.81,31.48) node[anchor=north west] {$-$};
        \draw [->] (31.96,31.61) -- (31.88,32.49);
        \draw [->] (31.39,31.39) -- (30.7,31.82);
        \draw [->] (31.33,30.68) -- (30.48,30);
        \draw [->] (31.94,30.6) -- (32.21,29.7);
        \draw [->] (32.45,30.98) -- (33.28,30.58);
        \draw [->] (32.53,31.51) -- (33.32,32.02);
        \draw [line width=1.6pt,color=ffqqqq] (36,31) circle (0.28cm);
        \draw [color=ffqqqq](35.84,31.34) node[anchor=north west] {$+$};
        \draw [line width=1.6pt,color=qqqqff] (38,31) circle (0.28cm);
        \draw [color=qqqqff](37.85,31.36) node[anchor=north west] {$-$};
        \draw [->] (36.28,31.02) -- (37.72,31.02);
        \draw [->] (35.76,31.16) -- (35.5,31.5);
        \draw [->] (35.77,30.84) -- (35.47,30.61);
        \draw [->] (38.68,30.46) -- (38.21,30.81);
        \draw [->] (38.5,31.5) -- (38.13,31.25);
        \draw [shift={(37.02,30.3)}] plot[domain=0.83:2.32,variable=\t]({1*1.31*cos(\t r)+0*1.31*sin(\t r)},{0*1.31*cos(\t r)+1*1.31*sin(\t r)});
        \draw [shift={(36.98,31.65)}] plot[domain=3.95:5.49,variable=\t]({1*1.23*cos(\t r)+0*1.23*sin(\t r)},{0*1.23*cos(\t r)+1*1.23*sin(\t r)});
        \draw [shift={(36.97,29.37)}] plot[domain=1.16:1.96,variable=\t]({1*1.95*cos(\t r)+0*1.95*sin(\t r)},{0*1.95*cos(\t r)+1*1.95*sin(\t r)});
        \draw [shift={(36.76,33.3)}] plot[domain=4.47:5.1,variable=\t]({1*2.63*cos(\t r)+0*2.63*sin(\t r)},{0*2.63*cos(\t r)+1*2.63*sin(\t r)});
        \draw [rotate around={0:(31,53)},line width=1.2pt,dash pattern=on 1pt off 1pt] (31,53) ellipse (2.05cm and 0.44cm);
        \draw [dash pattern=on 1pt off 1pt] (31,53) circle (2cm);
        \draw [rotate around={90:(31,53)},line width=1.2pt,dash pattern=on 1pt off 1pt] (31,53) ellipse (2.04cm and 0.39cm);
        \draw [line width=1.6pt,color=ffqqqq] (30.96,53.02) circle (0.28cm);
        \draw [color=ffqqqq](30.8,53.38) node[anchor=north west] {$+$};
        \draw [->] (31.13,53.25) -- (32,54);
        \draw [->] (30.73,53.19) -- (29.99,53.63);
        \draw [->] (30.87,52.76) -- (30.36,52.04);
        \draw [->] (31.17,52.85) -- (32,52.31);
    \end{tikzpicture}
\end{center}
Возьмем точечный заряд, окружим положительный заряд сферой радиуса $\vec r$
\[N = \oiint\limits_S \vec E \cdot \vec n \cdot dS \]
Очевидно, что $\vec E, \vec n$ будут сонаправлены, где $\vec n$ -  нормаль ко всей сфере. Распишем эту формулу:
\[N = \oiint\limits_S \vec E \cdot \vec n \cdot dS = \oiint \limits_S \frac{1}{4 \pi \epsilon_0}\frac{q_1 q_2}{r^2} dS
    = \frac{1}{4 \pi \epsilon_0}\frac{q_1 q_2}{r^2} \oiint \limits_S dS = \frac{1}{4 \pi \epsilon_0}\frac{q_1 q_2}{r^2} \cdot 4 \pi r^2
    = \frac{q}{\epsilon_0}\]
Тогда можно сделать вывод, что число линий проходящих через любую замкнутую поверхность:
\[ N = \frac{q}{\epsilon_0}\]

В произвольном случае, в котором имеется несколько электрических \linebreak полей: $\vec E = \sum_{i = 1}^{n} \vec E_i $ получаем:
\[ \oiint\limits_S \vec E \cdot \vec n \, dS = \oiint\limits_S \sum_{i=1}^{n} \vec E_i \cdot \vec n \, dS =
    \sum_{i=1}^{n} \oiint\limits_S \vec E_i \cdot \vec n \, dS = \sum_{i=1}^{n} \frac{q_i}{\epsilon_0} =
    \frac{1}{\epsilon_0}\sum_{i=1}^{n} q_i\]
\begin{center}
    где $q_i$ находятся внутри замкнутой поверхности S
\end{center}


Таким образом, теорема Гаусса может быть записана таким образом:
\[ \oiint\limits_S \vec E \cdot \vec n \,  dS = \frac{1}{\epsilon_0} \sum_{i=1}^{n} q_i\]
\begin{center}
    или
\end{center}
\[\iint \limits_S \vec E \cdot \vec n \, dS  = \frac{1}{\epsilon_0} \iiint\limits_D \rho_\epsilon \, dV\]
\begin{center}
    где D - объем, заключенный внутри поверхности S,\linebreak $\rho$ - объемная плотность электрического заряда
\end{center}


\define \textbf{Объемная плотность заряда} $\rho_q$ - общий заряд который имеется в объеме деленный на объем, то есть заряд в единице объема.
\[ \rho_q(x,y,z) = \lim_{\Delta V \to 0} \frac{\Delta q}{\Delta V}\]
Как правило распределена неравномерно.Для того чтобы получить заряд в точке уменьшаем объем до точки, то есть делаем бесконечно малый объем.

\define \textbf{Поверхностная плотность заряда} - средний заряд на единицу площади поверхности.

\[ \sigma = \lim_{\Delta S \to 0} \frac{\Delta q}{\Delta S}\]


\define \textbf{Линейная плотность заряда}  - собственно на линии, когда площадь стремится к нулю соответственно.

\[ \lambda = \lim_{\Delta \ell \to 0} \frac{\Delta q}{\Delta \ell}\]

\subsection{Поле бесконечно однородно заряженной плоскости}
\begin{center}
    \definecolor{qqttcc}{rgb}{0,0.2,0.8}
    \definecolor{ffttqq}{rgb}{1,0.2,0}
    \definecolor{qqqqff}{rgb}{0,0,1}
    \definecolor{ffqqqq}{rgb}{1,0,0}
    \begin{tikzpicture}[line cap=round,line join=round,>=triangle 45,x=1.0cm,y=1.0cm]
        \clip(27.04,62.43) rectangle (34.53,67.4);
        \draw [line width=1.6pt,color=ffqqqq] (29.17,45.85) circle (0.28cm);
        \draw [color=ffqqqq](29.04,46.11) node[anchor=north west] {$+$};
        \draw [line width=1.6pt,color=qqqqff] (31.87,45.96) circle (0.28cm);
        \draw [color=qqqqff](31.63,46.24) node[anchor=north west] {$-$};
        \draw [line width=1.6pt,color=ffttqq] (29,67)-- (29,63);
        \draw [line width=1.6pt,color=qqttcc] (32,67)-- (32,63);
        \draw [->] (29,66.64) -- (30.26,66.62);
        \draw [->] (29.03,64) -- (30.29,63.98);
        \draw [->] (29.04,65.3) -- (30.3,65.28);
        \draw [->] (30.69,66.61) -- (31.96,66.6);
        \draw [->] (30.71,65.31) -- (31.97,65.3);
        \draw [->] (30.74,64.01) -- (32,64);
        \draw [->] (29,66.64) -- (27.87,66.64);
        \draw [->] (28.99,65.31) -- (27.86,65.31);
        \draw [->] (28.96,64) -- (27.83,64);
        \draw [->] (33.08,66.6) -- (31.96,66.6);
        \draw [->] (33.1,65.3) -- (31.97,65.3);
        \draw [->] (33.13,64) -- (32,64);
        \draw (29.11,67.33) node[anchor=north west] {$+ \sigma $};
        \draw (32.13,67.28) node[anchor=north west] {$- \sigma$};
    \end{tikzpicture}
\end{center}
Пусть есть заряд $\sigma$ в некотором поле. Рассмотрим линии напряженности:в силу симметрии линии напряженности будут перпендикулярны плоскости, с плюсом от нее и с минусом к ней соответственно.
Воспользуемся \textbf{теоремой Гаусса} для определения напряженности этого поля.

\vspace{5px}

Выделим замкнутую поверхность: выделим цилиндр с площадью основы S, посчитаем поток через эту цилиндрическую поверхность:
\begin{center}
    \definecolor{qqttcc}{rgb}{0,0.2,0.8}
    \definecolor{ffttqq}{rgb}{1,0.2,0}
    \definecolor{qqqqff}{rgb}{0,0,1}
    \definecolor{ffqqqq}{rgb}{1,0,0}
    \begin{tikzpicture}[line cap=round,line join=round,>=triangle 45,x=1.0cm,y=1.0cm]
        \clip(24.1,85.59) rectangle (32.42,94.37);
        \draw [line width=1.6pt,color=ffqqqq] (29.17,45.85) circle (0.28cm);
        \draw [color=ffqqqq](29.03,46.11) node[anchor=north west] {$+$};
        \draw [line width=1.6pt,color=qqqqff] (31.87,45.96) circle (0.28cm);
        \draw [color=qqqqff](31.63,46.24) node[anchor=north west] {$-$};
        \draw [line width=1.6pt,color=ffttqq] (29,67)-- (29,63);
        \draw [line width=1.6pt,color=qqttcc] (32,67)-- (32,63);
        \draw [->] (29,66.64) -- (30.26,66.62);
        \draw [->] (29.03,64) -- (30.29,63.98);
        \draw [->] (29.04,65.3) -- (30.3,65.28);
        \draw [->] (30.69,66.61) -- (31.96,66.6);
        \draw [->] (30.71,65.31) -- (31.97,65.3);
        \draw [->] (30.74,64.01) -- (32,64);
        \draw [->] (29,66.64) -- (27.87,66.64);
        \draw [->] (28.99,65.31) -- (27.86,65.31);
        \draw [->] (28.96,64) -- (27.83,64);
        \draw [->] (33.08,66.6) -- (31.96,66.6);
        \draw [->] (33.1,65.3) -- (31.97,65.3);
        \draw [->] (33.13,64) -- (32,64);
        \draw (29.1,67.33) node[anchor=north west] {$+ \sigma $};
        \draw (32.12,67.28) node[anchor=north west] {$- \sigma$};
        \draw [line width=2pt,color=ffttqq] (28,94)-- (28,86);
        \draw [line width=0.4pt,dash pattern=on 2pt off 2pt] (26,91)-- (30,91);
        \draw [line width=0.4pt,dash pattern=on 2pt off 2pt] (26,89)-- (30,89);
        \draw [rotate around={90:(26,90)},dash pattern=on 2pt off 2pt] (26,90) ellipse (1.02cm and 0.22cm);
        \draw [rotate around={90:(30,90)},dash pattern=on 2pt off 2pt] (30,90) ellipse (1.02cm and 0.21cm);
        \draw [->] (30,90) -- (31.38,90.01);
        \draw [->] (26,90) -- (24.55,89.99);
        \draw [->] (28,87) -- (30,87);
        \draw [->] (28,87) -- (26,87);
        \draw [->] (28,93) -- (30,93);
        \draw [->] (28,93) -- (26,93);
        \draw [->] (29,91) -- (29,92);
        \draw [->] (29,89) -- (29,88);
        \draw [rotate around={90:(28,90)},dash pattern=on 2pt off 2pt] (28,90) ellipse (1.03cm and 0.24cm);
        \draw (27.76,90.34) node[anchor=north west] {$S$};
        \draw (28.04,94.12) node[anchor=north west] {$+ \sigma$};
        \draw (30.58,90.54) node[anchor=north west] {$\vec n$};
        \draw (29.06,91.81) node[anchor=north west] {$\vec n$};
        \draw (25.06,90.55) node[anchor=north west] {$\vec n$};
        \draw (29.13,88.83) node[anchor=north west] {$\vec n$};
        \draw (28.13,87.61) node[anchor=north west] {$\vec E$};
        \draw (29.14,88.48) node[anchor=north west] {$\vec E \cdot \vec n = 0$};
    \end{tikzpicture}
\end{center}

Первый способ, из физических соображений, используя формулу потока:
\[ \Phi_E = \oiint\limits_S \vec E \cdot \vec n dS = \frac{1}{\epsilon_0} \sum_{}^{} q
    = \frac{1}{\epsilon_0} \sigma \cdot S \]

Второй способ, возьмем тот же цилиндр. Рассмотрим боковую поверхность цилиндра, поверхностный интеграл можно расписать как сумму соотвенно поверхностных интегралов по боковой поверхности и по основаниям.

\vspace{5px}

Заметим, что нормаль к боковой поверхности будет перпендикулярно силовым линиям, то есть:
\[ \oiint\limits_S \vec E \cdot \vec n dS = \iint\limits_{S_{\text{бок}}} \vec E \cdot \vec n dS + 2\iint\limits_{S_{\text{осн}}} \vec E \cdot \vec n dS
    = 2 E \cdot S \]

Получается, можно приравнять, потому что мы считаем одну и ту же площадь одно и того же тела разными способами, то есть:
\[\frac{1}{\epsilon_0} \sigma S = 2 \cdot E \cdot S \Rightarrow E = \frac{\sigma}{2 \epsilon_0}\]

В реальности не существует бесконечной пластины, а у конечной пластины линии напряженности таковы:
\begin{center}
    \definecolor{qqttcc}{rgb}{0,0.2,0.8}
    \definecolor{ffttqq}{rgb}{1,0.2,0}
    \definecolor{qqqqff}{rgb}{0,0,1}
    \definecolor{ffqqqq}{rgb}{1,0,0}
    \begin{tikzpicture}[line cap=round,line join=round,>=triangle 45,x=1.0cm,y=1.0cm]
        \clip(25.66,86.82) rectangle (30.45,93.45);
        \draw [line width=1.6pt,color=ffqqqq] (29.17,45.85) circle (0.28cm);
        \draw [color=ffqqqq](29.03,46.11) node[anchor=north west] {$+$};
        \draw [line width=1.6pt,color=qqqqff] (31.87,45.96) circle (0.28cm);
        \draw [color=qqqqff](31.63,46.24) node[anchor=north west] {$-$};
        \draw [line width=1.6pt,color=ffttqq] (29,67)-- (29,63);
        \draw [line width=1.6pt,color=qqttcc] (32,67)-- (32,63);
        \draw [->] (29,66.64) -- (30.26,66.62);
        \draw [->] (29.03,64) -- (30.29,63.98);
        \draw [->] (29.04,65.3) -- (30.3,65.28);
        \draw [->] (30.69,66.61) -- (31.96,66.6);
        \draw [->] (30.71,65.31) -- (31.97,65.3);
        \draw [->] (30.74,64.01) -- (32,64);
        \draw [->] (29,66.64) -- (27.87,66.64);
        \draw [->] (28.99,65.31) -- (27.86,65.31);
        \draw [->] (28.96,64) -- (27.83,64);
        \draw [->] (33.08,66.6) -- (31.96,66.6);
        \draw [->] (33.1,65.3) -- (31.97,65.3);
        \draw [->] (33.13,64) -- (32,64);
        \draw (29.1,67.33) node[anchor=north west] {$+ \sigma $};
        \draw (32.12,67.28) node[anchor=north west] {$- \sigma$};
        \draw [line width=2pt,color=ffttqq] (28,93)-- (28,87);
        \draw [rotate around={90:(28,90)},dash pattern=on 1pt off 1pt] (28,90) ellipse (1.02cm and 0.21cm);
        \draw [shift={(28,93.5)}] plot[domain=3.79:5.64,variable=\t]({1*2.5*cos(\t r)+0*2.5*sin(\t r)},{0*2.5*cos(\t r)+1*2.5*sin(\t r)});
        \draw (26,90)-- (30,90);
        \draw [shift={(28,86.5)}] plot[domain=0.64:2.5,variable=\t]({1*2.5*cos(\t r)+0*2.5*sin(\t r)},{0*2.5*cos(\t r)+1*2.5*sin(\t r)});
        \draw [shift={(28,93)}] plot[domain=-3.14:0,variable=\t]({1*1*cos(\t r)+0*1*sin(\t r)},{0*1*cos(\t r)+1*1*sin(\t r)});
        \draw [shift={(28,87)}] plot[domain=0:3.14,variable=\t]({1*1*cos(\t r)+0*1*sin(\t r)},{0*1*cos(\t r)+1*1*sin(\t r)});
    \end{tikzpicture}
\end{center}
\subsection{Электрическое поле двух разноименно заряженных плоскостей}
\begin{center}
    \definecolor{qqttcc}{rgb}{0,0.2,0.8}
    \definecolor{ffttqq}{rgb}{1,0.2,0}
    \definecolor{qqqqff}{rgb}{0,0,1}
    \definecolor{ffqqqq}{rgb}{1,0,0}
    \begin{tikzpicture}[line cap=round,line join=round,>=triangle 45,x=1.0cm,y=1.0cm]
        \clip(24.82,85.73) rectangle (32.76,92.3);
        \draw [line width=1.6pt,color=ffqqqq] (29.17,45.85) circle (0.28cm);
        \draw [color=ffqqqq](29.03,46.11) node[anchor=north west] {$+$};
        \draw [line width=1.6pt,color=qqqqff] (31.87,45.96) circle (0.28cm);
        \draw [color=qqqqff](31.63,46.24) node[anchor=north west] {$-$};
        \draw [line width=1.6pt,color=ffttqq] (29,67)-- (29,63);
        \draw [line width=1.6pt,color=qqttcc] (32,67)-- (32,63);
        \draw [->] (29,66.64) -- (30.26,66.62);
        \draw [->] (29.03,64) -- (30.29,63.98);
        \draw [->] (29.04,65.3) -- (30.3,65.28);
        \draw [->] (30.69,66.61) -- (31.96,66.6);
        \draw [->] (30.71,65.31) -- (31.97,65.3);
        \draw [->] (30.74,64.01) -- (32,64);
        \draw [->] (29,66.64) -- (27.87,66.64);
        \draw [->] (28.99,65.31) -- (27.86,65.31);
        \draw [->] (28.96,64) -- (27.83,64);
        \draw [->] (33.08,66.6) -- (31.96,66.6);
        \draw [->] (33.1,65.3) -- (31.97,65.3);
        \draw [->] (33.13,64) -- (32,64);
        \draw (29.1,67.33) node[anchor=north west] {$+ \sigma $};
        \draw (32.12,67.28) node[anchor=north west] {$- \sigma$};
        \draw [line width=2pt,color=ffttqq] (26,92)-- (26,86);
        \draw [line width=2.4pt,color=qqttcc] (31,92)-- (31,86);
        \draw (26.25,92.27) node[anchor=north west] {$+ \sigma$};
        \draw (31.2,92.32) node[anchor=north west] {$- \sigma$};
        \draw [->] (26,91) -- (25,91);
        \draw [->] (26.04,89) -- (25,89);
        \draw [->] (26.04,87) -- (25,87);
        \draw [->] (32.08,91.64) -- (31.04,91.64);
        \draw [->] (32.04,89.89) -- (31.03,89.85);
        \draw [->] (26,91) -- (32,91);
        \draw [->] (32,88) -- (31,88);
        \draw [->] (26,89) -- (32,89);
        \draw [->] (26,87) -- (32,87);
        \draw [color=ffttqq](25.27,91.64) node[anchor=north west] {$\vec E_1$};
        \draw [color=ffttqq](25.29,89.66) node[anchor=north west] {$\vec E_1$};
        \draw [color=ffttqq](25.34,87.71) node[anchor=north west] {$\vec E_1$};
        \draw [->] (25,91.65) -- (31.04,91.64);
        \draw [->] (24.99,89.87) -- (31.03,89.85);
        \draw [->] (24.96,88.01) -- (31,88);
        \draw [color=qqttcc](30.12,92.24) node[anchor=north west] {$\vec E_2$};
        \draw [color=qqttcc](30.15,90.5) node[anchor=north west] {$\vec E_2$};
        \draw [color=qqttcc](30.19,88.75) node[anchor=north west] {$\vec E_2$};
        \draw (25.25,86.85) node[anchor=north west] {$I$};
        \draw (28.02,86.85) node[anchor=north west] {$II$};
        \draw (31.08,86.85) node[anchor=north west] {$III$};
    \end{tikzpicture}

\end{center}
Из геометрических соображений получаем, что поле будет сосредоточено только между
этими плоскостями: $I, III : E = 0; II = \frac{\sigma}{\epsilon_0}$.

Причем такое поле однородно и направленно в одну сторону.
\subsection{Поле бесконечного однородно заряженного цилиндра}
\begin{center}
    \definecolor{qqttcc}{rgb}{0,0.2,0.8}
    \definecolor{ffttqq}{rgb}{1,0.2,0}
    \definecolor{qqqqff}{rgb}{0,0,1}
    \definecolor{ffqqqq}{rgb}{1,0,0}
    \begin{tikzpicture}[line cap=round,line join=round,>=triangle 45,x=1.0cm,y=1.0cm]
        \clip(23.81,85.51) rectangle (32.28,92.21);
        \draw [line width=1.6pt,color=ffqqqq] (29.17,45.85) circle (0.28cm);
        \draw [color=ffqqqq](29.03,46.11) node[anchor=north west] {$+$};
        \draw [line width=1.6pt,color=qqqqff] (31.87,45.96) circle (0.28cm);
        \draw [color=qqqqff](31.63,46.24) node[anchor=north west] {$-$};
        \draw [line width=1.6pt,color=ffttqq] (29,67)-- (29,63);
        \draw [line width=1.6pt,color=qqttcc] (32,67)-- (32,63);
        \draw [->] (29,66.64) -- (30.26,66.62);
        \draw [->] (29.03,64) -- (30.29,63.98);
        \draw [->] (29.04,65.3) -- (30.3,65.28);
        \draw [->] (30.69,66.61) -- (31.96,66.6);
        \draw [->] (30.71,65.31) -- (31.97,65.3);
        \draw [->] (30.74,64.01) -- (32,64);
        \draw [->] (29,66.64) -- (27.87,66.64);
        \draw [->] (28.99,65.31) -- (27.86,65.31);
        \draw [->] (28.96,64) -- (27.83,64);
        \draw [->] (33.08,66.6) -- (31.96,66.6);
        \draw [->] (33.1,65.3) -- (31.97,65.3);
        \draw [->] (33.13,64) -- (32,64);
        \draw (29.1,67.33) node[anchor=north west] {$+ \sigma $};
        \draw (32.12,67.28) node[anchor=north west] {$- \sigma$};
        \draw (23.75,86.91)-- (29.75,91.91);
        \draw (26.55,85.9)-- (31,90);
        \draw [line width=1.2pt,dash pattern=on 1pt off 1pt] (24.1,85.49)-- (31.37,91.81);
        \draw(27.99,88.88) circle (1.21cm);
        \draw (23.85,86.08) node[anchor=north west] {\textit{вообще это бесконечный цилиндр,думайте}};
        \draw [->] (27.99,88.88) -- (30,87);
        \draw [->] (29.26,88.4) -- (29.99,87.78);
        \draw [->] (29.96,89.04) -- (30.79,88.34);
        \draw [->] (30.63,89.65) -- (31.5,88.95);
        \draw (29.56,91.92) node[anchor=north west] {$\sigma$};
        \draw (30.69,90.25) node[anchor=north west] {$\sigma$};
        \draw (28.46,89.4) node[anchor=north west] {$R$};
        \draw (29.17,88.25) node[anchor=north west] {$\vec r$};
        \draw (29.64,88.69) node[anchor=north west] {$\vec E$};
        \draw (30.45,89.28) node[anchor=north west] {$\vec E$};
        \draw (31.1,89.83) node[anchor=north west] {$\vec E$};
    \end{tikzpicture}
\end{center}

По поверхности цилиндра распределен заряд с плотностью $\sigma$, в силу симметрии наши линии напряженности будут перпендикулярны поверхности цилиндра.
Возьмем некоторую точку, и пусть расстояние до нее от цилиндра r.

\vspace{5px}

Поступим аналогично: применим \textit{теорему Гаусса}, возьмем замкнутую поверхность в виде цилиндра который окружает наш цилиндр и посчитаем поверхностные интегралы:
\begin{center}
    \definecolor{qqzzqq}{rgb}{0,0.6,0}
    \definecolor{fftttt}{rgb}{1,0.2,0.2}
    \definecolor{qqttcc}{rgb}{0,0.2,0.8}
    \definecolor{qqqqff}{rgb}{0,0,1}
    \definecolor{ffqqqq}{rgb}{1,0,0}
    \begin{tikzpicture}[line cap=round,line join=round,>=triangle 45,x=1.0cm,y=1.0cm]
        \clip(24.18,85.53) rectangle (32.29,91.39);
        \draw [line width=1.6pt,color=ffqqqq] (29.17,45.85) circle (0.28cm);
        \draw [color=ffqqqq](29.03,46.11) node[anchor=north west] {$+$};
        \draw [line width=1.6pt,color=qqqqff] (31.87,45.96) circle (0.28cm);
        \draw [color=qqqqff](31.63,46.24) node[anchor=north west] {$-$};
        \draw [line width=1.6pt,color=qqttcc] (32,67)-- (32,63);
        \draw [->] (29,66.64) -- (30.26,66.62);
        \draw [->] (29.03,64) -- (30.29,63.98);
        \draw [->] (29.04,65.3) -- (30.3,65.28);
        \draw [->] (30.69,66.61) -- (31.96,66.6);
        \draw [->] (30.71,65.31) -- (31.97,65.3);
        \draw [->] (30.74,64.01) -- (32,64);
        \draw [->] (29,66.64) -- (27.87,66.64);
        \draw [->] (28.99,65.31) -- (27.86,65.31);
        \draw [->] (28.96,64) -- (27.83,64);
        \draw [->] (33.08,66.6) -- (31.96,66.6);
        \draw [->] (33.1,65.3) -- (31.97,65.3);
        \draw [->] (33.13,64) -- (32,64);
        \draw (29.1,67.33) node[anchor=north west] {$+ \sigma $};
        \draw (32.12,67.28) node[anchor=north west] {$- \sigma$};
        \draw [color=fftttt] (26,88)-- (29,90);
        \draw [color=fftttt] (27,87)-- (30,89);
        \draw [rotate around={-48.83:(28.02,88.51)},line width=1.6pt,color=fftttt] (28.02,88.51) ellipse (0.73cm and 0.22cm);
        \draw [line width=1pt,dash pattern=on 1pt off 2pt] (29.52,91.17)-- (24.81,88.12);
        \draw [line width=1pt,dash pattern=on 1pt off 2pt] (31.67,88.78)-- (26.89,85.67);
        \draw [rotate around={-49.63:(25.85,86.89)},line width=1pt,dash pattern=on 1pt off 2pt] (25.85,86.89) ellipse (1.63cm and 0.26cm);
        \draw [rotate around={-47.98:(30.6,89.97)},line width=1pt,dash pattern=on 1pt off 2pt] (30.6,89.97) ellipse (1.63cm and 0.29cm);
        \draw [->] (26.21,89.03) -- (25.34,90.17);
        \draw [->] (25.8,86.87) -- (24.53,85.99);
        \draw [->] (30.53,90) -- (31.75,91.04);
        \draw [->] (26.67,89.33) -- (25.66,90.54);
        \draw [rotate around={-48.67:(28.13,88.37)},line width=1pt,dash pattern=on 1pt off 2pt] (28.13,88.37) ellipse (1.65cm and 0.38cm);
        \draw [->,color=qqzzqq] (28,88.54) -- (29.19,87.16);
        \draw [->,color=fftttt] (28,88.54) -- (27.56,89.04);
        \draw [color=fftttt](27.04,88.71) node[anchor=north west] {$R$};
        \draw [color=qqzzqq](28.76,88.1) node[anchor=north west] {$\vec r$};
        \draw (25.91,89.79) node[anchor=north west] {$\vec n$};
        \draw (24.69,86.74) node[anchor=north west] {$\vec n$};
        \draw (31.35,90.63) node[anchor=north west] {$\vec n$};
        \draw (26.38,90.2) node[anchor=north west] {$\vec E$};
    \end{tikzpicture}
\end{center}

Сначала рассматриваем поток сквозь этот цилиндр:
\[\Phi_E = \oiint\limits_S \vec E \cdot \vec n ds = \frac{1}{\epsilon_0} 2 \pi R \ell \cdot \sigma \]

И разобьем на поверхности соответственно: на основаниях нормаль будет перпендикулярно заряду, а соответственно нормали $n$ боковой поверхности будут со-направлены векторам силового поля, то есть получаем:

\[\oiint\limits_S \vec E \cdot \vec n ds = \iint\limits_{S_\text{бок}} \vec E \cdot \vec n dS + 2\iint\limits_{S_\text{осн}} \vec E \cdot \vec n dS = E \iint dS = 2 E \pi r \ell \]
\linebreak
\begin{center}
    где
\end{center}
\[2\iint\limits_{S_\text{осн}} \vec E \cdot \vec n dS = 0\]

Тогда сопоставляя два способа получаем(r - расстояние от оси цилиндра):
\[\frac{1}{\epsilon_0}R\sigma = Er \Rightarrow E = \frac{\sigma R}{\epsilon_0 r}\]

Формула выше справедлива только для случая когда $r > R$, при $r < R \Rightarrow E = 0$.

\vspace{5px}

Если $r >> R$ в этом случае мы переходим от поверхностной плотности заряда к линейной. Итак $\lambda = \sigma \cdot 2\pi R$, подставим:

\[ E = \frac{\lambda}{2\pi \epsilon_0 r}\]

\subsection{Сферически однородно заряженная поверхность}
\begin{center}
    \definecolor{ffttqq}{rgb}{1,0.2,0}
    \definecolor{qqwwqq}{rgb}{0,0.4,0}
    \definecolor{fftttt}{rgb}{1,0.2,0.2}
    \definecolor{qqttcc}{rgb}{0,0.2,0.8}
    \definecolor{qqqqff}{rgb}{0,0,1}
    \definecolor{ffqqqq}{rgb}{1,0,0}
    \begin{tikzpicture}[line cap=round,line join=round,>=triangle 45,x=1.0cm,y=1.0cm]
        \clip(22.43,84.42) rectangle (31.77,93.62);
        \draw [color=fftttt,fill=fftttt,fill opacity=0.1] (27,89) circle (2.64cm);
        \draw [line width=1pt,color=ffqqqq] (29.17,45.85) circle (0.28cm);
        \draw [color=ffqqqq](29.03,46.11) node[anchor=north west] {$+$};
        \draw [line width=1.6pt,color=qqqqff] (31.87,45.96) circle (0.28cm);
        \draw [color=qqqqff](31.63,46.24) node[anchor=north west] {$-$};
        \draw [line width=1.6pt,color=qqttcc] (32,67)-- (32,63);
        \draw [->] (29,66.64) -- (30.26,66.62);
        \draw [->] (29.03,64) -- (30.29,63.98);
        \draw [->] (29.04,65.3) -- (30.3,65.28);
        \draw [->] (30.69,66.61) -- (31.96,66.6);
        \draw [->] (30.71,65.31) -- (31.97,65.3);
        \draw [->] (30.74,64.01) -- (32,64);
        \draw [->] (29,66.64) -- (27.87,66.64);
        \draw [->] (28.99,65.31) -- (27.86,65.31);
        \draw [->] (28.96,64) -- (27.83,64);
        \draw [->] (33.08,66.6) -- (31.96,66.6);
        \draw [->] (33.1,65.3) -- (31.97,65.3);
        \draw [->] (33.13,64) -- (32,64);
        \draw (29.1,67.33) node[anchor=north west] {$+ \sigma $};
        \draw (32.12,67.28) node[anchor=north west] {$- \sigma$};
        \draw [line width=1pt,dash pattern=on 1pt off 2pt] (27,89) circle (3.76cm);
        \draw [rotate around={0:(27.01,89.01)},line width=1pt,dash pattern=on 1pt off 2pt] (27.01,89.01) ellipse (3.74cm and 0.63cm);
        \draw [rotate around={-0.28:(27,88.97)},color=fftttt,fill=fftttt,fill opacity=0.15] (27,88.97) ellipse (2.68cm and 0.45cm);
        \draw [->] (27,89) -- (29.68,88.96);
        \draw (28.1,89.41) node[anchor=north west] {$R$};
        \draw (27.75,91.48) node[anchor=north west] {$+ \sigma$};
        \draw [->,color=qqwwqq] (27,89) -- (24.35,91.66);
        \draw [->,color=ffttqq] (28.99,90.74) -- (29.7,91.22);
        \draw [->,color=ffttqq] (27.76,86.47) -- (27.98,85.73);
        \draw [->,color=ffttqq] (24.64,87.81) -- (23.87,87.37);
        \draw (29.06,91.51) node[anchor=north west] {$\vec E$};
        \draw (24,88.16) node[anchor=north west] {$\vec E$};
        \draw (27.97,86.53) node[anchor=north west] {$\vec E$};
        \draw [color=qqwwqq](24.91,91.54) node[anchor=north west] {$\vec r$};
    \end{tikzpicture}
\end{center}

Линии напряженности будут перпендикулярны поверхности сферы.
По аналогии с предыдущими доказательствами, возьмем сферу окружающую нашу сферу и применим \textit{теорему Гаусса:}
\[\Phi_E = \oiint\limits_{S_r} \vec{E} \cdot  \vec{n} dS = \frac{1}{\epsilon_0}\sigma 4 \pi R^2\]

Или можно посчитать с помощью формулы площади поверхности сферы:
\[ \Phi_E = \oiint\limits_{S_r} \vec{E} \cdot \vec{n} dS = E \cdot \oiint\limits_{S_r} \vec{n} dS = 4 \pi R^2 \cdot E\]

При этом сфера замкнута и мы можем посчитать общий заряд на сфере и он будет равен:

\[ q = \sigma \cdot 4\pi R^2 \Rightarrow \sigma = \frac{q}{4\pi R^2}\]

И тогда при $r > R :$
\[ E = \frac{q}{4 \pi \epsilon_0 r^2} \]

Заметим, что напряженность поля точечного заряда совпадает с этой формулой.

\subsection{Работа сил электростатического поля}

Рассмотрим два заряда: заряд создающий поле соответственно и пробный заряд,произвольно движущийся в этом поле. Нужно посчитать работу по перемещению этого заряда в поле.

\vspace{5px}

Заметим, что действие силы поля напоминает действие центральной силы, а именно что вектор силы будет на одной линии с неподвижной точкой, к тому же любая центральная сила консервативна и тогда пределы интегрирования будут определяться радиус векторами от начальной $q_0$ до первой точки и до второй соответственно:

\begin{center}
    \definecolor{qqwwqq}{rgb}{0,0.4,0}
    \definecolor{qqttcc}{rgb}{0,0.2,0.8}
    \definecolor{qqqqff}{rgb}{0,0,1}
    \definecolor{ffqqqq}{rgb}{1,0,0}
    \begin{tikzpicture}[line cap=round,line join=round,>=triangle 45,x=1.0cm,y=1.0cm]
        \clip(22.52,87.6) rectangle (30.35,92.61);
        \draw [line width=1.6pt,color=ffqqqq] (29.17,45.85) circle (0.28cm);
        \draw [color=ffqqqq](29.03,46.11) node[anchor=north west] {$+$};
        \draw [line width=1.6pt,color=qqqqff] (31.87,45.96) circle (0.28cm);
        \draw [color=qqqqff](31.63,46.24) node[anchor=north west] {$-$};
        \draw [line width=1.6pt,color=qqttcc] (32,67)-- (32,63);
        \draw [->] (29,66.64) -- (30.26,66.62);
        \draw [->] (29.03,64) -- (30.29,63.98);
        \draw [->] (29.04,65.3) -- (30.3,65.28);
        \draw [->] (30.69,66.61) -- (31.96,66.6);
        \draw [->] (30.71,65.31) -- (31.97,65.3);
        \draw [->] (30.74,64.01) -- (32,64);
        \draw [->] (29,66.64) -- (27.87,66.64);
        \draw [->] (28.99,65.31) -- (27.86,65.31);
        \draw [->] (28.96,64) -- (27.83,64);
        \draw [->] (33.08,66.6) -- (31.96,66.6);
        \draw [->] (33.1,65.3) -- (31.97,65.3);
        \draw [->] (33.13,64) -- (32,64);
        \draw (29.1,67.33) node[anchor=north west] {$+ \sigma $};
        \draw (32.12,67.28) node[anchor=north west] {$- \sigma$};
        \draw [shift={(25.57,94.32)}] plot[domain=4.11:5.03,variable=\t]({1*3.35*cos(\t r)+0*3.35*sin(\t r)},{0*3.35*cos(\t r)+1*3.35*sin(\t r)});
        \draw [shift={(26.51,87.81)}] plot[domain=0.72:1.54,variable=\t]({1*3.32*cos(\t r)+0*3.32*sin(\t r)},{0*3.32*cos(\t r)+1*3.32*sin(\t r)});
        \draw [->,dash pattern=on 1pt off 2pt] (25,88.53) -- (23.66,91.56);
        \draw [->,dash pattern=on 1pt off 2pt] (25,88.53) -- (29,90);
        \draw (23.48,91.59) node[anchor=north west] {$1$};
        \draw (28.9,89.96) node[anchor=north west] {$2$};
        \draw (23.61,92.07) node[anchor=north west] {$q$};
        \draw [->,color=qqwwqq] (23.66,91.56) -- (23.37,92.32);
        \draw [->,color=qqwwqq] (29,90) -- (29.8,90.29);
        \draw [color=qqwwqq](23.01,91.86) node[anchor=north west] {$\vec F$};
        \draw [color=qqwwqq](29.38,90.21) node[anchor=north west] {$\vec F$};
        \draw (24.82,88.53) node[anchor=north west] {$q_0$};
        \draw (24.32,90.56) node[anchor=north west] {$r_1$};
        \draw (27.13,89.87) node[anchor=north west] {$r_2$};
    \end{tikzpicture}
\end{center}

\[ A = \int\limits_{\hat{r_1 r_2}} \vec{F} \cdot d\vec{r} = \frac{1}{4 \pi \epsilon_0} \int\limits_{r_1}^{r_2} \frac{q_0 q}{r^2}dr = \frac{q_0 q}{4 \pi \epsilon_0} \left(\frac{1}{r_1} - \frac{1}{r_2}\right) \]
Соответственно, если $r_1 = r_2$, то А = 0.

\textit{При этом не обязательно что это будет одна и та же точка, главное чтобы радиус векторы совпадали по длине.}

\subsection{Потенциал электрического поля}
\define \textbf{Потенциалом электрического поля} называется величина равная отношению потенциальной энергии заряженной частицы к величине этого заряда.

Рассмотрим замкнутую дугу, то есть контур:

\[A = \oint\limits_q \vec{E} d\vec{\ell} = 0\]

Так как наш сила консервативная, то можем ввести понятие потенциальной энергии, то есть:

\[\frac{q_0 q}{4 \pi \epsilon_0} \left(\frac{1}{r_1} - \frac{1}{r_2}\right) = W_1 - W_2\]

\[W = \frac{q_0 q}{4 \pi \epsilon r} + C\]

Для того чтобы посчитать эту энергию, нужно взять точку в которой потенциальная энергия равна нулю, а именно можно взять точку расстояние от которой приближается к бесконечности.

\vspace{5px}

Тогда для нее $W \to 0$ при $r \to \infty$ получим что $C = 0$:

\[W = \frac{q_0 q}{4 \pi \epsilon_0 r}\]


Потенциальную энергию поле не принято считать характеристикой этого поля, поскольку зависит от заряда помещенного в это поле.
Тогда рассматривают величину потенциальной энергии деленное на величину заряда помещенное в это поле - она и будет являться характеристикой этого поля.


\[\phi = \frac{W}{q}\]

\textit{Если есть несколько частиц с потенциалами соответственно $\phi_1, \phi_2 \ldots \phi_n$, то общий потенциал равен сумме этих потенциалов:}

\[\phi = \frac{q_0}{4 \pi \epsilon_0 r} \]
\[\phi = \sum_{i = 1}^{n} \phi_i  \]

\define Величину $A = W_1 - W_2 = q\phi_1 - q\phi_2 = q(\phi_1 - \phi_2) = q\Delta \phi$ называют \textbf{разностью потенциалов.}

Данную формулу можно использовать для вычисления потенциала электрических полей. Если возьмем $q = +1 Кл, r_2 \to \infty, \phi_2 = 0$, получим что
потенциал численно равен работе, которую совершают силы поля по перемещению единичного положительного заряда из данной точки на бесконечности.

Соответсвенно для нескольких заряженных тел работает принцип суперпозиции:
\[\phi = \sum_{i=1}^{N} \phi_i \]

Из формулы для вычисления потенциала: $\phi = \frac{W}{q} \Rightarrow [q] = \frac{\text{Дж}}{\text{Кл}} = \text{B}$

\subsubsection{Связь между напряженностью электрического поля и потенциалом}

Из прошлых формул $\phi = \frac{W}{q} , E = \frac{F}{q} , F = -grad W$. Выразим потенциальную энергию из $\phi = \frac{W}{q} , F = -grad W$ и силу из $E = \frac{F}{q}$, получим:

\[ q \cdot \vec E = grad(q \cdot \phi)\]
\[ \vec E = -grad \phi\]
Выразим работу по перемещению заряда из точки 1 в точку 2:
\[ A = \int\limits_{r_1}^{r_2} \vec F d \vec r = q \int\limits_{r_1}^{r_2} \vec E d \vec r \Rightarrow \]
Из формулы для разности потенциалов получим что:
\[q(\phi_1 - \phi_2) = q \int\limits_{r_1}^{r_2} \vec E d \vec r \Rightarrow\]
\[\phi_1 - \phi_2 = \int\limits_{r_1}^{r_2} \vec E d \vec r\]
Рассмотрим две пластинки:

\begin{center}
    \definecolor{xdxdff}{rgb}{0.49,0.49,1}
    \definecolor{qqqqff}{rgb}{0,0,1}
    \definecolor{ffqqtt}{rgb}{1,0,0.2}
    \begin{tikzpicture}[line cap=round,line join=round,>=triangle 45,x=1.0cm,y=1.0cm]
        \clip(12.3,12.77) rectangle (16.74,17.22);
        \draw [line width=1.2pt,color=ffqqtt] (13,17)-- (13,13);
        \draw [line width=1.2pt,color=qqqqff] (16,17)-- (16,13);
        \draw [->] (13,16) -- (16,16);
        \draw [->] (13,15) -- (16,15);
        \draw [->] (13,14) -- (16,14);
        \draw [color=ffqqtt](12.59,17.14) node[anchor=north west] {$+$};
        \draw [color=qqqqff](16.19,17.22) node[anchor=north west] {$-$};
        \draw [dash pattern=on 1pt off 1pt] (13.16,16.45)-- (13.16,13);
        \draw [dash pattern=on 1pt off 1pt] (13.16,16.45)-- (16,16.47);
        \draw (14.35,13.42) node[anchor=north west] {$d$};
        \draw (13.22,13.32) node[anchor=north west] {$1$};
        \draw (15.57,16.92) node[anchor=north west] {$2$};
        \draw (13.32,16.5) node[anchor=north west] {$a$};
        \begin{scriptsize}
        \end{scriptsize}
    \end{tikzpicture}
\end{center}


Поле между данными пластинками будет потенциальным,
следовательно работа не будет зависеть от пути, тогда выберем такой путь как на картинке и посчитает работу по перемещению из точки 1 в точки 2.


\[A = q \int\limits_{a_1} \vec E d \vec r + q \int\limits_{a_2} \vec E d \vec r = q \int\limits_{a}^{2} \vec E d \vec r\]
То есть получаем формулу, важно отметить что только для потенциальных полей:
\[A = E \cdot q \cdot d \Rightarrow q \Delta \phi = E \cdot q \cdot d\]


\define Поверхность равного потенциала называется эквипотенциальной поверхностью.
\begin{itemize}
    \item   Линии напряженности электрического поля всегда перпендикулярны эквипотенциальной поверхности.
    \item   Потенциал электрического поля убывает в направлении линии напряженности электрического поля.
\end{itemize}
\section{Диэлектрики}
\subsection{Электрическое поле в диэлектриках}
Пусть имеется молекула некоторого вещества, найдем в ней радиус-векторы центра положительных и отрицательных зарядов:
\[\vec r_i^{+} = \frac{\sum_{i} \vec r_i^{+} \cdot \vec q_i^{+}}{\vec q_i^{+}}\]
\[\vec r_i^{-} = \frac{\sum_{i} \vec r_i^{-} \cdot \vec q_i^{-}}{\vec q_i^{-}}\]
Заметим, что: $|\sum_{i} \vec q_i| = \sum_{i} q_i^{+}$

Можно разделить на полярные и неполярные молекулы:
\begin{itemize}
    \item $\vec r_i^{+} \neq \vec r_i^{-}$ - полярная молекула
    \item $\vec r_i^{+} = \vec r_i^{-}$ - неполярная молекула
\end{itemize}
--- Любая \textbf{полярная молекула} представляет собой электрический диполь, у которой дипольный момент $P = ql$

Поместим полярную молекулу в электрическое поле:

\begin{center}
    \definecolor{qqqqff}{rgb}{0,0,1}
    \definecolor{ffqqtt}{rgb}{1,0,0.2}
    \begin{tikzpicture}[line cap=round,line join=round,>=triangle 45,x=1.0cm,y=1.0cm]
        \clip(10.12,23.09) rectangle (15.7,26.12);
        \draw [line width=1.2pt,color=ffqqtt] (13,17)-- (13,13);
        \draw [line width=1.2pt,color=qqqqff] (16,17)-- (16,13);
        \draw [->] (13,16) -- (16,16);
        \draw [->] (13,15) -- (16,15);
        \draw [->] (13,14) -- (16,14);
        \draw [color=qqqqff](12.34,17.26) node[anchor=north west] {$-\sigma^{\prime}$};
        \draw [color=ffqqtt](16.08,17.34) node[anchor=north west] {$+\sigma^{\prime}$};
        \draw [->] (13.02,16.6) -- (16.02,16.6);
        \draw [color=qqqqff] (13.2,16.4) circle (0.17cm);
        \draw [color=ffqqtt] (13.6,16.4) circle (0.16cm);
        \draw [color=qqqqff] (14.2,16.4) circle (0.17cm);
        \draw [color=ffqqtt] (14.6,16.4) circle (0.16cm);
        \draw [color=qqqqff] (15.6,16.4) circle (0.17cm);
        \draw (13.2,16.4)-- (13.6,16.4);
        \draw (14.2,16.4)-- (14.6,16.4);
        \draw [color=ffqqtt] (15.2,16.4) circle (0.16cm);
        \draw (15.2,16.4)-- (15.6,16.4);
        \draw [->] (11,24) -- (15,24);
        \draw [color=ffqqtt] (12.06,24.53) circle (0.16cm);
        \draw [color=qqqqff] (13.63,25.55) circle (0.17cm);
        \draw (12.13,24.68)-- (13.47,25.51);
        \draw [->] (12.18,24.43) -- (13.62,24.44);
        \draw [->] (13.53,25.69) -- (11.97,25.67);
        \draw (12.4,25.15)-- (13.39,25.16);
        \draw (14.59,24.85) node[anchor=north west] {$\vec E$};
        \draw [color=ffqqtt](11.77,24.78) node[anchor=north west] {$+$};
        \draw [color=qqqqff](13.37,25.8) node[anchor=north west] {$-$};
    \end{tikzpicture}
\end{center}

Таким образом, поле оказывает ориентирующее воздействие на молекулу, как бы стремясь развернуть от минуса к плюсу.

\vspace{5px}

--- Рассмотрим \textbf{неполярную молекулу}, поскольку у нее в некотором роде заряды сосредоточены в одном месте,
под воздействием электрического поля она поляризуется и образует диполь с некоторым дипольным моментом.

\begin{center}
    \definecolor{qqqqff}{rgb}{0,0,1}
    \definecolor{ffqqtt}{rgb}{1,0,0.2}
    \begin{tikzpicture}[line cap=round,line join=round,>=triangle 45,x=1.0cm,y=1.0cm]
        \clip(10.77,23.47) rectangle (15.24,25.21);
        \draw [line width=1.2pt,color=ffqqtt] (13,17)-- (13,13);
        \draw [line width=1.2pt,color=qqqqff] (16,17)-- (16,13);
        \draw [->] (13,16) -- (16,16);
        \draw [->] (13,15) -- (16,15);
        \draw [->] (13,14) -- (16,14);
        \draw [color=qqqqff](12.34,17.26) node[anchor=north west] {$-\sigma^{\prime}$};
        \draw [color=ffqqtt](16.08,17.34) node[anchor=north west] {$+\sigma^{\prime}$};
        \draw [->] (13.02,16.6) -- (16.02,16.6);
        \draw [color=qqqqff] (13.2,16.4) circle (0.17cm);
        \draw [color=ffqqtt] (13.6,16.4) circle (0.16cm);
        \draw [color=qqqqff] (14.2,16.4) circle (0.17cm);
        \draw [color=ffqqtt] (14.6,16.4) circle (0.16cm);
        \draw [color=qqqqff] (15.6,16.4) circle (0.17cm);
        \draw (13.2,16.4)-- (13.6,16.4);
        \draw (14.2,16.4)-- (14.6,16.4);
        \draw [color=ffqqtt] (15.2,16.4) circle (0.16cm);
        \draw (15.2,16.4)-- (15.6,16.4);
        \draw [->] (11,24) -- (15,24);
        \draw [color=ffqqtt] (12.86,24.63) circle (0.16cm);
        \draw [color=qqqqff] (12.72,24.49) circle (0.17cm);
        \draw (14.42,24.8) node[anchor=north west] {$\vec E$};
        \draw [->] (12.57,24.56) -- (12.57,24.97);
        \draw [->] (12.99,24.53) -- (12.99,24.19);
    \end{tikzpicture}
\end{center}

\define Диэлектрик состоящий из полярных молекул называется полярным диэлектриком.
\begin{center}
    \definecolor{qqqqff}{rgb}{0,0,1}
    \definecolor{ffqqtt}{rgb}{1,0,0.2}
    \begin{tikzpicture}[line cap=round,line join=round,>=triangle 45,x=1.0cm,y=1.0cm]
        \clip(11.91,12.32) rectangle (17.46,17.56);
        \draw [line width=1.2pt,color=ffqqtt] (13,17)-- (13,13);
        \draw [line width=1.2pt,color=qqqqff] (16,17)-- (16,13);
        \draw [->] (13,16) -- (16,16);
        \draw [->] (13,15) -- (16,15);
        \draw [->] (13,14) -- (16,14);
        \draw [color=qqqqff](12.33,17.22) node[anchor=north west] {$-\sigma^{\prime}$};
        \draw [color=ffqqtt](16.07,17.3) node[anchor=north west] {$+\sigma^{\prime}$};
        \draw [->] (13.02,16.6) -- (16.02,16.6);
        \draw [color=qqqqff] (13.2,16.4) circle (0.17cm);
        \draw [color=ffqqtt] (13.6,16.4) circle (0.16cm);
        \draw [color=qqqqff] (14.2,16.4) circle (0.17cm);
        \draw [color=ffqqtt] (14.6,16.4) circle (0.16cm);
        \draw [color=qqqqff] (15.6,16.4) circle (0.17cm);
        \draw (13.2,16.4)-- (13.6,16.4);
        \draw (14.2,16.4)-- (14.6,16.4);
        \draw [color=ffqqtt] (15.2,16.4) circle (0.16cm);
        \draw (15.2,16.4)-- (15.6,16.4);
    \end{tikzpicture}
\end{center}
На рисунке можно заметить что диполи на концах компенсируют друг друга, но на концах остаются заряды,
из этого делаем вывод что любые диэлектрики \textbf{поляризуются.}

\vspace{5px}

\textit{\textbf{Вывод:} в электрическом поле весь диэлектрик поляризуется, причем это характеризуется вектором поляризации диэлектрика.}

\vspace{5px}

\define Вектор $\vec P$ называется вектором поляризации диэлектрика и вычисляется он таким образом:
\[ \vec P = \frac{\sum_{\Delta V} \vec P_i}{\Delta V}\]
Важно отметить что вектор поляризации \textit{связан с напряженностью электрического поля.}

Как я поняла, физический смысл: поляризованность диэлектрика - дипольный момент, который приобретают полярные молекулы в единице объема диэлектрика.
\[ \vec P  = \frac{\sum_{\Delta V} \vec P_i}{\Delta V} = \kappa \cdot \epsilon_0 \cdot E\]
Следовательно, \textbf{единицы измерения:} $[p] = \frac{\text{Кл}}{\text{м}^3}$

\vspace{5px}

$\kappa > 0$ - диэлектрическая восприимчивость диэлектрика, \textit{не зависит от напряженности электрического поля}, является характеристикой диэлектрика.

\vspace{5px}

Важно что диэлектрическая восприимчивость ведет себя по разному для полярных и неполярных диэклектриков, в полярном диэлектрике $\kappa$ зависит от температуры, в неполярном зависит от коцентрации молекул.

В общем случае, \[ \sigma^\prime = \vec P \cdot \vec n = (\vec P, \vec n) \]


\subsection{Описание поля в диэлектриках}
\begin{center}
    \definecolor{qqqqff}{rgb}{0,0,1}
    \definecolor{ffqqtt}{rgb}{1,0,0.2}
    \begin{tikzpicture}[line cap=round,line join=round,>=triangle 45,x=1.0cm,y=1.0cm]
        \clip(11.71,12.41) rectangle (17.56,17.59);
        \draw [line width=1.2pt,color=ffqqtt] (13,17)-- (13,13);
        \draw [line width=1.2pt,color=qqqqff] (16,17)-- (16,13);
        \draw [->] (13,16) -- (16,16);
        \draw [->] (13,15) -- (16,15);
        \draw [->] (13,14) -- (16,14);
        \draw [color=qqqqff](12.25,17.31) node[anchor=north west] {$-\sigma^{\prime}$};
        \draw [color=ffqqtt](16.08,17.34) node[anchor=north west] {$+\sigma^{\prime}$};
        \draw [->] (13.02,16.6) -- (16.02,16.6);
        \draw [color=qqqqff] (13.2,16.4) circle (0.17cm);
        \draw [color=ffqqtt] (13.6,16.4) circle (0.16cm);
        \draw [color=qqqqff] (14.2,16.4) circle (0.17cm);
        \draw [color=ffqqtt] (14.6,16.4) circle (0.16cm);
        \draw [color=qqqqff] (15.6,16.4) circle (0.17cm);
        \draw (13.2,16.4)-- (13.6,16.4);
        \draw (14.2,16.4)-- (14.6,16.4);
        \draw [color=ffqqtt] (15.2,16.4) circle (0.16cm);
        \draw (15.2,16.4)-- (15.6,16.4);
        \draw [->] (11,24) -- (15,24);
        \draw [color=ffqqtt] (12.86,24.63) circle (0.16cm);
        \draw [color=qqqqff] (12.72,24.49) circle (0.17cm);
        \draw (14.42,24.8) node[anchor=north west] {$\vec E$};
        \draw [->] (12.57,24.56) -- (12.57,24.97);
        \draw [->] (12.99,24.53) -- (12.99,24.19);
        \draw [->] (15.26,15.62) -- (13.81,15.62);
        \draw (15.08,17.51) node[anchor=north west] {$\vec E_0$};
        \draw (14.34,15.62) node[anchor=north west] {$\vec E^{\prime}$};
    \end{tikzpicture}
\end{center}


Помещая диэлектрик в внешнее поле $E_0$, то образуется некоторое внутреннее поле $E^{\prime}$(молекулярные заряды),

тем самым общее поле складывается как сумма $E = E^{\prime} + E_0$, причем сумма векторная:


Найдем поток по теореме Гаусса:

\[\Phi_E = \oiint\limits_S (\vec E_0 + \vec E^{\prime})dS = \oiint\limits_S \vec E_0 \cdot \vec n dS + \oiint\limits_S \vec E^{\prime} \cdot \vec n dS \Rightarrow\]
\[\Phi_E = \frac{1}{\epsilon_0} \sum_{i} q_i + \frac{1}{\epsilon_0} \sum_{i} q_i^{\prime}\]
Для того чтобы не учитывать связные(молякулярные заряды), ввели величину называемую \textbf{\textit{электрическим смещением}}.
Это векторная величина и вычисляется она так:
\[\vec D = \vec E \epsilon_0 + \vec P\]
\begin{center}
    где $\vec P$ - вектор поляризации диэлектрика
\end{center}



Тогда применив теорему Гаусса и получим:
\[\Phi_D = \oiint\limits_S \vec D \cdot \vec n dS = \sum_{i} q_i\]
Поскольку мы можем переписать $\vec P = \kappa \epsilon_0 \vec E$, то подставив это в формулу для $\vec D$ получим:

\[\vec D = \epsilon_0 \vec E + \kappa \cdot \epsilon_0 \vec E = (1+\kappa)\epsilon_0 E\]
\begin{center}
    где $(1+\kappa) = \epsilon > 1 (\epsilon = 1 \text{ - только в вакууме})$ - электрическая проницаемость диэлектрика, тогда:
\end{center}

\[\vec D = \epsilon \cdot \epsilon_0 \cdot \vec E\]

\textit{Замечание:  }Если диэлектрик изотропный (то есть по всем направлениям его электрические свойства одинаковы), то $\vec D$ сонаправлен с $\vec E$, а если он анизотропный то равенство выше не будет верным(например в кристаллах).

\vspace{5px}

Рассмотрим $\vec D$ в вакууме: $\vec D_0 = \epsilon_0 \vec E_0$. Тогда:
\[\epsilon \cdot \epsilon_0 \vec E = \epsilon_0 \vec E_0 \Rightarrow \vec E = \frac{\vec E_0}{\epsilon}\]


\textit{\textbf{Вывод:} Диэлектрическая проницаемость показывает во сколько раз уменьшается электрическое поле внутри диэлектрика,
    то есть диэлектрик можно использовать для ослабления влияния внешних электрических полей.}

\begin{center}
    \definecolor{qqccqq}{rgb}{0,0.8,0}
    \definecolor{qqqqff}{rgb}{0,0,1}
    \definecolor{ffqqtt}{rgb}{1,0,0.2}
    \begin{tikzpicture}[line cap=round,line join=round,>=triangle 45,x=1.0cm,y=1.0cm]
        \clip(7.44,39.36) rectangle (17.81,43.35);
        \draw [line width=1.2pt,color=ffqqtt] (13,17)-- (13,13);
        \draw [line width=1.2pt,color=qqqqff] (16,17)-- (16,13);
        \draw [->] (13,16) -- (16,16);
        \draw [->] (13,15) -- (16,15);
        \draw [->] (13,14) -- (16,14);
        \draw [color=qqqqff](12.25,17.31) node[anchor=north west] {$-\sigma^{\prime}$};
        \draw [color=ffqqtt](16.08,17.34) node[anchor=north west] {$+\sigma^{\prime}$};
        \draw [->] (13.02,16.6) -- (16.02,16.6);
        \draw [color=qqqqff] (13.2,16.4) circle (0.17cm);
        \draw [color=ffqqtt] (13.6,16.4) circle (0.16cm);
        \draw [color=qqqqff] (14.2,16.4) circle (0.17cm);
        \draw [color=ffqqtt] (14.6,16.4) circle (0.16cm);
        \draw [color=qqqqff] (15.6,16.4) circle (0.17cm);
        \draw (13.2,16.4)-- (13.6,16.4);
        \draw (14.2,16.4)-- (14.6,16.4);
        \draw [color=ffqqtt] (15.2,16.4) circle (0.16cm);
        \draw (15.2,16.4)-- (15.6,16.4);
        \draw [->] (11,24) -- (15,24);
        \draw [color=ffqqtt] (12.86,24.63) circle (0.16cm);
        \draw [color=qqqqff] (12.72,24.49) circle (0.17cm);
        \draw (14.42,24.8) node[anchor=north west] {$\vec E$};
        \draw [->] (12.57,24.56) -- (12.57,24.97);
        \draw [->] (12.99,24.53) -- (12.99,24.19);
        \draw [->] (15.26,15.62) -- (13.81,15.62);
        \draw (15.08,17.51) node[anchor=north west] {$\vec E_0$};
        \draw (14.34,15.62) node[anchor=north west] {$\vec E^{\prime}$};
        \draw (11,43)-- (9,40);
        \draw (13,43)-- (11,40);
        \draw [->] (8,42) -- (10.32,41.98);
        \draw [->] (11.58,41.23) -- (11,42);
        \draw [->] (17,41) -- (16.42,41.77);
        \draw [->] (14.68,41.02) -- (17.01,41);
        \draw [->,color=qqccqq] (14.68,41.02) -- (16.42,41.77);
        \draw [->,color=qqccqq] (10.32,41.98) -- (12.06,42.74);
        \draw (9.33,42.72) node[anchor=north west] {$\vec E_0$};
        \draw (11.45,42.15) node[anchor=north west] {$\vec E^{\prime}$};
        \draw (11.05,43.15) node[anchor=north west] {$\vec E$};
        \draw (15.6,41.03) node[anchor=north west] {$\vec E_0$};
        \draw (16.9,41.89) node[anchor=north west] {$\vec E^{\prime}$};
        \draw (15.26,42.03) node[anchor=north west] {$\vec E$};
    \end{tikzpicture}
\end{center}

\textit{\textbf{Вывод:} Граница диэлектрика преломляет линии электрического поля}

\section{Проводники}
\subsection{Проводники во внешнем электрическом поле}
В отличии от диэлектриков, в проводниках есть свободные заряды которые проводят ток.

\begin{center}
    \definecolor{ttttff}{rgb}{0.2,0.2,1}
    \definecolor{fftttt}{rgb}{1,0.2,0.2}
    \begin{tikzpicture}[line cap=round,line join=round,>=triangle 45,x=1.0cm,y=1.0cm]
        \clip(3.5,19.54) rectangle (10.56,24.52);
        \fill[color=ttttff,fill=ttttff,fill opacity=0.1] (4,24) -- (7,24) -- (7,20) -- (4,20) -- cycle;
        \fill[color=fftttt,fill=fftttt,fill opacity=0.1] (7,24) -- (10,24) -- (10,20) -- (7,20) -- cycle;
        \draw [->,dash pattern=on 2pt off 2pt] (4,23) -- (10,23);
        \draw [->,dash pattern=on 2pt off 2pt] (4,24) -- (10,24);
        \draw [->,dash pattern=on 2pt off 2pt] (4,22) -- (10,22);
        \draw [->,dash pattern=on 2pt off 2pt] (4,20) -- (10,20);
        \draw [->,dash pattern=on 2pt off 2pt] (4,21) -- (10,21);
        \draw [rotate around={-1.69:(6.94,22.27)}] (6.94,22.27) ellipse (1.87cm and 0.55cm);
        \draw [color=fftttt](7.85,22.61) node[anchor=north west] {$+$};
        \draw [color=ttttff](5.54,22.63) node[anchor=north west] {$-$};
        \draw [color=fftttt](7.6,22.84) node[anchor=north west] {$+$};
        \draw [color=ttttff](5.37,22.9) node[anchor=north west] {$-$};
        \draw (7,24)-- (7,20);
        \draw [color=ttttff] (4,24)-- (7,24);
        \draw [color=ttttff] (7,24)-- (7,20);
        \draw [color=ttttff] (7,20)-- (4,20);
        \draw [color=ttttff] (4,20)-- (4,24);
        \draw [color=fftttt] (7,24)-- (10,24);
        \draw [color=fftttt] (10,24)-- (10,20);
        \draw [color=fftttt] (10,20)-- (7,20);
        \draw [color=fftttt] (7,20)-- (7,24);
        \draw [color=ttttff](6.35,20.69) node[anchor=north west] {$-$};
        \draw [color=fftttt](7.22,20.71) node[anchor=north west] {$+$};
    \end{tikzpicture}
\end{center}

При разделении проводника одна половина окажется положительно заряженной(+), другая отрицательно(-). При этом линии
поле будут выглядет так:

\begin{center}
    \definecolor{qqccqq}{rgb}{0,0.8,0}
    \definecolor{ttttff}{rgb}{0.2,0.2,1}
    \definecolor{fftttt}{rgb}{1,0.2,0.2}
    \begin{tikzpicture}[line cap=round,line join=round,>=triangle 45,x=1.0cm,y=1.0cm]
        \clip(3.12,19.52) rectangle (11.72,23.69);
        \draw [->] (4,23) -- (11,23);
        \draw [->] (4,22) -- (11,22);
        \draw [->] (4,21) -- (11,21);
        \draw [->] (4,20) -- (11,20);
        \draw [rotate around={-2.2:(7.29,21.83)}] (7.29,21.83) ellipse (1.6cm and 0.66cm);
        \draw [color=fftttt](8.26,21.94) node[anchor=north west] {$+$};
        \draw [color=ttttff](5.99,22.32) node[anchor=north west] {$-$};
        \draw [color=fftttt](8.47,22.32) node[anchor=north west] {$+$};
        \draw [color=fftttt](8.04,22.36) node[anchor=north west] {$+$};
        \draw [color=ttttff](6.27,22.61) node[anchor=north west] {$-$};
        \draw [color=ttttff](6.41,22.15) node[anchor=north west] {$-$};
        \draw [shift={(7.27,39.65)},color=qqccqq]  plot[domain=4.58:4.87,variable=\t]({1*17.04*cos(\t r)+0*17.04*sin(\t r)},{0*17.04*cos(\t r)+1*17.04*sin(\t r)});
        \draw [shift={(10.18,18.92)},color=qqccqq]  plot[domain=1.34:2.01,variable=\t]({1*3.53*cos(\t r)+0*3.53*sin(\t r)},{0*3.53*cos(\t r)+1*3.53*sin(\t r)});
        \draw [shift={(9.93,24.36)},color=qqccqq]  plot[domain=4.31:5.08,variable=\t]({1*3.13*cos(\t r)+0*3.13*sin(\t r)},{0*3.13*cos(\t r)+1*3.13*sin(\t r)});
        \draw [shift={(7.89,6.58)},color=qqccqq]  plot[domain=1.42:1.78,variable=\t]({1*13.94*cos(\t r)+0*13.94*sin(\t r)},{0*13.94*cos(\t r)+1*13.94*sin(\t r)});
        \draw [shift={(4.93,19.63)},color=qqccqq]  plot[domain=1.18:1.88,variable=\t]({1*2.85*cos(\t r)+0*2.85*sin(\t r)},{0*2.85*cos(\t r)+1*2.85*sin(\t r)});
        \draw [shift={(5.09,23.68)},color=qqccqq]  plot[domain=4.23:5.12,variable=\t]({1*2.42*cos(\t r)+0*2.42*sin(\t r)},{0*2.42*cos(\t r)+1*2.42*sin(\t r)});
        \draw [color=qqccqq] (4.99,22.77)-- (4.02,22.75);
        \draw [color=qqccqq] (4.95,20.21)-- (4,20.21);
        \draw [->,color=qqccqq] (9.95,22.82) -- (11.01,22.82);
        \draw [->,color=qqccqq] (10.01,20.36) -- (11.05,20.38);
    \end{tikzpicture}
\end{center}

Получаем вывод, что \textit{проводник либо искривляет линии поля либо прерывает их}.

\begin{center}
    \definecolor{qqccqq}{rgb}{0,0.8,0}
    \definecolor{qqffqq}{rgb}{0,1,0}
    \definecolor{ttttff}{rgb}{0.2,0.2,1}
    \definecolor{fftttt}{rgb}{1,0.2,0.2}
    \begin{tikzpicture}[line cap=round,line join=round,>=triangle 45,x=1.0cm,y=1.0cm]
        \clip(3.25,17.52) rectangle (9.49,23.31);
        \draw (4,23)-- (4,18);
        \draw [shift={(6.92,20.5)}] plot[domain=-1.51:1.51,variable=\t]({1*1.5*cos(\t r)+0*1.5*sin(\t r)},{0*1.5*cos(\t r)+1*1.5*sin(\t r)});
        \draw (7,22)-- (4.74,20.55);
        \draw (7,19)-- (4.74,20.55);
        \draw [color=fftttt](7.29,21.69) node[anchor=north west] {$+$};
        \draw [color=fftttt](7.58,21) node[anchor=north west] {$+$};
        \draw [color=fftttt](7.56,20.46) node[anchor=north west] {$+$};
        \draw [color=fftttt](7.41,20.04) node[anchor=north west] {$+$};
        \draw [color=fftttt](7.72,21.48) node[anchor=north west] {$+$};
        \draw [color=ttttff](5.16,20.94) node[anchor=north west] {$-$};
        \draw [color=ttttff](5.62,20.69) node[anchor=north west] {$-$};
        \draw [color=ttttff](5.54,21.27) node[anchor=north west] {$-$};
        \draw [shift={(4.28,20)},color=qqffqq]  plot[domain=0.67:1.69,variable=\t]({1*2.42*cos(\t r)+0*2.42*sin(\t r)},{0*2.42*cos(\t r)+1*2.42*sin(\t r)});
        \draw [shift={(4,18.79)},color=qqffqq]  plot[domain=0.95:1.57,variable=\t]({1*2.9*cos(\t r)+0*2.9*sin(\t r)},{0*2.9*cos(\t r)+1*2.9*sin(\t r)});
        \draw [shift={(4.47,21.39)},color=qqffqq]  plot[domain=4.44:5.4,variable=\t]({1*2.04*cos(\t r)+0*2.04*sin(\t r)},{0*2.04*cos(\t r)+1*2.04*sin(\t r)});
        \draw [shift={(4.92,20.88)},color=qqffqq]  plot[domain=4.29:5.52,variable=\t]({1*2.23*cos(\t r)+0*2.23*sin(\t r)},{0*2.23*cos(\t r)+1*2.23*sin(\t r)});
        \draw [color=qqccqq] (4.74,20.55)-- (4,20.57);
    \end{tikzpicture}
\end{center}

Рассмотрим рисунок, на острие будет очень высокая напряженность поля,из-за которой может ионизироваться воздуха в результате отрыва
электронов и их перетекания, приводит к его свечению.

\subsection{Равновесие заряда на проводнике}

Из предыдущих формул, известна зависимость напряжение электрического поля и потенциала соотвественно.

Рассмотрим поверхность проводника, поскольку движение частиц прекращается напряженность такого электрического поля будет равна нулю:
\[\vec E = - grad(\phi) = 0 \Rightarrow \phi - const \]
Так получается, что поверхность проводника есть эквипотенциальная поверхность. Тогда линии напряженности $\vec E$ вблизи
повехности проводника будут перпендикулярны поверхности проводника, из свойств эквипотенциальных поверхностей.

Рассмотрим проводник вида:
\begin{center}
    \definecolor{ttttff}{rgb}{0.2,0.2,1}
    \definecolor{fftttt}{rgb}{1,0.2,0.2}
    \definecolor{qqccqq}{rgb}{0,0.8,0}
    \begin{tikzpicture}[line cap=round,line join=round,>=triangle 45,x=1.0cm,y=1.0cm]
        \draw (4,23)-- (4,18);
        \draw [shift={(6.92,20.5)},color=qqccqq]  plot[domain=-1.51:1.51,variable=\t]({1*1.5*cos(\t r)+0*1.5*sin(\t r)},{0*1.5*cos(\t r)+1*1.5*sin(\t r)});
        \draw (7,22)-- (4.74,20.55);
        \draw (7,19)-- (4.74,20.55);
        \draw [color=fftttt](7.29,21.69) node[anchor=north west] {$+$};
        \draw [color=fftttt](7.58,21) node[anchor=north west] {$+$};
        \draw [color=fftttt](7.56,20.46) node[anchor=north west] {$+$};
        \draw [color=fftttt](7.41,20.04) node[anchor=north west] {$+$};
        \draw [color=fftttt](7.72,21.48) node[anchor=north west] {$+$};
        \draw [color=ttttff](5.16,20.94) node[anchor=north west] {$-$};
        \draw [color=ttttff](5.62,20.69) node[anchor=north west] {$-$};
        \draw [color=ttttff](5.54,21.27) node[anchor=north west] {$-$};
        \draw [shift={(4.28,20)},color=qqccqq]  plot[domain=0.67:1.69,variable=\t]({1*2.42*cos(\t r)+0*2.42*sin(\t r)},{0*2.42*cos(\t r)+1*2.42*sin(\t r)});
        \draw [shift={(4,18.79)},color=qqccqq]  plot[domain=0.95:1.57,variable=\t]({1*2.9*cos(\t r)+0*2.9*sin(\t r)},{0*2.9*cos(\t r)+1*2.9*sin(\t r)});
        \draw [shift={(4.47,21.39)},color=qqccqq]  plot[domain=4.44:5.4,variable=\t]({1*2.04*cos(\t r)+0*2.04*sin(\t r)},{0*2.04*cos(\t r)+1*2.04*sin(\t r)});
        \draw [shift={(4.92,20.88)},color=qqccqq]  plot[domain=4.29:5.52,variable=\t]({1*2.23*cos(\t r)+0*2.23*sin(\t r)},{0*2.23*cos(\t r)+1*2.23*sin(\t r)});
        \draw [color=qqccqq] (4.74,20.55)-- (4,20.57);
        \draw [color=qqccqq] (6.04,38.25) circle (1.29cm);
        \draw [shift={(11.95,48.86)},color=qqccqq]  plot[domain=4.23:4.62,variable=\t]({1*10.9*cos(\t r)+0*10.9*sin(\t r)},{0*10.9*cos(\t r)+1*10.9*sin(\t r)});
        \draw [shift={(11.11,26.24)},color=qqccqq]  plot[domain=1.58:1.96,variable=\t]({1*11.76*cos(\t r)+0*11.76*sin(\t r)},{0*11.76*cos(\t r)+1*11.76*sin(\t r)});
        \draw [shift={(-1.11,50.71)},color=qqccqq]  plot[domain=4.95:5.24,variable=\t]({1*13.08*cos(\t r)+0*13.08*sin(\t r)},{0*13.08*cos(\t r)+1*13.08*sin(\t r)});
        \draw [shift={(0.42,24.36)},color=qqccqq]  plot[domain=1.21:1.46,variable=\t]({1*13.73*cos(\t r)+0*13.73*sin(\t r)},{0*13.73*cos(\t r)+1*13.73*sin(\t r)});
        \draw (3.93,37.08) node[anchor=north west] {$\textit{да прибудет с нами понимание}$};
    \end{tikzpicture}
\end{center}
Важно отметить, что заряды на проводниках будут распределены неравномерно, так на рисунке ниже они будут скапливаться на остриях.

\section{Электроёмкость}
Экспериментально доказана следующая линейная зависимость:
\[q = C \phi\]
Здесь коэфициент C называется электроемкостью проводника.

Для сферы:
\[ C = 4 \pi \epsilon_0 \epsilon R\]
\begin{center}
    где $\epsilon$ есть диэлектрическая проницаемость
\end{center}
В системе СИ соотвественно: $[C = \frac{q}{\phi}] = \frac{\text{Кл}}{\text{В}} - Фарад$
\subsection{Конденсаторы}
Уединенные проводники, то есть те рядом с которыми нет других проводников, обладают малой емкостью.

Рассмотрим два проводника.Работа по перемещению от положительно заряженного проводника к отрицательно заряденному уменьшается, поскольку
такое перемещение совершается на r а не на $\infty$ $\Rightarrow$ уменьшается потенциал при сохранении заряда $\Rightarrow$
растет электроёмкость.
Устройство состоящее из таких проводников называется конденсатором.


\subsubsection{Плоский кондесатор}

\begin{center}
    \definecolor{ttttff}{rgb}{0.2,0.2,1}
    \definecolor{fftttt}{rgb}{1,0.2,0.2}
    \definecolor{qqccqq}{rgb}{0,0.8,0}
    \begin{tikzpicture}[line cap=round,line join=round,>=triangle 45,x=1.0cm,y=1.0cm]
        \clip(2.68,34.42) rectangle (9.54,41.25);
        \draw (4,23)-- (4,18);
        \draw [shift={(6.92,20.5)},color=qqccqq]  plot[domain=-1.51:1.51,variable=\t]({1*1.5*cos(\t r)+0*1.5*sin(\t r)},{0*1.5*cos(\t r)+1*1.5*sin(\t r)});
        \draw (7,22)-- (4.74,20.55);
        \draw (7,19)-- (4.74,20.55);
        \draw [color=fftttt](7.29,21.69) node[anchor=north west] {$+$};
        \draw [color=fftttt](7.58,21) node[anchor=north west] {$+$};
        \draw [color=fftttt](7.56,20.46) node[anchor=north west] {$+$};
        \draw [color=fftttt](7.41,20.04) node[anchor=north west] {$+$};
        \draw [color=fftttt](7.72,21.48) node[anchor=north west] {$+$};
        \draw [color=ttttff](5.16,20.94) node[anchor=north west] {$-$};
        \draw [color=ttttff](5.62,20.69) node[anchor=north west] {$-$};
        \draw [color=ttttff](5.54,21.27) node[anchor=north west] {$-$};
        \draw [shift={(4.28,20)},color=qqccqq]  plot[domain=0.67:1.69,variable=\t]({1*2.42*cos(\t r)+0*2.42*sin(\t r)},{0*2.42*cos(\t r)+1*2.42*sin(\t r)});
        \draw [shift={(4,18.79)},color=qqccqq]  plot[domain=0.95:1.57,variable=\t]({1*2.9*cos(\t r)+0*2.9*sin(\t r)},{0*2.9*cos(\t r)+1*2.9*sin(\t r)});
        \draw [shift={(4.47,21.39)},color=qqccqq]  plot[domain=4.44:5.4,variable=\t]({1*2.04*cos(\t r)+0*2.04*sin(\t r)},{0*2.04*cos(\t r)+1*2.04*sin(\t r)});
        \draw [shift={(4.92,20.88)},color=qqccqq]  plot[domain=4.29:5.52,variable=\t]({1*2.23*cos(\t r)+0*2.23*sin(\t r)},{0*2.23*cos(\t r)+1*2.23*sin(\t r)});
        \draw [color=qqccqq] (4.74,20.55)-- (4,20.57);
        \draw [color=fftttt] (4,41)-- (4,35);
        \draw [color=ttttff] (8,41)-- (8,35);
        \draw (3,38)-- (4,38);
        \draw (8,38)-- (9,38);
        \draw [->] (4,40) -- (7.74,40.02);
        \draw [->] (4,38.73) -- (7.74,38.75);
        \draw [->] (4,37.31) -- (7.64,37.37);
        \draw [->] (4,36.19) -- (7.7,36.25);
        \draw (5.85,41.18) node[anchor=north west] {$\vec E$};
        \draw [color=fftttt](3.54,41.16) node[anchor=north west] {$+$};
        \draw [color=ttttff](8.06,41.18) node[anchor=north west] {$-$};
        \draw (4.06,36.04) node[anchor=north west] {$\phi_1$};
        \draw (7.54,36.02) node[anchor=north west] {$\phi_2$};
    \end{tikzpicture}
\end{center}

\[q = C(\phi_1 - \phi_2) = CU\]

Далее: $Ed = U$, и в силу того что $E = \frac{\sigma}{\epsilon \epsilon_0}$

\[q = CEd = C\frac{\sigma}{\epsilon \epsilon_0} = C\frac{qd}{s\epsilon \epsilon_0} \Rightarrow 1 = C \frac{d}{s \epsilon \epsilon_0}\]

\[C = \frac{s\epsilon_0\epsilon}{d}\]

\begin{center}
    , где S- площадь пластины, d - расстояние между пластинами
    \newline
    \textbf{\textit{- электроемкость}}
\end{center}

Оптимальный способ увеличения электроемкости без увеличения размер пластины --- подбор других диэлектриков. то есть
изменение значения $\epsilon$.

\define Диэлектрики удовлетворяющие свойствам:
\begin{enumerate}
    \item Высокая диэлектрическая проницаемость
    \item При исчезновении электрического поля у них остается некоторая остаточная поляризация.
\end{enumerate}
называются \textit{\textbf{сегнетоэлектриками}}.

\subsubsection{Цилиндрический кондесатор}

Вычислим заряд всего цилиндра с помощью площади боковой поверхности:
\[2 \pi R \cdot l \cdot \sigma = q \Rightarrow \sigma = \frac{q}{2 \pi R l}\]

\[\phi_1 - \phi_2 = \int_{r_1}^{r_2} \vec E d\vec r = \int_{r_1}^{r_2} \frac{\sigma R}{\epsilon_0 \epsilon} dr = \frac{\sigma R}{\epsilon_0 \epsilon}(ln r_2 - ln r_1)
    = \frac{\sigma R}{\epsilon_0 \epsilon} \cdot ln{|\frac{r_2}{r_1}|} = \frac{q \cdot R}{2 \pi R l \epsilon_0 \epsilon} ln{\frac{r_2}{r_1}}\]
Отсюда, в силу $q = C(\phi_1 - \phi_2) = CU \Rightarrow$:
\[C = \frac{\epsilon \epsilon_0 2 \pi l}{ln{\frac{r_2}{r_1}}}\]

Если $d = r_2 - r_1 << r_2$, то:
\[ln(1+ \frac{r_2 - r_1}{r_1}) \sim \frac{d}{r_1}\]
Тогда заменяя:
\[C = \frac{\epsilon \epsilon_0 2 \pi l r_1}{d} \approx \frac{\epsilon \epsilon_0 s}{d}\]
Равенство выше верно для разности размеров много меньше, чем радиус большего цилиндра.

\begin{center}
    \definecolor{ttttff}{rgb}{0.2,0.2,1}
    \definecolor{fftttt}{rgb}{1,0.2,0.2}
    \definecolor{qqccqq}{rgb}{0,0.8,0}
    \begin{tikzpicture}[line cap=round,line join=round,>=triangle 45,x=1.0cm,y=1.0cm]
        \clip(2.91,36.73) rectangle (10.28,42.08);
        \draw (4,23)-- (4,18);
        \draw [shift={(6.92,20.5)},color=qqccqq]  plot[domain=-1.51:1.51,variable=\t]({1*1.5*cos(\t r)+0*1.5*sin(\t r)},{0*1.5*cos(\t r)+1*1.5*sin(\t r)});
        \draw (7,22)-- (4.74,20.55);
        \draw (7,19)-- (4.74,20.55);
        \draw [color=fftttt](7.29,21.69) node[anchor=north west] {$+$};
        \draw [color=fftttt](7.58,21) node[anchor=north west] {$+$};
        \draw [color=fftttt](7.56,20.46) node[anchor=north west] {$+$};
        \draw [color=fftttt](7.41,20.04) node[anchor=north west] {$+$};
        \draw [color=fftttt](7.72,21.48) node[anchor=north west] {$+$};
        \draw [color=ttttff](5.16,20.94) node[anchor=north west] {$-$};
        \draw [color=ttttff](5.62,20.69) node[anchor=north west] {$-$};
        \draw [color=ttttff](5.54,21.27) node[anchor=north west] {$-$};
        \draw [shift={(4.28,20)},color=qqccqq]  plot[domain=0.67:1.69,variable=\t]({1*2.42*cos(\t r)+0*2.42*sin(\t r)},{0*2.42*cos(\t r)+1*2.42*sin(\t r)});
        \draw [shift={(4,18.79)},color=qqccqq]  plot[domain=0.95:1.57,variable=\t]({1*2.9*cos(\t r)+0*2.9*sin(\t r)},{0*2.9*cos(\t r)+1*2.9*sin(\t r)});
        \draw [shift={(4.47,21.39)},color=qqccqq]  plot[domain=4.44:5.4,variable=\t]({1*2.04*cos(\t r)+0*2.04*sin(\t r)},{0*2.04*cos(\t r)+1*2.04*sin(\t r)});
        \draw [shift={(4.92,20.88)},color=qqccqq]  plot[domain=4.29:5.52,variable=\t]({1*2.23*cos(\t r)+0*2.23*sin(\t r)},{0*2.23*cos(\t r)+1*2.23*sin(\t r)});
        \draw [color=qqccqq] (4.74,20.55)-- (4,20.57);
        \draw(5.65,39.21) circle (1.63cm);
        \draw [color=qqccqq] (5.6,39.21) circle (0.87cm);
        \draw (5.08,40.74)-- (7.76,41.47);
        \draw (7.27,39.41)-- (9,40);
        \draw [shift={(7.54,40.02)}] plot[domain=-0.02:1.42,variable=\t]({1*1.46*cos(\t r)+0*1.46*sin(\t r)},{0*1.46*cos(\t r)+1*1.46*sin(\t r)});
        \draw (6.12,37.64)-- (8.7,38.68);
        \draw [shift={(6.3,39.92)}] plot[domain=-0.48:0.03,variable=\t]({1*2.7*cos(\t r)+0*2.7*sin(\t r)},{0*2.7*cos(\t r)+1*2.7*sin(\t r)});
        \draw [->,color=qqccqq] (5.6,39.23) -- (5.79,40.06);
        \draw [->] (5.6,39.23) -- (7,38.28);
        \draw (5.27,40.02) node[anchor=north west] {$r_1$};
        \draw (6.58,39.12) node[anchor=north west] {$r_2$};
        \draw (7.85,40.31) node[anchor=north west] {$l$};
    \end{tikzpicture}
\end{center}

\subsubsection{Сферический конденсатор}
Рассмотрим сферический конденсатор:
\[C = \frac{4 \pi \epsilon \epsilon_0 R_1 R_2}{(R_1 - R_2)}\]
\begin{center}
    где $R_1 \approx R_2$
\end{center}
Если $d = R_1 -R_2 << R_2$, то можно использовать:
\[C = \frac{\epsilon \epsilon_0 s}{d}\]

\subsubsection{Соединение конденсаторов}

Предельное значение напряжения конденсатора $U_{max}$, если превысить это значение может случится так называемый пробой конденсаторов.
Таким образом, получаем что у конденсатора можно выделить две характеристики: $U_{max}$ и $C$. Часто требуется использовать несколько
конденсаторов объединяя их в конденсаторные батареи.

Соединения могут быть двух типов:
\begin{enumerate}
    \item Параллельные:
          \begin{center}
              \definecolor{ttttff}{rgb}{0.2,0.2,1}
              \definecolor{fftttt}{rgb}{1,0.2,0.2}
              \definecolor{qqccqq}{rgb}{0,0.8,0}
              \begin{tikzpicture}[line cap=round,line join=round,>=triangle 45,x=1.0cm,y=1.0cm]
                  \clip(2.27,35.89) rectangle (7.89,41.18);
                  \draw (4,23)-- (4,18);
                  \draw [shift={(6.92,20.5)},color=qqccqq]  plot[domain=-1.51:1.51,variable=\t]({1*1.5*cos(\t r)+0*1.5*sin(\t r)},{0*1.5*cos(\t r)+1*1.5*sin(\t r)});
                  \draw (7,22)-- (4.74,20.55);
                  \draw (7,19)-- (4.74,20.55);
                  \draw [color=fftttt](7.29,21.69) node[anchor=north west] {$+$};
                  \draw [color=fftttt](7.58,21) node[anchor=north west] {$+$};
                  \draw [color=fftttt](7.56,20.46) node[anchor=north west] {$+$};
                  \draw [color=fftttt](7.41,20.04) node[anchor=north west] {$+$};
                  \draw [color=fftttt](7.72,21.48) node[anchor=north west] {$+$};
                  \draw [color=ttttff](5.16,20.94) node[anchor=north west] {$-$};
                  \draw [color=ttttff](5.62,20.69) node[anchor=north west] {$-$};
                  \draw [color=ttttff](5.54,21.27) node[anchor=north west] {$-$};
                  \draw [shift={(4.28,20)},color=qqccqq]  plot[domain=0.67:1.69,variable=\t]({1*2.42*cos(\t r)+0*2.42*sin(\t r)},{0*2.42*cos(\t r)+1*2.42*sin(\t r)});
                  \draw [shift={(4,18.79)},color=qqccqq]  plot[domain=0.95:1.57,variable=\t]({1*2.9*cos(\t r)+0*2.9*sin(\t r)},{0*2.9*cos(\t r)+1*2.9*sin(\t r)});
                  \draw [shift={(4.47,21.39)},color=qqccqq]  plot[domain=4.44:5.4,variable=\t]({1*2.04*cos(\t r)+0*2.04*sin(\t r)},{0*2.04*cos(\t r)+1*2.04*sin(\t r)});
                  \draw [shift={(4.92,20.88)},color=qqccqq]  plot[domain=4.29:5.52,variable=\t]({1*2.23*cos(\t r)+0*2.23*sin(\t r)},{0*2.23*cos(\t r)+1*2.23*sin(\t r)});
                  \draw [color=qqccqq] (4.74,20.55)-- (4,20.57);
                  \draw (3,40)-- (7,40);
                  \draw (7,39)-- (7,40);
                  \draw (3,39)-- (3,40);
                  \draw (2.5,39)-- (3.54,39);
                  \draw (6.6,38.98)-- (7.52,38.98);
                  \draw (5,40)-- (5,39);
                  \draw (4.56,38.98)-- (5.58,38.98);
                  \draw (2.5,38)-- (3.54,38);
                  \draw (3,38)-- (3,37);
                  \draw (3,37)-- (7,37);
                  \draw (7,38)-- (7,37);
                  \draw (6.62,37.98)-- (7.56,37.98);
                  \draw (4.56,37.98)-- (5.52,37.98);
                  \draw (5,37.98)-- (5,37);
                  \draw (5,40)-- (5,41);
                  \draw (5,37)-- (4.97,35.87);
                  \draw [color=fftttt](5.2,40.79) node[anchor=north west] {$+$};
                  \draw [color=ttttff](5.12,36.96) node[anchor=north west] {$-$};
                  \draw (2.79,38.89) node[anchor=north west] {$C_1$};
                  \draw (4.85,38.95) node[anchor=north west] {$C_2$};
                  \draw (5.81,38.96) node[anchor=north west] {$\dots$};
                  \draw (6.91,38.96) node[anchor=north west] {$C_n$};
                  \draw (2.52,39.83) node[anchor=north west] {$\phi_1$};
                  \draw (4.66,39.77) node[anchor=north west] {$\phi_1$};
                  \draw (6.56,39.71) node[anchor=north west] {$\phi_1$};
                  \draw (2.54,38.1) node[anchor=north west] {$\phi_2$};
                  \draw (4.56,38.04) node[anchor=north west] {$\phi_2$};
                  \draw (6.6,38.04) node[anchor=north west] {$\phi_2$};
              \end{tikzpicture}
          \end{center}
          \[U_1 = U_2 = \ldots = U_n = U\]
          \[q = \sum_{i = 1}^{n} q_i = \sum_{i = 1}^{n}(C_i \cdot U) = U \sum_{i = 1}^{n} C_i\]
          Для параллельного соединения:
          \[C = \sum_{i = 1}^{n} C_i\]
          При этом
          \[U_{max} = min_{(i = 1,n)} U_{i_{max}}\]
    \item Последовательные
          \begin{center}
              \definecolor{ttttff}{rgb}{0.2,0.2,1}
              \definecolor{fftttt}{rgb}{1,0.2,0.2}
              \definecolor{qqccqq}{rgb}{0,0.8,0}
              \begin{tikzpicture}[line cap=round,line join=round,>=triangle 45,x=1.0cm,y=1.0cm]
                  \clip(2.61,54.88) rectangle (5.46,62.12);
                  \draw (4,23)-- (4,18);
                  \draw [shift={(6.92,20.5)},color=qqccqq]  plot[domain=-1.51:1.51,variable=\t]({1*1.5*cos(\t r)+0*1.5*sin(\t r)},{0*1.5*cos(\t r)+1*1.5*sin(\t r)});
                  \draw (7,22)-- (4.74,20.55);
                  \draw (7,19)-- (4.74,20.55);
                  \draw [color=fftttt](7.3,21.67) node[anchor=north west] {$+$};
                  \draw [color=fftttt](7.59,20.98) node[anchor=north west] {$+$};
                  \draw [color=fftttt](7.56,20.44) node[anchor=north west] {$+$};
                  \draw [color=fftttt](7.42,20.01) node[anchor=north west] {$+$};
                  \draw [color=fftttt](7.71,21.45) node[anchor=north west] {$+$};
                  \draw [color=ttttff](5.16,20.91) node[anchor=north west] {$-$};
                  \draw [color=ttttff](5.63,20.67) node[anchor=north west] {$-$};
                  \draw [color=ttttff](5.54,21.24) node[anchor=north west] {$-$};
                  \draw [shift={(4.28,20)},color=qqccqq]  plot[domain=0.67:1.69,variable=\t]({1*2.42*cos(\t r)+0*2.42*sin(\t r)},{0*2.42*cos(\t r)+1*2.42*sin(\t r)});
                  \draw [shift={(4,18.79)},color=qqccqq]  plot[domain=0.95:1.57,variable=\t]({1*2.9*cos(\t r)+0*2.9*sin(\t r)},{0*2.9*cos(\t r)+1*2.9*sin(\t r)});
                  \draw [shift={(4.47,21.39)},color=qqccqq]  plot[domain=4.44:5.4,variable=\t]({1*2.04*cos(\t r)+0*2.04*sin(\t r)},{0*2.04*cos(\t r)+1*2.04*sin(\t r)});
                  \draw [shift={(4.92,20.88)},color=qqccqq]  plot[domain=4.29:5.52,variable=\t]({1*2.23*cos(\t r)+0*2.23*sin(\t r)},{0*2.23*cos(\t r)+1*2.23*sin(\t r)});
                  \draw [color=qqccqq] (4.74,20.55)-- (4,20.57);
                  \draw (3,40)-- (7,40);
                  \draw (7,39)-- (7,40);
                  \draw (3,39)-- (3,40);
                  \draw (2.5,39)-- (3.54,39);
                  \draw (6.6,38.98)-- (7.52,38.98);
                  \draw (5,40)-- (5,39);
                  \draw (4.56,38.98)-- (5.58,38.98);
                  \draw (2.5,38)-- (3.54,38);
                  \draw (3,38)-- (3,37);
                  \draw (3,37)-- (7,37);
                  \draw (7,38)-- (7,37);
                  \draw (6.62,37.98)-- (7.56,37.98);
                  \draw (4.56,37.98)-- (5.52,37.98);
                  \draw (5,37.98)-- (5,37);
                  \draw (5,40)-- (5,41);
                  \draw (5,37)-- (4.97,35.87);
                  \draw (2.8,38.86) node[anchor=north west] {$C_1$};
                  \draw (4.85,38.93) node[anchor=north west] {$C_2$};
                  \draw (5.82,38.95) node[anchor=north west] {$\dots$};
                  \draw (6.91,38.95) node[anchor=north west] {$C_n$};
                  \draw (2.52,39.82) node[anchor=north west] {$\phi_1$};
                  \draw (4.66,39.75) node[anchor=north west] {$\phi_1$};
                  \draw (6.57,39.69) node[anchor=north west] {$\phi_1$};
                  \draw (2.54,38.08) node[anchor=north west] {$\phi_2$};
                  \draw (4.57,38.01) node[anchor=north west] {$\phi_2$};
                  \draw (6.6,38.01) node[anchor=north west] {$\phi_2$};
                  \draw (4,62)-- (4,61);
                  \draw [color=fftttt] (3,61)-- (5,61);
                  \draw [color=ttttff] (3,60)-- (5,60);
                  \draw (4,60)-- (4,59);
                  \draw [color=fftttt] (3,59)-- (5,59);
                  \draw [color=ttttff] (3,58)-- (5,58);
                  \draw (3.99,58)-- (4,57);
                  \draw [color=fftttt] (3,57)-- (5,57);
                  \draw [color=ttttff] (3,56)-- (5,56);
                  \draw (4,56)-- (4,55);
                  \draw [color=fftttt](2.92,61.69) node[anchor=north west] {$+$};
                  \draw [color=fftttt](3.16,59.54) node[anchor=north west] {$+$};
                  \draw [color=fftttt](3.2,57.56) node[anchor=north west] {$+$};
                  \draw [color=ttttff](3.03,60.56) node[anchor=north west] {$-$};
                  \draw [color=ttttff](2.97,58.65) node[anchor=north west] {$-$};
                  \draw [color=ttttff](3.03,56.6) node[anchor=north west] {$-$};
                  \draw (4.81,60.99) node[anchor=north west] {$C_1$};
                  \draw (4.88,59.97) node[anchor=north west] {$C_2$};
                  \draw (4.9,57.99) node[anchor=north west] {$C_3$};
                  \draw (4.9,55.89) node[anchor=north west] {$C_n$};
                  \draw (4.62,56.88) node[anchor=north west] {$\ldots$};
              \end{tikzpicture}
          \end{center}
          \[q_1 =q_2 = \ldots = q_n = q\]
          \[U = \frac{q}{C}\]
          \[U = \sum_{i = 1}^{n} U_i = \sum_{i=1}^{n} \frac{q}{C_i} \Rightarrow \frac{1}{C} = \sum_{i=1}^{n}\frac{1}{C_i}\]
          \begin{center}
              если все конденсаторы одинаковые
          \end{center}
          \[U_{max} = n \cdot U_{i_{max}}\]
          \begin{center}
              если ёмкость конденсаторов одинаковая
          \end{center}
\end{enumerate}

\section{Энергия электрического поля}
\subsection{Энергия системы зарядов}
Возьмем два заряда, $q_1$ $q_2$,

Обозначим потенциал создаваемый первым зарядом в точке $q_2$ назовем $\phi_2$, и наоборот потенциал вторым зарядом в точке $q_1$ назовем $\phi_1$
Энергия взаимодействия зарядов:
\[W_{12} = \frac{1}{4 \pi \epsilon_0} \frac{q_1 \cdot q_2}{r_{12}} = (\frac{1}{4 \pi \epsilon_0} \frac{q_2}{r_{12}}) \cdot q_1\]
\[W_{12} = \phi_2 q_2 = \phi_1 q_1 = \frac{1}{2} (\phi_1 q_1 + \phi_2 q_2)\]

К этой системе добавим заряд $q_3$ и добавим обозначение $\phi_3$ - потенциал который создают заряды $q_1$ $q_2$ в точке $q_3$,
тогда выразим энергию заряда $q_3$ в этом поле:
\[W_3 = \phi_3 q_3 = \frac{1}{4 \pi \epsilon_0} \cdot \frac{q_1 \cdot q_3}{r_{13}} + \frac{1}{4 \pi \epsilon_0} \cdot \frac{q_2 \cdot q_3}{r_{23}}\]

Тогда можем получить полную формулу энергии трех зарядов в этой системе:
\[W = W_{12} + W_{13} + W_{23} = \frac{1}{4 \pi \epsilon_0} (\frac{q_1 \cdot q_2}{r_{12}} + \frac{q_1 \cdot q_3}{r_{13}}+ \frac{q_2 \cdot q_3}{r_{23}}) = \]
\[ \frac{1}{2}\frac{1}{4 \pi \epsilon_0}(\frac{q_1 \cdot q_2}{r_{12}} + \frac{q_1 \cdot q_3}{r_{13}}+ \frac{q_2 \cdot q_3}{r_{23}} + \frac{q_2 \cdot q_1}{r_{21}} + \frac{q_3 \cdot q_1}{r_{31}}+ \frac{q_3 \cdot q_2}{r_{32}}) = \]
\[ \frac{1}{2}(\frac{q_1}{4 \pi \epsilon_0}(\frac{q_2}{r_{12}} + \frac{q_3}{r_{13}}) + \frac{q_2}{4 \pi \epsilon_0}(\frac{q_1}{r_{12}} + \frac{q_3}{r_{13}}) + \frac{q_3}{4 \pi \epsilon_0}(\frac{q_1}{r_{12}} + \frac{q_2}{r_{13}}) ) = \]
\[ \frac{1}{2} (q_1 \phi_1 + q_2 \phi_2  + q_3 \phi_3) \]

Продолжая аналогичным образом получим итоговую энергию системы зарядов:
\[W = \frac{1}{2} \sum_{i=1}^{N} \phi_i q_i\]
где потнециал $\phi_i$ создаваемый в точке qi всеми остальными зарядами

\subsection{Энергия заряженного проводника}
Возьмем проводник, заряд распределен на поверхности проводника, так разобьем всю поверхность на маленькие кусочки.
Важно отметить что поверхность проводника есть эквипотенциальная поверхность, тогда посчитаем энергию зарядов на этих кусочках:
\[W = \frac{1}{2} \sum_{i=1}^{N} (\phi_i \Delta q_i) = \frac{1}{2} \sum_{i=1}^{N} (\phi \Delta q_i) =
    \frac{1}{2} \phi \sum_{i=1}^{N} (\Delta q_i) = \frac{1}{2} \phi q = \frac{1}{2} \phi^2 \cdot C = \frac{q^2}{2C}
\]
\subsection{Энергия заряженного кондесатора}

\[ W = \frac{1}{2}\phi_1 q + \frac{1}{2}\phi_2 -q = \frac{1}{2}q (\phi_1 - \phi_2) = \frac{1}{2} q \Delta \phi
    = \frac{1}{2} q U = \frac{1}{2} C U^2 =\frac{1}{2} \frac{q^2}{C}\]
\subsection{Энергия электрического поля}
Наличие диэлектрика влияет на общую энергию электрического поля. Будем расматривать плоский конденсатор, подставим в эту формулы выражение для емкости:
\[ W = \frac{1}{2} C \cdot U^2 = \frac{1}{2} \frac{\epsilon \epsilon_0 S}{d} \cdot U^2 = \frac{1}{2} \frac{\epsilon \epsilon_0 S}{d} \cdot E^2 d^2 = \frac{1}{2}(\epsilon \epsilon_0 E^2) S \cdot d =
    \frac{\epsilon \epsilon_ 0 E^2}{2} \cdot V\]
\begin{center}
    где $V$ - объем конденсатора
\end{center}

Объемная плотность энергии электрического поля:
\[w = \frac{W}{V} = \frac{\epsilon \epsilon_ 0 E^2}{2} = \frac{D E}{2} = \frac{D^2}{2} \epsilon \epsilon_0\]
В анизотропных средах заметим что направления D и E не совпадают тогда берем скалярное поризведение векторов D и E:
\[w = \frac{\vec D \cdot \vec E}{2} = \frac{\vec E (\epsilon_0 \vec E + \vec P)}{2} = \frac{\epsilon_0 E^2}{2} + \frac{\vec E \vec P}{2}\]
\begin{center}
    $\frac{\epsilon_0 E^2}{2}$ - энергия электрического поля в вакууме,
    \newline
    $\frac{\vec E \vec P}{2}$ - энергия поляризации
\end{center}

\section{Постоянный электрический ток}

\define Электрическим током называется упорядоченное движение заряженных частиц.

За направление тока принимается движения положительно заряженных частиц.
\define Такое движение принято характеризовать \textbf{силой тока}  --- заряд прошедший через сечение проводника за единицу времени.
\[i = \frac{dq}{dt} = \frac{dq ^{+}}{dt} + \frac{dq ^{-}}{dt}\] --- сила переменного электрического тока.

Поскольку наши частицы есть своободные частицы, то движутся они хаотично. Тогда можем определить скорость хаотического движения зарядов $\vec v$,
а $\vec u$ - скрость упорядоченного движения заряда. Рассматривая ток внутри материала вводят величину плотности тока.

Сила тока величина \textit{скалярная}, плотность тока величина \textit{векторная} --- $\iota = \frac{I}{S}$,
плотность тока всегда направлена по линии движения тока. S есть площадь перпендикулярная направлению тока.

\[i = \iint_{s} \vec \iota \cdot \vec n ds\]

\[\vec \iota = q \cdot n \cdot \vec u = q^{+} \cdot n^{+} \cdot \vec u^{+} + q^{-} \cdot n^{-} \cdot \vec u^{-}\]
В системе СИ:

[I] = А (Ампер)
В свою очередь, через Амперы может вычислить и другие величины:
\begin{center}
    Кл = $A \cdot c$
    В = $\frac{\text{Дж}}{\text{Кл}}$ = $\frac{\text{кг} \cdot \text{м}^2}{\text{A} \cdot \text{с}^2}$
\end{center}

\subsection{Электродвижущая сила}
Возьмем уединенный проводник и поместим его в электрическое поле. Движение частиц по такому проводнику будет недолгим.
\begin{center}
    \definecolor{zzttqq}{rgb}{0.6,0.2,0}
    \definecolor{ttttff}{rgb}{0.2,0.2,1}
    \definecolor{fftttt}{rgb}{1,0.2,0.2}
    \definecolor{qqccqq}{rgb}{0,0.8,0}
    \begin{tikzpicture}[line cap=round,line join=round,>=triangle 45,x=1.0cm,y=1.0cm]
        \clip(0.87,59.87) rectangle (9.6,61.9);
        \fill[color=zzttqq,fill=zzttqq,fill opacity=0.1] (1.98,61.55) -- (9,61.57) -- (9,61) -- (2,61) -- cycle;
        \draw (4,23)-- (4,18);
        \draw [shift={(6.92,20.5)},color=qqccqq]  plot[domain=-1.51:1.51,variable=\t]({1*1.5*cos(\t r)+0*1.5*sin(\t r)},{0*1.5*cos(\t r)+1*1.5*sin(\t r)});
        \draw (7,22)-- (4.74,20.55);
        \draw (7,19)-- (4.74,20.55);
        \draw [color=fftttt](7.3,21.67) node[anchor=north west] {$+$};
        \draw [color=fftttt](7.59,20.98) node[anchor=north west] {$+$};
        \draw [color=fftttt](7.56,20.44) node[anchor=north west] {$+$};
        \draw [color=fftttt](7.42,20.01) node[anchor=north west] {$+$};
        \draw [color=fftttt](7.71,21.45) node[anchor=north west] {$+$};
        \draw [color=ttttff](5.16,20.91) node[anchor=north west] {$-$};
        \draw [color=ttttff](5.63,20.67) node[anchor=north west] {$-$};
        \draw [color=ttttff](5.54,21.24) node[anchor=north west] {$-$};
        \draw [shift={(4.28,20)},color=qqccqq]  plot[domain=0.67:1.69,variable=\t]({1*2.42*cos(\t r)+0*2.42*sin(\t r)},{0*2.42*cos(\t r)+1*2.42*sin(\t r)});
        \draw [shift={(4,18.79)},color=qqccqq]  plot[domain=0.95:1.57,variable=\t]({1*2.9*cos(\t r)+0*2.9*sin(\t r)},{0*2.9*cos(\t r)+1*2.9*sin(\t r)});
        \draw [shift={(4.47,21.39)},color=qqccqq]  plot[domain=4.44:5.4,variable=\t]({1*2.04*cos(\t r)+0*2.04*sin(\t r)},{0*2.04*cos(\t r)+1*2.04*sin(\t r)});
        \draw [shift={(4.92,20.88)},color=qqccqq]  plot[domain=4.29:5.52,variable=\t]({1*2.23*cos(\t r)+0*2.23*sin(\t r)},{0*2.23*cos(\t r)+1*2.23*sin(\t r)});
        \draw [color=qqccqq] (4.74,20.55)-- (4,20.57);
        \draw (3,40)-- (7,40);
        \draw (7,39)-- (7,40);
        \draw (3,39)-- (3,40);
        \draw (2.5,39)-- (3.54,39);
        \draw (6.6,38.98)-- (7.52,38.98);
        \draw (5,40)-- (5,39);
        \draw (4.56,38.98)-- (5.58,38.98);
        \draw (2.5,38)-- (3.54,38);
        \draw (3,38)-- (3,37);
        \draw (3,37)-- (7,37);
        \draw (7,38)-- (7,37);
        \draw (6.62,37.98)-- (7.56,37.98);
        \draw (4.56,37.98)-- (5.52,37.98);
        \draw (5,37.98)-- (5,37);
        \draw (5,40)-- (5,41);
        \draw (5,37)-- (4.97,35.87);
        \draw (2.8,38.86) node[anchor=north west] {$C_1$};
        \draw (4.85,38.93) node[anchor=north west] {$C_2$};
        \draw (5.82,38.95) node[anchor=north west] {$\dots$};
        \draw (6.91,38.95) node[anchor=north west] {$C_n$};
        \draw (2.52,39.82) node[anchor=north west] {$\phi_1$};
        \draw (4.66,39.75) node[anchor=north west] {$\phi_1$};
        \draw (6.57,39.69) node[anchor=north west] {$\phi_1$};
        \draw (2.54,38.08) node[anchor=north west] {$\phi_2$};
        \draw (4.57,38.01) node[anchor=north west] {$\phi_2$};
        \draw (6.6,38.01) node[anchor=north west] {$\phi_2$};
        \draw [color=zzttqq] (1.98,61.55)-- (9,61.57);
        \draw [color=zzttqq] (9,61.57)-- (9,61);
        \draw [color=zzttqq] (9,61)-- (2,61);
        \draw [color=zzttqq] (2,61)-- (1.98,61.55);
        \draw (2,61.64) node[anchor=north west] {$-$};
        \draw (8.53,61.67) node[anchor=north west] {$+$};
        \draw [shift={(4.54,73.01)}] plot[domain=4.51:5.07,variable=\t]({1*12.81*cos(\t r)+0*12.81*sin(\t r)},{0*12.81*cos(\t r)+1*12.81*sin(\t r)});
        \draw [shift={(1.92,60.87)}] plot[domain=1.45:4.7,variable=\t]({1*0.4*cos(\t r)+0*0.4*sin(\t r)},{0*0.4*cos(\t r)+1*0.4*sin(\t r)});
        \draw [color=fftttt](6.43,60.94) node[anchor=north west] {$+$};
        \draw [color=fftttt](5.72,60.8) node[anchor=north west] {$+$};
        \draw [color=fftttt](4.62,60.73) node[anchor=north west] {$+$};
        \draw [color=fftttt](3.58,60.77) node[anchor=north west] {$+$};
        \draw [color=fftttt](2.59,60.86) node[anchor=north west] {$+$};
    \end{tikzpicture}
\end{center}
Так для того чтобы движение частиц продолжилось нужны посторонние силы, которые засставляют заряженные частицы переместиться в начало:

Эти сторонние силы совершают работу по переносу заряда.

\define Величина равная работе сторонних сил, деленная на велиичну перенесенного заряда называется электродвижующей силы.

\[E = \frac{A}{q}\]

В системе СИ: [E] = В




\end{document}





















































