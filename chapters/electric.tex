\documentclass[../main.tex]{subfiles}
\graphicspath{{\subfix{../images/}}}

\begin{document}

\chapter{Электричество и магнетизм}

\section{Электрическое поле в вакууме}
Наименьший заряд в природе обозначается как \texttt{e} и называется электроном. Из такого утверждения, следует что любой заряд в природе может быть определен как  $ q = \pm e \cdot N, N \in \mathbb{Z}$

Заряд величина дискретная, может быть положительной и отрицательной, при соприкосновении разноименных зарядов они взаимоунитчтожаются.
\subsection{Закон сохранения заряда}
\textit{\textbf{Формулировка:} Заряд сохраняется в электрически замкнутой системе.}

--- \textit{Электрически замкнутая система} - система, через границу которой не проходят электрические заряды.

\subsection{Закон Кулона}
\textit{\textbf{Формулировка:} Все заряженные тела взаимодействуют между собой. При этом одноименные заряды отталкиваются, а разноименные притягиваются.}
Причем сила с которой они взаимодействуют равна:
\[ F = k\frac{q_1 q_2}{r^2} \text{, где k - коэффициент пропорциональности, q1,q2 - величины зарядов}\]

\define \textbf{Точечным зарядом} называется заряженное тело, размером которого можно пренебречь по сравнению с расстоянием до других заряженных тел.

В системе СИ заряд измеряют в Кулонах, при этом Кулон не основная единица измерения, а составная: $[q] = \text{Кл} = \text{Ам} \cdot \text{c}$. При измерении в Кулонах коэффициент k будет равен $k = 9 \cdot 10^9 \frac{\text{Нм}^2}{\text{Кл}^2}$

На практике в опытах также возникал некоторый коэффициент, поэтому при решении задач используют уже нормализованную форму:
\[ k = \frac{1}{4 \pi \epsilon_0} \text{где $\epsilon_0$ электрическая постоянная,  } \epsilon_0 = 8,85 \cdot10^{-12} \frac{\text{Ф}}{\text{м}}\]

\[F = \frac{1}{4 \pi \epsilon_0}\frac{q_1 q_2}{r^2}\]

\section{Электрическое поле}
\define Напряженность электрического поля
\[ \vec E = \frac{\vec F}{q} \text{(векторная)}\]
\begin{center}
    \definecolor{qqqqff}{rgb}{0,0,1}
    \definecolor{ffqqqq}{rgb}{1,0,0}
    \begin{tikzpicture}[line cap=round,line join=round,>=triangle 45,x=1.5755455765500463cm,y=1.4141895319101883cm]
        \clip(26.95,39.98) rectangle (37.37,42.39);
        \draw [line width=1.6pt,color=ffqqqq] (33,41) ellipse (0.45cm and 0.4cm);
        \draw [color=ffqqqq](32.83,41.34) node[anchor=north west] {$+$};
        \draw [line width=1.6pt,color=qqqqff] (35,41) ellipse (0.45cm and 0.4cm);
        \draw [color=qqqqff](34.82,41.34) node[anchor=north west] {$-$};
        \draw [->] (35.28,40.99) -- (36.4,40.97);
        \draw (35.57,41.77) node[anchor=north west] {$\vec E$};
        \draw [->] (34.71,40.99) -- (33.69,40.97);
        \draw (34.19,41.75) node[anchor=north west] {$\vec F$};
        \draw [line width=1.6pt,color=ffqqqq] (29.17,45.85) ellipse (0.45cm and 0.4cm);
        \draw [color=ffqqqq](29,46.2) node[anchor=north west] {$+$};
        \draw [line width=1.6pt,color=qqqqff] (31.87,45.96) ellipse (0.45cm and 0.4cm);
        \draw [color=qqqqff](31.75,46.34) node[anchor=north west] {$-$};
        \draw (32.91,41.74) node[anchor=north west] {q};
        \draw (34.82,41.94) node[anchor=north west] {$q_{\text{пр}}$};
        \draw [line width=1.6pt,color=ffqqqq] (27.96,41) ellipse (0.45cm and 0.4cm);
        \draw [color=ffqqqq](27.8,41.35) node[anchor=north west] {$+$};
        \draw [line width=1.6pt,color=ffqqqq] (29.68,41.02) ellipse (0.45cm and 0.4cm);
        \draw [color=ffqqqq](29.51,41.37) node[anchor=north west] {$+$};
        \draw [->] (29.96,41) -- (31.24,40.99);
        \draw (30.4,41.78) node[anchor=north west] {$\vec F$};
        \draw (30.34,41.03) node[anchor=north west] {$\vec E$};
        \draw (29.44,42.01) node[anchor=north west] {$q_{\text{пр}}$};
        \draw (27.89,41.82) node[anchor=north west] {q};
        \draw [line width=1.6pt,color=qqqqff] (28.02,39.04) ellipse (0.45cm and 0.4cm);
        \draw [color=qqqqff](27.83,39.37) node[anchor=north west] {$-$};
        \draw [line width=1.6pt,color=qqqqff] (29.69,39.02) ellipse (0.45cm and 0.4cm);
        \draw [color=qqqqff](29.51,39.35) node[anchor=north west] {$-$};
        \draw [->] (29.98,39) -- (31.22,39);
        \draw [line width=1.6pt,color=qqqqff] (33.03,39.02) ellipse (0.45cm and 0.4cm);
        \draw [color=qqqqff](32.85,39.35) node[anchor=north west] {$-$};
        \draw [line width=1.6pt,color=ffqqqq] (35,39.03) ellipse (0.45cm and 0.4cm);
        \draw [color=ffqqqq](34.83,39.37) node[anchor=north west] {$+$};
        \draw [->] (34.74,39.13) -- (33.63,39.15);
        \draw [->] (34.75,38.9) -- (33.59,38.88);
        \draw (30.31,39.83) node[anchor=north west] {$\vec F$};
        \draw (34.16,39.86) node[anchor=north west] {$\vec F$};
        \draw (30.28,38.92) node[anchor=north west] {$\vec E$};
        \draw (34.11,38.89) node[anchor=north west] {$\vec E$};
        \draw [line width=1.6pt,color=ffqqqq] (28.01,36.06) ellipse (0.45cm and 0.4cm);
        \draw [color=ffqqqq](27.85,36.4) node[anchor=north west] {$+$};
        \draw [line width=1.6pt,color=qqqqff] (30.96,36.05) ellipse (0.45cm and 0.4cm);
        \draw [color=qqqqff](30.84,36.44) node[anchor=north west] {$-$};
        \draw (28.29,36.02)-- (30.68,36.01);
        \draw [->] (30.68,36.01) -- (29,36.02);
        \draw (29.51,36.78) node[anchor=north west] {$\vec P$};
        \draw (29.56,36.09) node[anchor=north west] {$l$};
        \draw (27.96,39.7) node[anchor=north west] {q};
        \draw (27.96,36.83) node[anchor=north west] {q};
        \draw (31.04,36.78) node[anchor=north west] {q};
        \draw (32.98,39.7) node[anchor=north west] {q};
        \draw [line width=1.6pt,color=ffqqqq] (28.04,31.02) ellipse (0.45cm and 0.4cm);
        \draw [color=ffqqqq](27.86,31.37) node[anchor=north west] {$+$};
        \draw [->] (28.01,31.61) -- (27.99,32.55);
        \draw [->] (28.42,31.55) -- (29.28,32.33);
        \draw [->] (28.6,31.19) -- (29.66,31.25);
        \draw [->] (28.58,30.6) -- (29.54,30.15);
        \draw [->] (28.2,30.29) -- (28.71,29.58);
        \draw [->] (27.57,30.25) -- (26.96,29.66);
        \draw [->] (27.51,30.7) -- (26.45,30.35);
        \draw [->] (27.4,31.21) -- (26.51,31.73);
        \draw [line width=1.6pt,color=qqqqff] (31.99,31.14) ellipse (0.45cm and 0.4cm);
        \draw [color=qqqqff](31.81,31.47) node[anchor=north west] {$-$};
        \draw [->] (31.96,31.61) -- (31.88,32.49);
        \draw [->] (31.39,31.39) -- (30.7,31.82);
        \draw [->] (31.33,30.68) -- (30.48,30);
        \draw [->] (31.94,30.6) -- (32.21,29.7);
        \draw [->] (32.45,30.98) -- (33.28,30.58);
        \draw [->] (32.53,31.51) -- (33.32,32.02);
        \draw [line width=1.6pt,color=ffqqqq] (36,31) ellipse (0.45cm and 0.4cm);
        \draw [color=ffqqqq](35.86,31.33) node[anchor=north west] {$+$};
        \draw [line width=1.6pt,color=qqqqff] (38,31) ellipse (0.45cm and 0.4cm);
        \draw [color=qqqqff](37.85,31.36) node[anchor=north west] {$-$};
        \draw [->] (36.28,31.02) -- (37.72,31.02);
        \draw [->] (35.76,31.16) -- (35.5,31.5);
        \draw [->] (35.77,30.84) -- (35.47,30.61);
        \draw [->] (38.68,30.46) -- (38.21,30.81);
        \draw [->] (38.5,31.5) -- (38.13,31.25);
        \draw [shift={(37.02,30.3)}] plot[domain=0.83:2.32,variable=\t]({1*1.31*cos(\t r)+0*1.31*sin(\t r)},{0*1.31*cos(\t r)+1*1.31*sin(\t r)});
        \draw [shift={(36.98,31.65)}] plot[domain=3.95:5.49,variable=\t]({1*1.23*cos(\t r)+0*1.23*sin(\t r)},{0*1.23*cos(\t r)+1*1.23*sin(\t r)});
        \draw [shift={(36.97,29.37)}] plot[domain=1.16:1.96,variable=\t]({1*1.95*cos(\t r)+0*1.95*sin(\t r)},{0*1.95*cos(\t r)+1*1.95*sin(\t r)});
        \draw [shift={(36.76,33.3)}] plot[domain=4.47:5.1,variable=\t]({1*2.63*cos(\t r)+0*2.63*sin(\t r)},{0*2.63*cos(\t r)+1*2.63*sin(\t r)});
        \draw [rotate around={0:(31,53)},line width=1.2pt,dash pattern=on 1pt off 1pt] (31,53) ellipse (3.23cm and 0.63cm);
        \draw [dash pattern=on 1pt off 1pt] (31,53) ellipse (3.15cm and 2.83cm);
        \draw [rotate around={90:(31,53)},line width=1.2pt,dash pattern=on 1pt off 1pt] (31,53) ellipse (3.21cm and 0.56cm);
        \draw [line width=1.6pt,color=ffqqqq] (30.96,53.02) ellipse (0.45cm and 0.4cm);
        \draw [color=ffqqqq](30.79,53.36) node[anchor=north west] {$+$};
        \draw [->] (31.13,53.25) -- (32,54);
        \draw [->] (30.73,53.19) -- (29.99,53.63);
        \draw [->] (30.87,52.76) -- (30.36,52.04);
        \draw [->] (31.17,52.85) -- (32,52.31);
    \end{tikzpicture} %TODO: Фиксануть переполнение
\end{center}
Получим, что линии напряженности электрического поля направлены от положительного заряда к отрицательному:

\begin{center}

    \definecolor{qqqqff}{rgb}{0,0,1}
    \definecolor{ffqqqq}{rgb}{1,0,0}
    \begin{tikzpicture}[line cap=round,line join=round,>=triangle 45,x=1.3229395145530378cm,y=1.0cm]
        \clip(27.33,37.96) rectangle (36.62,40.02);
        \draw [line width=1.6pt,color=ffqqqq] (33,41) ellipse (0.37cm and 0.28cm);
        \draw [color=ffqqqq](32.83,41.34) node[anchor=north west] {$+$};
        \draw [line width=1.6pt,color=qqqqff] (35,41) ellipse (0.37cm and 0.28cm);
        \draw [color=qqqqff](34.82,41.34) node[anchor=north west] {$-$};
        \draw [->] (35.28,40.99) -- (36.4,40.97);
        \draw (35.57,41.77) node[anchor=north west] {$\vec E$};
        \draw [->] (34.71,40.99) -- (33.69,40.97);
        \draw (34.19,41.75) node[anchor=north west] {$\vec F$};
        \draw [line width=1.6pt,color=ffqqqq] (29.17,45.85) ellipse (0.37cm and 0.28cm);
        \draw [color=ffqqqq](29,46.2) node[anchor=north west] {$+$};
        \draw [line width=1.6pt,color=qqqqff] (31.87,45.96) ellipse (0.37cm and 0.28cm);
        \draw [color=qqqqff](31.75,46.34) node[anchor=north west] {$-$};
        \draw (32.91,41.74) node[anchor=north west] {q};
        \draw (34.82,41.94) node[anchor=north west] {$q_{\text{пр}}$};
        \draw [line width=1.6pt,color=ffqqqq] (27.96,41) ellipse (0.37cm and 0.28cm);
        \draw [color=ffqqqq](27.8,41.35) node[anchor=north west] {$+$};
        \draw [line width=1.6pt,color=ffqqqq] (29.68,41.02) ellipse (0.37cm and 0.28cm);
        \draw [color=ffqqqq](29.51,41.37) node[anchor=north west] {$+$};
        \draw [->] (29.96,41) -- (31.24,40.99);
        \draw (30.4,41.78) node[anchor=north west] {$\vec F$};
        \draw (30.34,41.03) node[anchor=north west] {$\vec E$};
        \draw (29.44,42.01) node[anchor=north west] {$q_{\text{пр}}$};
        \draw (27.89,41.82) node[anchor=north west] {q};
        \draw [line width=1.6pt,color=qqqqff] (28.02,39.04) ellipse (0.37cm and 0.28cm);
        \draw [color=qqqqff](27.83,39.37) node[anchor=north west] {$-$};
        \draw [line width=1.6pt,color=qqqqff] (29.69,39.02) ellipse (0.37cm and 0.28cm);
        \draw [color=qqqqff](29.51,39.35) node[anchor=north west] {$-$};
        \draw [->] (29.98,39) -- (31.22,39);
        \draw [line width=1.6pt,color=qqqqff] (33.03,39.02) ellipse (0.37cm and 0.28cm);
        \draw [color=qqqqff](32.85,39.35) node[anchor=north west] {$-$};
        \draw [line width=1.6pt,color=ffqqqq] (35,39.03) ellipse (0.37cm and 0.28cm);
        \draw [color=ffqqqq](34.83,39.37) node[anchor=north west] {$+$};
        \draw [->] (34.74,39.13) -- (33.63,39.15);
        \draw [->] (34.75,38.9) -- (33.59,38.88);
        \draw (30.31,39.83) node[anchor=north west] {$\vec F$};
        \draw (34.16,39.86) node[anchor=north west] {$\vec F$};
        \draw (30.28,38.92) node[anchor=north west] {$\vec E$};
        \draw (34.11,38.89) node[anchor=north west] {$\vec E$};
        \draw [line width=1.6pt,color=ffqqqq] (28.01,36.06) ellipse (0.37cm and 0.28cm);
        \draw [color=ffqqqq](27.85,36.4) node[anchor=north west] {$+$};
        \draw [line width=1.6pt,color=qqqqff] (30.96,36.05) ellipse (0.37cm and 0.28cm);
        \draw [color=qqqqff](30.84,36.44) node[anchor=north west] {$-$};
        \draw (28.29,36.02)-- (30.68,36.01);
        \draw [->] (30.68,36.01) -- (29,36.02);
        \draw (29.51,36.78) node[anchor=north west] {$\vec P$};
        \draw (29.56,36.09) node[anchor=north west] {$l$};
        \draw (27.96,39.7) node[anchor=north west] {q};
        \draw (27.96,36.83) node[anchor=north west] {q};
        \draw (31.04,36.78) node[anchor=north west] {q};
        \draw (32.98,39.7) node[anchor=north west] {q};
        \draw [line width=1.6pt,color=ffqqqq] (28.04,31.02) ellipse (0.37cm and 0.28cm);
        \draw [color=ffqqqq](27.86,31.37) node[anchor=north west] {$+$};
        \draw [->] (28.01,31.61) -- (27.99,32.55);
        \draw [->] (28.42,31.55) -- (29.28,32.33);
        \draw [->] (28.6,31.19) -- (29.66,31.25);
        \draw [->] (28.58,30.6) -- (29.54,30.15);
        \draw [->] (28.2,30.29) -- (28.71,29.58);
        \draw [->] (27.57,30.25) -- (26.96,29.66);
        \draw [->] (27.51,30.7) -- (26.45,30.35);
        \draw [->] (27.4,31.21) -- (26.51,31.73);
        \draw [line width=1.6pt,color=qqqqff] (31.99,31.14) ellipse (0.37cm and 0.28cm);
        \draw [color=qqqqff](31.81,31.47) node[anchor=north west] {$-$};
        \draw [->] (31.96,31.61) -- (31.88,32.49);
        \draw [->] (31.39,31.39) -- (30.7,31.82);
        \draw [->] (31.33,30.68) -- (30.48,30);
        \draw [->] (31.94,30.6) -- (32.21,29.7);
        \draw [->] (32.45,30.98) -- (33.28,30.58);
        \draw [->] (32.53,31.51) -- (33.32,32.02);
        \draw [line width=1.6pt,color=ffqqqq] (36,31) ellipse (0.37cm and 0.28cm);
        \draw [color=ffqqqq](35.86,31.33) node[anchor=north west] {$+$};
        \draw [line width=1.6pt,color=qqqqff] (38,31) ellipse (0.37cm and 0.28cm);
        \draw [color=qqqqff](37.85,31.36) node[anchor=north west] {$-$};
        \draw [->] (36.28,31.02) -- (37.72,31.02);
        \draw [->] (35.76,31.16) -- (35.5,31.5);
        \draw [->] (35.77,30.84) -- (35.47,30.61);
        \draw [->] (38.68,30.46) -- (38.21,30.81);
        \draw [->] (38.5,31.5) -- (38.13,31.25);
        \draw [shift={(37.02,30.3)}] plot[domain=0.83:2.32,variable=\t]({1*1.31*cos(\t r)+0*1.31*sin(\t r)},{0*1.31*cos(\t r)+1*1.31*sin(\t r)});
        \draw [shift={(36.98,31.65)}] plot[domain=3.95:5.49,variable=\t]({1*1.23*cos(\t r)+0*1.23*sin(\t r)},{0*1.23*cos(\t r)+1*1.23*sin(\t r)});
        \draw [shift={(36.97,29.37)}] plot[domain=1.16:1.96,variable=\t]({1*1.95*cos(\t r)+0*1.95*sin(\t r)},{0*1.95*cos(\t r)+1*1.95*sin(\t r)});
        \draw [shift={(36.76,33.3)}] plot[domain=4.47:5.1,variable=\t]({1*2.63*cos(\t r)+0*2.63*sin(\t r)},{0*2.63*cos(\t r)+1*2.63*sin(\t r)});
        \draw [rotate around={0:(31,53)},line width=1.2pt,dash pattern=on 1pt off 1pt] (31,53) ellipse (2.71cm and 0.44cm);
        \draw [dash pattern=on 1pt off 1pt] (31,53) ellipse (2.65cm and 2cm);
        \draw [rotate around={90:(31,53)},line width=1.2pt,dash pattern=on 1pt off 1pt] (31,53) ellipse (2.7cm and 0.39cm);
        \draw [line width=1.6pt,color=ffqqqq] (30.96,53.02) ellipse (0.37cm and 0.28cm);
        \draw [color=ffqqqq](30.79,53.36) node[anchor=north west] {$+$};
        \draw [->] (31.13,53.25) -- (32,54);
        \draw [->] (30.73,53.19) -- (29.99,53.63);
        \draw [->] (30.87,52.76) -- (30.36,52.04);
        \draw [->] (31.17,52.85) -- (32,52.31);
    \end{tikzpicture}
\end{center}

Величина напряженности точечного заряда:
\[ E = \frac{1}{4 \pi \epsilon_0}\frac{q}{r^2}\]

\textbf{Принцип суперпозиции полей:} Если имеется несколько электрических полей $\vec E_1, \vec E_2, \ldots , \vec E_n$, то
\[ \vec E = \sum_{i = 1}^{n} \vec E_i\]
Это вытекает из принципа суперпозции сил.
\subsection{Электрический диполь}
\define \textit{\textbf{Электрический диполь} - это такая структура, состоящая из пары зарядов. }
\begin{center}
    \definecolor{qqqqff}{rgb}{0,0,1}
    \definecolor{ffqqqq}{rgb}{1,0,0}
    \begin{tikzpicture}[line cap=round,line join=round,>=triangle 45,x=1.9772284268505984cm,y=1.6583433612467202cm]
        \clip(27.11,34.97) rectangle (32.23,37.1);
        \draw [line width=1.6pt,color=ffqqqq] (33,41) ellipse (0.56cm and 0.47cm);
        \draw [color=ffqqqq](32.83,41.34) node[anchor=north west] {$+$};
        \draw [line width=1.6pt,color=qqqqff] (35,41) ellipse (0.56cm and 0.47cm);
        \draw [color=qqqqff](34.82,41.34) node[anchor=north west] {$-$};
        \draw [->] (35.28,40.99) -- (36.4,40.97);
        \draw (35.57,41.77) node[anchor=north west] {$\vec E$};
        \draw [->] (34.71,40.99) -- (33.69,40.97);
        \draw (34.19,41.75) node[anchor=north west] {$\vec F$};
        \draw [line width=1.6pt,color=ffqqqq] (29.17,45.85) ellipse (0.56cm and 0.47cm);
        \draw [color=ffqqqq](29,46.2) node[anchor=north west] {$+$};
        \draw [line width=1.6pt,color=qqqqff] (31.87,45.96) ellipse (0.56cm and 0.47cm);
        \draw [color=qqqqff](31.75,46.34) node[anchor=north west] {$-$};
        \draw (32.91,41.74) node[anchor=north west] {q};
        \draw (34.82,41.94) node[anchor=north west] {$q_{\text{пр}}$};
        \draw [line width=1.6pt,color=ffqqqq] (27.96,41) ellipse (0.56cm and 0.47cm);
        \draw [color=ffqqqq](27.8,41.35) node[anchor=north west] {$+$};
        \draw [line width=1.6pt,color=ffqqqq] (29.68,41.02) ellipse (0.56cm and 0.47cm);
        \draw [color=ffqqqq](29.51,41.37) node[anchor=north west] {$+$};
        \draw [->] (29.96,41) -- (31.24,40.99);
        \draw (30.4,41.78) node[anchor=north west] {$\vec F$};
        \draw (30.34,41.03) node[anchor=north west] {$\vec E$};
        \draw (29.44,42.01) node[anchor=north west] {$q_{\text{пр}}$};
        \draw (27.89,41.82) node[anchor=north west] {q};
        \draw [line width=1.6pt,color=qqqqff] (28.02,39.04) ellipse (0.56cm and 0.47cm);
        \draw [color=qqqqff](27.83,39.37) node[anchor=north west] {$-$};
        \draw [line width=1.6pt,color=qqqqff] (29.69,39.02) ellipse (0.56cm and 0.47cm);
        \draw [color=qqqqff](29.51,39.35) node[anchor=north west] {$-$};
        \draw [->] (29.98,39) -- (31.22,39);
        \draw [line width=1.6pt,color=qqqqff] (33.03,39.02) ellipse (0.56cm and 0.47cm);
        \draw [color=qqqqff](32.85,39.35) node[anchor=north west] {$-$};
        \draw [line width=1.6pt,color=ffqqqq] (35,39.03) ellipse (0.56cm and 0.47cm);
        \draw [color=ffqqqq](34.83,39.37) node[anchor=north west] {$+$};
        \draw [->] (34.74,39.13) -- (33.63,39.15);
        \draw [->] (34.75,38.9) -- (33.59,38.88);
        \draw (30.31,39.83) node[anchor=north west] {$\vec F$};
        \draw (34.16,39.86) node[anchor=north west] {$\vec F$};
        \draw (30.28,38.92) node[anchor=north west] {$\vec E$};
        \draw (34.11,38.89) node[anchor=north west] {$\vec E$};
        \draw [line width=1.6pt,color=ffqqqq] (28.01,36.06) ellipse (0.56cm and 0.47cm);
        \draw [color=ffqqqq](27.85,36.4) node[anchor=north west] {$+$};
        \draw [line width=1.6pt,color=qqqqff] (30.96,36.05) ellipse (0.56cm and 0.47cm);
        \draw [color=qqqqff](30.84,36.44) node[anchor=north west] {$-$};
        \draw (28.29,36.02)-- (30.68,36.01);
        \draw [->] (30.68,36.01) -- (29,36.02);
        \draw (29.51,36.78) node[anchor=north west] {$\vec P$};
        \draw (29.56,36.09) node[anchor=north west] {$l$};
        \draw (27.96,39.7) node[anchor=north west] {q};
        \draw (27.96,36.83) node[anchor=north west] {q};
        \draw (31.04,36.78) node[anchor=north west] {q};
        \draw (32.98,39.7) node[anchor=north west] {q};
        \draw [line width=1.6pt,color=ffqqqq] (28.04,31.02) ellipse (0.56cm and 0.47cm);
        \draw [color=ffqqqq](27.86,31.37) node[anchor=north west] {$+$};
        \draw [->] (28.01,31.61) -- (27.99,32.55);
        \draw [->] (28.42,31.55) -- (29.28,32.33);
        \draw [->] (28.6,31.19) -- (29.66,31.25);
        \draw [->] (28.58,30.6) -- (29.54,30.15);
        \draw [->] (28.2,30.29) -- (28.71,29.58);
        \draw [->] (27.57,30.25) -- (26.96,29.66);
        \draw [->] (27.51,30.7) -- (26.45,30.35);
        \draw [->] (27.4,31.21) -- (26.51,31.73);
        \draw [line width=1.6pt,color=qqqqff] (31.99,31.14) ellipse (0.56cm and 0.47cm);
        \draw [color=qqqqff](31.81,31.47) node[anchor=north west] {$-$};
        \draw [->] (31.96,31.61) -- (31.88,32.49);
        \draw [->] (31.39,31.39) -- (30.7,31.82);
        \draw [->] (31.33,30.68) -- (30.48,30);
        \draw [->] (31.94,30.6) -- (32.21,29.7);
        \draw [->] (32.45,30.98) -- (33.28,30.58);
        \draw [->] (32.53,31.51) -- (33.32,32.02);
        \draw [line width=1.6pt,color=ffqqqq] (36,31) ellipse (0.56cm and 0.47cm);
        \draw [color=ffqqqq](35.86,31.33) node[anchor=north west] {$+$};
        \draw [line width=1.6pt,color=qqqqff] (38,31) ellipse (0.56cm and 0.47cm);
        \draw [color=qqqqff](37.85,31.36) node[anchor=north west] {$-$};
        \draw [->] (36.28,31.02) -- (37.72,31.02);
        \draw [->] (35.76,31.16) -- (35.5,31.5);
        \draw [->] (35.77,30.84) -- (35.47,30.61);
        \draw [->] (38.68,30.46) -- (38.21,30.81);
        \draw [->] (38.5,31.5) -- (38.13,31.25);
        \draw [shift={(37.02,30.3)}] plot[domain=0.83:2.32,variable=\t]({1*1.31*cos(\t r)+0*1.31*sin(\t r)},{0*1.31*cos(\t r)+1*1.31*sin(\t r)});
        \draw [shift={(36.98,31.65)}] plot[domain=3.95:5.49,variable=\t]({1*1.23*cos(\t r)+0*1.23*sin(\t r)},{0*1.23*cos(\t r)+1*1.23*sin(\t r)});
        \draw [shift={(36.97,29.37)}] plot[domain=1.16:1.96,variable=\t]({1*1.95*cos(\t r)+0*1.95*sin(\t r)},{0*1.95*cos(\t r)+1*1.95*sin(\t r)});
        \draw [shift={(36.76,33.3)}] plot[domain=4.47:5.1,variable=\t]({1*2.63*cos(\t r)+0*2.63*sin(\t r)},{0*2.63*cos(\t r)+1*2.63*sin(\t r)});
        \draw [rotate around={0:(31,53)},line width=1.2pt,dash pattern=on 1pt off 1pt] (31,53) ellipse (4.05cm and 0.73cm);
        \draw [dash pattern=on 1pt off 1pt] (31,53) ellipse (3.95cm and 3.32cm);
        \draw [rotate around={90:(31,53)},line width=1.2pt,dash pattern=on 1pt off 1pt] (31,53) ellipse (4.03cm and 0.65cm);
        \draw [line width=1.6pt,color=ffqqqq] (30.96,53.02) ellipse (0.56cm and 0.47cm);
        \draw [color=ffqqqq](30.79,53.36) node[anchor=north west] {$+$};
        \draw [->] (31.13,53.25) -- (32,54);
        \draw [->] (30.73,53.19) -- (29.99,53.63);
        \draw [->] (30.87,52.76) -- (30.36,52.04);
        \draw [->] (31.17,52.85) -- (32,52.31);
    \end{tikzpicture}
\end{center}

Для характеристики данной структуры используют \textbf{момент диполя}: $\vec P = q \cdot l$ , причем направлен он от отрицательного заряда к положительному.
\subsection{Линии напряженности электрического поля}
\begin{center}
    \definecolor{qqqqff}{rgb}{0,0,1}
    \definecolor{ffqqqq}{rgb}{1,0,0}
    \begin{tikzpicture}[line cap=round,line join=round,>=triangle 45,x=1.2300461177083677cm,y=1.2479929779551238cm]
        \clip(26.1,28.86) rectangle (39.14,32.89);
        \draw [line width=1.6pt,color=ffqqqq] (33,41) ellipse (0.35cm and 0.35cm);
        \draw [color=ffqqqq](32.84,41.36) node[anchor=north west] {$+$};
        \draw [line width=1.6pt,color=qqqqff] (35,41) ellipse (0.35cm and 0.35cm);
        \draw [color=qqqqff](34.83,41.35) node[anchor=north west] {$-$};
        \draw [->] (35.28,40.99) -- (36.4,40.97);
        \draw (35.57,41.79) node[anchor=north west] {$\vec E$};
        \draw [->] (34.71,40.99) -- (33.69,40.97);
        \draw (34.19,41.76) node[anchor=north west] {$\vec F$};
        \draw [line width=1.6pt,color=ffqqqq] (29.17,45.85) ellipse (0.35cm and 0.35cm);
        \draw [color=ffqqqq](29,46.2) node[anchor=north west] {$+$};
        \draw [line width=1.6pt,color=qqqqff] (31.87,45.96) ellipse (0.35cm and 0.35cm);
        \draw [color=qqqqff](31.74,46.36) node[anchor=north west] {$-$};
        \draw (32.93,41.74) node[anchor=north west] {q};
        \draw (34.81,41.95) node[anchor=north west] {$q_{\text{пр}}$};
        \draw [line width=1.6pt,color=ffqqqq] (27.96,41) ellipse (0.35cm and 0.35cm);
        \draw [color=ffqqqq](27.79,41.36) node[anchor=north west] {$+$};
        \draw [line width=1.6pt,color=ffqqqq] (29.68,41.02) ellipse (0.35cm and 0.35cm);
        \draw [color=ffqqqq](29.51,41.38) node[anchor=north west] {$+$};
        \draw [->] (29.96,41) -- (31.24,40.99);
        \draw (30.4,41.81) node[anchor=north west] {$\vec F$};
        \draw (30.33,41.06) node[anchor=north west] {$\vec E$};
        \draw (29.44,42.02) node[anchor=north west] {$q_{\text{пр}}$};
        \draw (27.9,41.83) node[anchor=north west] {q};
        \draw [line width=1.6pt,color=qqqqff] (28.02,39.04) ellipse (0.35cm and 0.35cm);
        \draw [color=qqqqff](27.83,39.39) node[anchor=north west] {$-$};
        \draw [line width=1.6pt,color=qqqqff] (29.69,39.02) ellipse (0.35cm and 0.35cm);
        \draw [color=qqqqff](29.51,39.37) node[anchor=north west] {$-$};
        \draw [->] (29.98,39) -- (31.22,39);
        \draw [line width=1.6pt,color=qqqqff] (33.03,39.02) ellipse (0.35cm and 0.35cm);
        \draw [color=qqqqff](32.84,39.37) node[anchor=north west] {$-$};
        \draw [line width=1.6pt,color=ffqqqq] (35,39.03) ellipse (0.35cm and 0.35cm);
        \draw [color=ffqqqq](34.83,39.39) node[anchor=north west] {$+$};
        \draw [->] (34.74,39.13) -- (33.63,39.15);
        \draw [->] (34.75,38.9) -- (33.59,38.88);
        \draw (30.3,39.84) node[anchor=north west] {$\vec F$};
        \draw (34.16,39.89) node[anchor=north west] {$\vec F$};
        \draw (30.28,38.93) node[anchor=north west] {$\vec E$};
        \draw (34.11,38.91) node[anchor=north west] {$\vec E$};
        \draw [line width=1.6pt,color=ffqqqq] (28.01,36.06) ellipse (0.35cm and 0.35cm);
        \draw [color=ffqqqq](27.85,36.41) node[anchor=north west] {$+$};
        \draw [line width=1.6pt,color=qqqqff] (30.96,36.05) ellipse (0.35cm and 0.35cm);
        \draw [color=qqqqff](30.83,36.46) node[anchor=north west] {$-$};
        \draw (28.29,36.02)-- (30.68,36.01);
        \draw [->] (30.68,36.01) -- (29,36.02);
        \draw (29.51,36.78) node[anchor=north west] {$\vec P$};
        \draw (29.54,36.1) node[anchor=north west] {$l$};
        \draw (27.95,39.72) node[anchor=north west] {q};
        \draw (27.95,36.85) node[anchor=north west] {q};
        \draw (31.04,36.78) node[anchor=north west] {q};
        \draw (32.98,39.72) node[anchor=north west] {q};
        \draw [line width=1.6pt,color=ffqqqq] (28.04,31.02) ellipse (0.35cm and 0.35cm);
        \draw [color=ffqqqq](27.86,31.38) node[anchor=north west] {$+$};
        \draw [->] (28.01,31.61) -- (27.99,32.55);
        \draw [->] (28.42,31.55) -- (29.28,32.33);
        \draw [->] (28.6,31.19) -- (29.66,31.25);
        \draw [->] (28.58,30.6) -- (29.54,30.15);
        \draw [->] (28.2,30.29) -- (28.71,29.58);
        \draw [->] (27.57,30.25) -- (26.96,29.66);
        \draw [->] (27.51,30.7) -- (26.45,30.35);
        \draw [->] (27.4,31.21) -- (26.51,31.73);
        \draw [line width=1.6pt,color=qqqqff] (31.99,31.14) ellipse (0.35cm and 0.35cm);
        \draw [color=qqqqff](31.81,31.48) node[anchor=north west] {$-$};
        \draw [->] (31.96,31.61) -- (31.88,32.49);
        \draw [->] (31.39,31.39) -- (30.7,31.82);
        \draw [->] (31.33,30.68) -- (30.48,30);
        \draw [->] (31.94,30.6) -- (32.21,29.7);
        \draw [->] (32.45,30.98) -- (33.28,30.58);
        \draw [->] (32.53,31.51) -- (33.32,32.02);
        \draw [line width=1.6pt,color=ffqqqq] (36,31) ellipse (0.35cm and 0.35cm);
        \draw [color=ffqqqq](35.84,31.34) node[anchor=north west] {$+$};
        \draw [line width=1.6pt,color=qqqqff] (38,31) ellipse (0.35cm and 0.35cm);
        \draw [color=qqqqff](37.85,31.36) node[anchor=north west] {$-$};
        \draw [->] (36.28,31.02) -- (37.72,31.02);
        \draw [->] (35.76,31.16) -- (35.5,31.5);
        \draw [->] (35.77,30.84) -- (35.47,30.61);
        \draw [->] (38.68,30.46) -- (38.21,30.81);
        \draw [->] (38.5,31.5) -- (38.13,31.25);
        \draw [shift={(37.02,30.3)}] plot[domain=0.83:2.32,variable=\t]({1*1.31*cos(\t r)+0*1.31*sin(\t r)},{0*1.31*cos(\t r)+1*1.31*sin(\t r)});
        \draw [shift={(36.98,31.65)}] plot[domain=3.95:5.49,variable=\t]({1*1.23*cos(\t r)+0*1.23*sin(\t r)},{0*1.23*cos(\t r)+1*1.23*sin(\t r)});
        \draw [shift={(36.97,29.37)}] plot[domain=1.16:1.96,variable=\t]({1*1.95*cos(\t r)+0*1.95*sin(\t r)},{0*1.95*cos(\t r)+1*1.95*sin(\t r)});
        \draw [shift={(36.76,33.3)}] plot[domain=4.47:5.1,variable=\t]({1*2.63*cos(\t r)+0*2.63*sin(\t r)},{0*2.63*cos(\t r)+1*2.63*sin(\t r)});
        \draw [rotate around={0:(31,53)},line width=1.2pt,dash pattern=on 1pt off 1pt] (31,53) ellipse (2.52cm and 0.55cm);
        \draw [dash pattern=on 1pt off 1pt] (31,53) ellipse (2.46cm and 2.5cm);
        \draw [rotate around={90:(31,53)},line width=1.2pt,dash pattern=on 1pt off 1pt] (31,53) ellipse (2.51cm and 0.49cm);
        \draw [line width=1.6pt,color=ffqqqq] (30.96,53.02) ellipse (0.35cm and 0.35cm);
        \draw [color=ffqqqq](30.8,53.38) node[anchor=north west] {$+$};
        \draw [->] (31.13,53.25) -- (32,54);
        \draw [->] (30.73,53.19) -- (29.99,53.63);
        \draw [->] (30.87,52.76) -- (30.36,52.04);
        \draw [->] (31.17,52.85) -- (32,52.31);
    \end{tikzpicture} %TODO: Переполнение
\end{center}
Линии напряженности электрического поля проводятся так, чтобы касательная к ним совпадала по направлению с вектором напряженности электрического поля.

Количество линий пропорционально величине напряженности электрического поля. Для подсчета таких линий, возьмем $dS$ - элементарная площадка, $dN$ - количество линий на этой площадке, тогда очевидно: $dN = EdS$, а для подсчета на всей поверхности возьмем поверхностны интеграл:

\[N = \iint\limits_S \vec E \cdot \vec n \cdot dS \text{где $\vec n$ - нормаль к поверхности S}\]

Произвольный интеграл такого вида называется потоком, то есть для произвольного вектора A поток $\Phi_{A_i}$

\[\vec \Phi_A = \iint\limits_S \vec A \cdot \vec n \cdot dS\]

\subsection{Теорема Гаусса}
\textit{\textbf{Формулировка:}} поток вектора электрического поля через замкнутую
поверхность равен алгебраической сумме зарядов, заключенной внутри поверхности и деленному
на электрическую постоянную $\epsilon_0$
\begin{center}
    \definecolor{qqqqff}{rgb}{0,0,1}
    \definecolor{ffqqqq}{rgb}{1,0,0}
    \begin{tikzpicture}[line cap=round,line join=round,>=triangle 45,x=1.0cm,y=1.0cm]
        \clip(28.02,50.44) rectangle (34.06,55.52);
        \draw [line width=1.6pt,color=ffqqqq] (33,41) circle (0.28cm);
        \draw [color=ffqqqq](32.84,41.36) node[anchor=north west] {$+$};
        \draw [line width=1.6pt,color=qqqqff] (35,41) circle (0.28cm);
        \draw [color=qqqqff](34.83,41.35) node[anchor=north west] {$-$};
        \draw [->] (35.28,40.99) -- (36.4,40.97);
        \draw (35.57,41.79) node[anchor=north west] {$\vec E$};
        \draw [->] (34.71,40.99) -- (33.69,40.97);
        \draw (34.19,41.76) node[anchor=north west] {$\vec F$};
        \draw [line width=1.6pt,color=ffqqqq] (29.17,45.85) circle (0.28cm);
        \draw [color=ffqqqq](29,46.2) node[anchor=north west] {$+$};
        \draw [line width=1.6pt,color=qqqqff] (31.87,45.96) circle (0.28cm);
        \draw [color=qqqqff](31.74,46.36) node[anchor=north west] {$-$};
        \draw (32.93,41.74) node[anchor=north west] {q};
        \draw (34.81,41.95) node[anchor=north west] {$q_{\text{пр}}$};
        \draw [line width=1.6pt,color=ffqqqq] (27.96,41) circle (0.28cm);
        \draw [color=ffqqqq](27.79,41.36) node[anchor=north west] {$+$};
        \draw [line width=1.6pt,color=ffqqqq] (29.68,41.02) circle (0.28cm);
        \draw [color=ffqqqq](29.51,41.38) node[anchor=north west] {$+$};
        \draw [->] (29.96,41) -- (31.24,40.99);
        \draw (30.4,41.81) node[anchor=north west] {$\vec F$};
        \draw (30.33,41.06) node[anchor=north west] {$\vec E$};
        \draw (29.44,42.02) node[anchor=north west] {$q_{\text{пр}}$};
        \draw (27.9,41.83) node[anchor=north west] {q};
        \draw [line width=1.6pt,color=qqqqff] (28.02,39.04) circle (0.28cm);
        \draw [color=qqqqff](27.83,39.39) node[anchor=north west] {$-$};
        \draw [line width=1.6pt,color=qqqqff] (29.69,39.02) circle (0.28cm);
        \draw [color=qqqqff](29.51,39.37) node[anchor=north west] {$-$};
        \draw [->] (29.98,39) -- (31.22,39);
        \draw [line width=1.6pt,color=qqqqff] (33.03,39.02) circle (0.28cm);
        \draw [color=qqqqff](32.84,39.37) node[anchor=north west] {$-$};
        \draw [line width=1.6pt,color=ffqqqq] (35,39.03) circle (0.28cm);
        \draw [color=ffqqqq](34.83,39.39) node[anchor=north west] {$+$};
        \draw [->] (34.74,39.13) -- (33.63,39.15);
        \draw [->] (34.75,38.9) -- (33.59,38.88);
        \draw (30.3,39.84) node[anchor=north west] {$\vec F$};
        \draw (34.16,39.89) node[anchor=north west] {$\vec F$};
        \draw (30.28,38.93) node[anchor=north west] {$\vec E$};
        \draw (34.11,38.91) node[anchor=north west] {$\vec E$};
        \draw [line width=1.6pt,color=ffqqqq] (28.01,36.06) circle (0.28cm);
        \draw [color=ffqqqq](27.85,36.41) node[anchor=north west] {$+$};
        \draw [line width=1.6pt,color=qqqqff] (30.96,36.05) circle (0.28cm);
        \draw [color=qqqqff](30.83,36.46) node[anchor=north west] {$-$};
        \draw (28.29,36.02)-- (30.68,36.01);
        \draw [->] (30.68,36.01) -- (29,36.02);
        \draw (29.51,36.78) node[anchor=north west] {$\vec P$};
        \draw (29.54,36.1) node[anchor=north west] {$l$};
        \draw (27.95,39.72) node[anchor=north west] {q};
        \draw (27.95,36.85) node[anchor=north west] {q};
        \draw (31.04,36.78) node[anchor=north west] {q};
        \draw (32.98,39.72) node[anchor=north west] {q};
        \draw [line width=1.6pt,color=ffqqqq] (28.04,31.02) circle (0.28cm);
        \draw [color=ffqqqq](27.86,31.38) node[anchor=north west] {$+$};
        \draw [->] (28.01,31.61) -- (27.99,32.55);
        \draw [->] (28.42,31.55) -- (29.28,32.33);
        \draw [->] (28.6,31.19) -- (29.66,31.25);
        \draw [->] (28.58,30.6) -- (29.54,30.15);
        \draw [->] (28.2,30.29) -- (28.71,29.58);
        \draw [->] (27.57,30.25) -- (26.96,29.66);
        \draw [->] (27.51,30.7) -- (26.45,30.35);
        \draw [->] (27.4,31.21) -- (26.51,31.73);
        \draw [line width=1.6pt,color=qqqqff] (31.99,31.14) circle (0.28cm);
        \draw [color=qqqqff](31.81,31.48) node[anchor=north west] {$-$};
        \draw [->] (31.96,31.61) -- (31.88,32.49);
        \draw [->] (31.39,31.39) -- (30.7,31.82);
        \draw [->] (31.33,30.68) -- (30.48,30);
        \draw [->] (31.94,30.6) -- (32.21,29.7);
        \draw [->] (32.45,30.98) -- (33.28,30.58);
        \draw [->] (32.53,31.51) -- (33.32,32.02);
        \draw [line width=1.6pt,color=ffqqqq] (36,31) circle (0.28cm);
        \draw [color=ffqqqq](35.84,31.34) node[anchor=north west] {$+$};
        \draw [line width=1.6pt,color=qqqqff] (38,31) circle (0.28cm);
        \draw [color=qqqqff](37.85,31.36) node[anchor=north west] {$-$};
        \draw [->] (36.28,31.02) -- (37.72,31.02);
        \draw [->] (35.76,31.16) -- (35.5,31.5);
        \draw [->] (35.77,30.84) -- (35.47,30.61);
        \draw [->] (38.68,30.46) -- (38.21,30.81);
        \draw [->] (38.5,31.5) -- (38.13,31.25);
        \draw [shift={(37.02,30.3)}] plot[domain=0.83:2.32,variable=\t]({1*1.31*cos(\t r)+0*1.31*sin(\t r)},{0*1.31*cos(\t r)+1*1.31*sin(\t r)});
        \draw [shift={(36.98,31.65)}] plot[domain=3.95:5.49,variable=\t]({1*1.23*cos(\t r)+0*1.23*sin(\t r)},{0*1.23*cos(\t r)+1*1.23*sin(\t r)});
        \draw [shift={(36.97,29.37)}] plot[domain=1.16:1.96,variable=\t]({1*1.95*cos(\t r)+0*1.95*sin(\t r)},{0*1.95*cos(\t r)+1*1.95*sin(\t r)});
        \draw [shift={(36.76,33.3)}] plot[domain=4.47:5.1,variable=\t]({1*2.63*cos(\t r)+0*2.63*sin(\t r)},{0*2.63*cos(\t r)+1*2.63*sin(\t r)});
        \draw [rotate around={0:(31,53)},line width=1.2pt,dash pattern=on 1pt off 1pt] (31,53) ellipse (2.05cm and 0.44cm);
        \draw [dash pattern=on 1pt off 1pt] (31,53) circle (2cm);
        \draw [rotate around={90:(31,53)},line width=1.2pt,dash pattern=on 1pt off 1pt] (31,53) ellipse (2.04cm and 0.39cm);
        \draw [line width=1.6pt,color=ffqqqq] (30.96,53.02) circle (0.28cm);
        \draw [color=ffqqqq](30.8,53.38) node[anchor=north west] {$+$};
        \draw [->] (31.13,53.25) -- (32,54);
        \draw [->] (30.73,53.19) -- (29.99,53.63);
        \draw [->] (30.87,52.76) -- (30.36,52.04);
        \draw [->] (31.17,52.85) -- (32,52.31);
    \end{tikzpicture}
\end{center}
Возьмем точечный заряд, окружим положительный заряд сферой радиуса $\vec r$
\[N = \iint\limits_S \vec E \cdot \vec n \cdot dS \]
Очевидно, что $\vec E, \vec n$ будут сонаправлены, где $\vec n$ -  нормаь ко всей сфере. Распишем эту формулу:
\[N = \iint\limits_S \vec E \cdot \vec n \cdot dS = \iint \limits_S \frac{1}{4 \pi \epsilon_0}\frac{q_1 q_2}{r^2} dS = \frac{1}{4 \pi \epsilon_0}\frac{q_1 q_2}{r^2} \iint \limits_S dS = \frac{1}{4 \pi \epsilon_0}\frac{q_1 q_2}{r^2} \cdot 4 \pi r^2 = \frac{q}{\epsilon_0}\]
Тогда можно сделать вывод, что число линий проходящих через любую замкнутую поверхность: $\frac{q}{\epsilon_0}$

В произвольном случае, в котором имеется несколько электрических \linebreak полей: $\vec E = \sum_{i = 1}^{n} \vec E_i $ получаем:
\[ \oiint\limits_S \vec E \cdot \vec n \, dS = \oiint\limits_S \sum_{i=1}^{n} \vec E_i \cdot \vec n \, dS =
    \sum_{i=1}^{n} \oiint\limits_S \vec E_i \cdot \vec n \, dS = \sum_{i=1}^{n} \frac{q_i}{\epsilon_0} =
    \frac{1}{\epsilon_0}\sum_{i=1}^{n} q_i\]
\begin{center}
    где $q_i$ находятся внутри замкнутого контура S
\end{center}


Таким образом, теорема Гаусса может быть записана таким образом:
\[ \oiint\limits_S \vec E \cdot \vec n \,  dS = \frac{1}{\epsilon_0} \sum_{i=1}^{n} q_i\]
\begin{center}
    или
\end{center}
\[\iint \limits_S \vec E \cdot \vec n \, dS  = \frac{1}{\epsilon_0} \iiint\limits_D \rho_\epsilon \, dV\]
\begin{center}
    где D - объем, заключенный внутри поверхности S,\linebreak $\rho$ - плотность электрического заряда
\end{center}

\define Объемная плотность заряда $\rho_q$ - общий заряд который имеется в объеме деленный на объем, то есть заряд в единице объема. 
Как правило распределена нераавномерно.Для того чтобы получить заряд в точке уменьшаем объем до точки, то есть делаем бесконечно малый объем.

\[ \rho_q(x,y,x) = \lim_{\Delta V \to 0} \frac{\Delta q}{\Delta V}\]

Поверхностная плотность заряда - средний заряд на единицу площади поверхности. 

%TODO : формула

Линейная плотность заряда  - собсвенно на линии, когда площадь стремится к нулю соотвественно.

%TODO : формула

\subsection{Поле бесконечно однородно заряженной плоскости}
\begin{center}
    \definecolor{qqttcc}{rgb}{0,0.2,0.8}
    \definecolor{ffttqq}{rgb}{1,0.2,0}
    \definecolor{qqqqff}{rgb}{0,0,1}
    \definecolor{ffqqqq}{rgb}{1,0,0}
    \begin{tikzpicture}[line cap=round,line join=round,>=triangle 45,x=1.0cm,y=1.0cm]
    \clip(27.04,62.43) rectangle (34.53,67.4);
    \draw [line width=1.6pt,color=ffqqqq] (29.17,45.85) circle (0.28cm);
    \draw [color=ffqqqq](29.04,46.11) node[anchor=north west] {$+$};
    \draw [line width=1.6pt,color=qqqqff] (31.87,45.96) circle (0.28cm);
    \draw [color=qqqqff](31.63,46.24) node[anchor=north west] {$-$};
    \draw [line width=1.6pt,color=ffttqq] (29,67)-- (29,63);
    \draw [line width=1.6pt,color=qqttcc] (32,67)-- (32,63);
    \draw [->] (29,66.64) -- (30.26,66.62);
    \draw [->] (29.03,64) -- (30.29,63.98);
    \draw [->] (29.04,65.3) -- (30.3,65.28);
    \draw [->] (30.69,66.61) -- (31.96,66.6);
    \draw [->] (30.71,65.31) -- (31.97,65.3);
    \draw [->] (30.74,64.01) -- (32,64);
    \draw [->] (29,66.64) -- (27.87,66.64);
    \draw [->] (28.99,65.31) -- (27.86,65.31);
    \draw [->] (28.96,64) -- (27.83,64);
    \draw [->] (33.08,66.6) -- (31.96,66.6);
    \draw [->] (33.1,65.3) -- (31.97,65.3);
    \draw [->] (33.13,64) -- (32,64);
    \draw (29.11,67.33) node[anchor=north west] {$+ \sigma $};
    \draw (32.13,67.28) node[anchor=north west] {$- \sigma$};
    \end{tikzpicture}
\end{center}
Пусть есть заряд $\sigma$ в некотором поле. Рассмотрим линии напряженности: силу симметрии линии напряженности будут перпендикулярны плоскости, с плюсом от нее и с минусом к ней соотвественно.
Воспользуемся теоремой гаусса для определения напряженности этого поля.Выделим замкнутую поверхность: выделим цилиндр с площадью основы S, посчитаем поток через эту цилиндрическую поверхность:
\begin{center}
    \definecolor{qqttcc}{rgb}{0,0.2,0.8}
    \definecolor{ffttqq}{rgb}{1,0.2,0}
    \definecolor{qqqqff}{rgb}{0,0,1}
    \definecolor{ffqqqq}{rgb}{1,0,0}
    \begin{tikzpicture}[line cap=round,line join=round,>=triangle 45,x=1.0cm,y=1.0cm]
    \clip(24.1,85.59) rectangle (32.42,94.37);
    \draw [line width=1.6pt,color=ffqqqq] (29.17,45.85) circle (0.28cm);
    \draw [color=ffqqqq](29.03,46.11) node[anchor=north west] {$+$};
    \draw [line width=1.6pt,color=qqqqff] (31.87,45.96) circle (0.28cm);
    \draw [color=qqqqff](31.63,46.24) node[anchor=north west] {$-$};
    \draw [line width=1.6pt,color=ffttqq] (29,67)-- (29,63);
    \draw [line width=1.6pt,color=qqttcc] (32,67)-- (32,63);
    \draw [->] (29,66.64) -- (30.26,66.62);
    \draw [->] (29.03,64) -- (30.29,63.98);
    \draw [->] (29.04,65.3) -- (30.3,65.28);
    \draw [->] (30.69,66.61) -- (31.96,66.6);
    \draw [->] (30.71,65.31) -- (31.97,65.3);
    \draw [->] (30.74,64.01) -- (32,64);
    \draw [->] (29,66.64) -- (27.87,66.64);
    \draw [->] (28.99,65.31) -- (27.86,65.31);
    \draw [->] (28.96,64) -- (27.83,64);
    \draw [->] (33.08,66.6) -- (31.96,66.6);
    \draw [->] (33.1,65.3) -- (31.97,65.3);
    \draw [->] (33.13,64) -- (32,64);
    \draw (29.1,67.33) node[anchor=north west] {$+ \sigma $};
    \draw (32.12,67.28) node[anchor=north west] {$- \sigma$};
    \draw [line width=2pt,color=ffttqq] (28,94)-- (28,86);
    \draw [line width=0.4pt,dash pattern=on 2pt off 2pt] (26,91)-- (30,91);
    \draw [line width=0.4pt,dash pattern=on 2pt off 2pt] (26,89)-- (30,89);
    \draw [rotate around={90:(26,90)},dash pattern=on 2pt off 2pt] (26,90) ellipse (1.02cm and 0.22cm);
    \draw [rotate around={90:(30,90)},dash pattern=on 2pt off 2pt] (30,90) ellipse (1.02cm and 0.21cm);
    \draw [->] (30,90) -- (31.38,90.01);
    \draw [->] (26,90) -- (24.55,89.99);
    \draw [->] (28,87) -- (30,87);
    \draw [->] (28,87) -- (26,87);
    \draw [->] (28,93) -- (30,93);
    \draw [->] (28,93) -- (26,93);
    \draw [->] (29,91) -- (29,92);
    \draw [->] (29,89) -- (29,88);
    \draw [rotate around={90:(28,90)},dash pattern=on 2pt off 2pt] (28,90) ellipse (1.03cm and 0.24cm);
    \draw (27.76,90.34) node[anchor=north west] {$S$};
    \draw (28.04,94.12) node[anchor=north west] {$+ \sigma$};
    \draw (30.58,90.54) node[anchor=north west] {$\vec n$};
    \draw (29.06,91.81) node[anchor=north west] {$\vec n$};
    \draw (25.06,90.55) node[anchor=north west] {$\vec n$};
    \draw (29.13,88.83) node[anchor=north west] {$\vec n$};
    \draw (28.13,87.61) node[anchor=north west] {$\vec E$};
    \draw (29.14,88.48) node[anchor=north west] {$\vec E \cdot \vec n = 0$};
    \end{tikzpicture}
\end{center}
%TODO : формула
Первый способ, из физических соображений, используя формулу потока:
\[ \Phi_E = \oiint\limits_S \vec E \cdot \vec n dS = \frac{1}{\epsilon_0} \sum_{}^{} q 
= \frac{1}{\epsilon_0} \sigma \cdot S \]

Второй способ, возьмем тот же цилиндр. Рассмотрим боковую поверхность цилиндра, поверхностный интеграл можно расписать как сумму соотвенно поверхностных интегралов по боковой поверхности и по основаниям. 
Заметим, что нормаль к боковой поверхности будет перпендикулярно силовым линиям, то есть:
\[ \oiint\limits_S \vec E \cdot \vec n dS = \iint\limits_{S_{\text{бок}}} \vec E \cdot \vec n dS + 2\iint\limits_{S_{\text{осн}}} \vec E \cdot \vec n dS 
= 2 E \cdot S \]

%TODO : формула

Получается, можно приравнять, потому что мы считаем одну и ту же площадь одно и того же тела разными способами, то есть:
\[\frac{1}{\epsilon_0} \sigma S = 2 \cdot E \cdot S \Rightarrow E = \frac{\sigma}{2 \epsilon_0}\]
рисунок
В реальности не существует бесконечной пластины, а у конечной пластины линии напряженности таковы:
\begin{center}
    \definecolor{qqttcc}{rgb}{0,0.2,0.8}
    \definecolor{ffttqq}{rgb}{1,0.2,0}
    \definecolor{qqqqff}{rgb}{0,0,1}
    \definecolor{ffqqqq}{rgb}{1,0,0}
    \begin{tikzpicture}[line cap=round,line join=round,>=triangle 45,x=1.0cm,y=1.0cm]
    \clip(25.66,86.82) rectangle (30.45,93.45);
    \draw [line width=1.6pt,color=ffqqqq] (29.17,45.85) circle (0.28cm);
    \draw [color=ffqqqq](29.03,46.11) node[anchor=north west] {$+$};
    \draw [line width=1.6pt,color=qqqqff] (31.87,45.96) circle (0.28cm);
    \draw [color=qqqqff](31.63,46.24) node[anchor=north west] {$-$};
    \draw [line width=1.6pt,color=ffttqq] (29,67)-- (29,63);
    \draw [line width=1.6pt,color=qqttcc] (32,67)-- (32,63);
    \draw [->] (29,66.64) -- (30.26,66.62);
    \draw [->] (29.03,64) -- (30.29,63.98);
    \draw [->] (29.04,65.3) -- (30.3,65.28);
    \draw [->] (30.69,66.61) -- (31.96,66.6);
    \draw [->] (30.71,65.31) -- (31.97,65.3);
    \draw [->] (30.74,64.01) -- (32,64);
    \draw [->] (29,66.64) -- (27.87,66.64);
    \draw [->] (28.99,65.31) -- (27.86,65.31);
    \draw [->] (28.96,64) -- (27.83,64);
    \draw [->] (33.08,66.6) -- (31.96,66.6);
    \draw [->] (33.1,65.3) -- (31.97,65.3);
    \draw [->] (33.13,64) -- (32,64);
    \draw (29.1,67.33) node[anchor=north west] {$+ \sigma $};
    \draw (32.12,67.28) node[anchor=north west] {$- \sigma$};
    \draw [line width=2pt,color=ffttqq] (28,93)-- (28,87);
    \draw [rotate around={90:(28,90)},dash pattern=on 1pt off 1pt] (28,90) ellipse (1.02cm and 0.21cm);
    \draw [shift={(28,93.5)}] plot[domain=3.79:5.64,variable=\t]({1*2.5*cos(\t r)+0*2.5*sin(\t r)},{0*2.5*cos(\t r)+1*2.5*sin(\t r)});
    \draw (26,90)-- (30,90);
    \draw [shift={(28,86.5)}] plot[domain=0.64:2.5,variable=\t]({1*2.5*cos(\t r)+0*2.5*sin(\t r)},{0*2.5*cos(\t r)+1*2.5*sin(\t r)});
    \draw [shift={(28,93)}] plot[domain=-3.14:0,variable=\t]({1*1*cos(\t r)+0*1*sin(\t r)},{0*1*cos(\t r)+1*1*sin(\t r)});
    \draw [shift={(28,87)}] plot[domain=0:3.14,variable=\t]({1*1*cos(\t r)+0*1*sin(\t r)},{0*1*cos(\t r)+1*1*sin(\t r)});
    \end{tikzpicture}
\end{center}
\subsection{Электрическое поле двух разноименно заряженных плоскостей}
\begin{center}
    \definecolor{qqttcc}{rgb}{0,0.2,0.8}
    \definecolor{ffttqq}{rgb}{1,0.2,0}
    \definecolor{qqqqff}{rgb}{0,0,1}
    \definecolor{ffqqqq}{rgb}{1,0,0}
    \begin{tikzpicture}[line cap=round,line join=round,>=triangle 45,x=1.0cm,y=1.0cm]
    \clip(24.82,85.73) rectangle (32.76,92.3);
    \draw [line width=1.6pt,color=ffqqqq] (29.17,45.85) circle (0.28cm);
    \draw [color=ffqqqq](29.03,46.11) node[anchor=north west] {$+$};
    \draw [line width=1.6pt,color=qqqqff] (31.87,45.96) circle (0.28cm);
    \draw [color=qqqqff](31.63,46.24) node[anchor=north west] {$-$};
    \draw [line width=1.6pt,color=ffttqq] (29,67)-- (29,63);
    \draw [line width=1.6pt,color=qqttcc] (32,67)-- (32,63);
    \draw [->] (29,66.64) -- (30.26,66.62);
    \draw [->] (29.03,64) -- (30.29,63.98);
    \draw [->] (29.04,65.3) -- (30.3,65.28);
    \draw [->] (30.69,66.61) -- (31.96,66.6);
    \draw [->] (30.71,65.31) -- (31.97,65.3);
    \draw [->] (30.74,64.01) -- (32,64);
    \draw [->] (29,66.64) -- (27.87,66.64);
    \draw [->] (28.99,65.31) -- (27.86,65.31);
    \draw [->] (28.96,64) -- (27.83,64);
    \draw [->] (33.08,66.6) -- (31.96,66.6);
    \draw [->] (33.1,65.3) -- (31.97,65.3);
    \draw [->] (33.13,64) -- (32,64);
    \draw (29.1,67.33) node[anchor=north west] {$+ \sigma $};
    \draw (32.12,67.28) node[anchor=north west] {$- \sigma$};
    \draw [line width=2pt,color=ffttqq] (26,92)-- (26,86);
    \draw [line width=2.4pt,color=qqttcc] (31,92)-- (31,86);
    \draw (26.25,92.27) node[anchor=north west] {$+ \sigma$};
    \draw (31.2,92.32) node[anchor=north west] {$- \sigma$};
    \draw [->] (26,91) -- (25,91);
    \draw [->] (26.04,89) -- (25,89);
    \draw [->] (26.04,87) -- (25,87);
    \draw [->] (32.08,91.64) -- (31.04,91.64);
    \draw [->] (32.04,89.89) -- (31.03,89.85);
    \draw [->] (26,91) -- (32,91);
    \draw [->] (32,88) -- (31,88);
    \draw [->] (26,89) -- (32,89);
    \draw [->] (26,87) -- (32,87);
    \draw [color=ffttqq](25.27,91.64) node[anchor=north west] {$\vec E_1$};
    \draw [color=ffttqq](25.29,89.66) node[anchor=north west] {$\vec E_1$};
    \draw [color=ffttqq](25.34,87.71) node[anchor=north west] {$\vec E_1$};
    \draw [->] (25,91.65) -- (31.04,91.64);
    \draw [->] (24.99,89.87) -- (31.03,89.85);
    \draw [->] (24.96,88.01) -- (31,88);
    \draw [color=qqttcc](30.12,92.24) node[anchor=north west] {$\vec E_2$};
    \draw [color=qqttcc](30.15,90.5) node[anchor=north west] {$\vec E_2$};
    \draw [color=qqttcc](30.19,88.75) node[anchor=north west] {$\vec E_2$};
    \draw (25.25,86.85) node[anchor=north west] {$I$};
    \draw (28.02,86.85) node[anchor=north west] {$II$};
    \draw (31.08,86.85) node[anchor=north west] {$III$};
    \end{tikzpicture}

\end{center}
Из геометрических соображений получаем, что поле будет сосредоточено только между этими плоскостями: $I, III : E = 0; II = \frac{\sigma}{epsilon_0}$. Причем такое поле однородно и направленно в одну сторону.
\subsection{Поле бесконечного однородно заряженного цилиндра}
\begin{center}
    \definecolor{qqttcc}{rgb}{0,0.2,0.8}
    \definecolor{ffttqq}{rgb}{1,0.2,0}
    \definecolor{qqqqff}{rgb}{0,0,1}
    \definecolor{ffqqqq}{rgb}{1,0,0}
    \begin{tikzpicture}[line cap=round,line join=round,>=triangle 45,x=1.0cm,y=1.0cm]
    \clip(23.81,85.51) rectangle (32.28,92.21);
    \draw [line width=1.6pt,color=ffqqqq] (29.17,45.85) circle (0.28cm);
    \draw [color=ffqqqq](29.03,46.11) node[anchor=north west] {$+$};
    \draw [line width=1.6pt,color=qqqqff] (31.87,45.96) circle (0.28cm);
    \draw [color=qqqqff](31.63,46.24) node[anchor=north west] {$-$};
    \draw [line width=1.6pt,color=ffttqq] (29,67)-- (29,63);
    \draw [line width=1.6pt,color=qqttcc] (32,67)-- (32,63);
    \draw [->] (29,66.64) -- (30.26,66.62);
    \draw [->] (29.03,64) -- (30.29,63.98);
    \draw [->] (29.04,65.3) -- (30.3,65.28);
    \draw [->] (30.69,66.61) -- (31.96,66.6);
    \draw [->] (30.71,65.31) -- (31.97,65.3);
    \draw [->] (30.74,64.01) -- (32,64);
    \draw [->] (29,66.64) -- (27.87,66.64);
    \draw [->] (28.99,65.31) -- (27.86,65.31);
    \draw [->] (28.96,64) -- (27.83,64);
    \draw [->] (33.08,66.6) -- (31.96,66.6);
    \draw [->] (33.1,65.3) -- (31.97,65.3);
    \draw [->] (33.13,64) -- (32,64);
    \draw (29.1,67.33) node[anchor=north west] {$+ \sigma $};
    \draw (32.12,67.28) node[anchor=north west] {$- \sigma$};
    \draw (23.75,86.91)-- (29.75,91.91);
    \draw (26.55,85.9)-- (31,90);
    \draw [line width=1.2pt,dash pattern=on 1pt off 1pt] (24.1,85.49)-- (31.37,91.81);
    \draw(27.99,88.88) circle (1.21cm);
    \draw (23.85,86.08) node[anchor=north west] {\textit{вообще это бесконечный цилиндр,думайте}};
    \draw [->] (27.99,88.88) -- (30,87);
    \draw [->] (29.26,88.4) -- (29.99,87.78);
    \draw [->] (29.96,89.04) -- (30.79,88.34);
    \draw [->] (30.63,89.65) -- (31.5,88.95);
    \draw (29.56,91.92) node[anchor=north west] {$\sigma$};
    \draw (30.69,90.25) node[anchor=north west] {$\sigma$};
    \draw (28.46,89.4) node[anchor=north west] {$R$};
    \draw (29.17,88.25) node[anchor=north west] {$\vec r$};
    \draw (29.64,88.69) node[anchor=north west] {$\vec E$};
    \draw (30.45,89.28) node[anchor=north west] {$\vec E$};
    \draw (31.1,89.83) node[anchor=north west] {$\vec E$};
    \end{tikzpicture}
\end{center}

По поверхности цилиндра распределен заряд с плотностью $\sigma$, в силу симметрии наши линии напряженности будут перпендикулярны поверхности цилиндра.
Возьмем некоторую точку, и пусть расстояние до нее от цилиндра r. Поступим аналогично: применим теорему гаусса, возмем замкнутутю поверхность в виде цилиндра который окружает наш цилиндр и посчитаем повехностные интегралы:
\begin{center}
    \definecolor{qqzzqq}{rgb}{0,0.6,0}
    \definecolor{fftttt}{rgb}{1,0.2,0.2}
    \definecolor{qqttcc}{rgb}{0,0.2,0.8}
    \definecolor{qqqqff}{rgb}{0,0,1}
    \definecolor{ffqqqq}{rgb}{1,0,0}
    \begin{tikzpicture}[line cap=round,line join=round,>=triangle 45,x=1.0cm,y=1.0cm]
    \clip(24.18,85.53) rectangle (32.29,91.39);
    \draw [line width=1.6pt,color=ffqqqq] (29.17,45.85) circle (0.28cm);
    \draw [color=ffqqqq](29.03,46.11) node[anchor=north west] {$+$};
    \draw [line width=1.6pt,color=qqqqff] (31.87,45.96) circle (0.28cm);
    \draw [color=qqqqff](31.63,46.24) node[anchor=north west] {$-$};
    \draw [line width=1.6pt,color=qqttcc] (32,67)-- (32,63);
    \draw [->] (29,66.64) -- (30.26,66.62);
    \draw [->] (29.03,64) -- (30.29,63.98);
    \draw [->] (29.04,65.3) -- (30.3,65.28);
    \draw [->] (30.69,66.61) -- (31.96,66.6);
    \draw [->] (30.71,65.31) -- (31.97,65.3);
    \draw [->] (30.74,64.01) -- (32,64);
    \draw [->] (29,66.64) -- (27.87,66.64);
    \draw [->] (28.99,65.31) -- (27.86,65.31);
    \draw [->] (28.96,64) -- (27.83,64);
    \draw [->] (33.08,66.6) -- (31.96,66.6);
    \draw [->] (33.1,65.3) -- (31.97,65.3);
    \draw [->] (33.13,64) -- (32,64);
    \draw (29.1,67.33) node[anchor=north west] {$+ \sigma $};
    \draw (32.12,67.28) node[anchor=north west] {$- \sigma$};
    \draw [color=fftttt] (26,88)-- (29,90);
    \draw [color=fftttt] (27,87)-- (30,89);
    \draw [rotate around={-48.83:(28.02,88.51)},line width=1.6pt,color=fftttt] (28.02,88.51) ellipse (0.73cm and 0.22cm);
    \draw [line width=1pt,dash pattern=on 1pt off 2pt] (29.52,91.17)-- (24.81,88.12);
    \draw [line width=1pt,dash pattern=on 1pt off 2pt] (31.67,88.78)-- (26.89,85.67);
    \draw [rotate around={-49.63:(25.85,86.89)},line width=1pt,dash pattern=on 1pt off 2pt] (25.85,86.89) ellipse (1.63cm and 0.26cm);
    \draw [rotate around={-47.98:(30.6,89.97)},line width=1pt,dash pattern=on 1pt off 2pt] (30.6,89.97) ellipse (1.63cm and 0.29cm);
    \draw [->] (26.21,89.03) -- (25.34,90.17);
    \draw [->] (25.8,86.87) -- (24.53,85.99);
    \draw [->] (30.53,90) -- (31.75,91.04);
    \draw [->] (26.67,89.33) -- (25.66,90.54);
    \draw [rotate around={-48.67:(28.13,88.37)},line width=1pt,dash pattern=on 1pt off 2pt] (28.13,88.37) ellipse (1.65cm and 0.38cm);
    \draw [->,color=qqzzqq] (28,88.54) -- (29.19,87.16);
    \draw [->,color=fftttt] (28,88.54) -- (27.56,89.04);
    \draw [color=fftttt](27.04,88.71) node[anchor=north west] {$R$};
    \draw [color=qqzzqq](28.76,88.1) node[anchor=north west] {$\vec r$};
    \draw (25.91,89.79) node[anchor=north west] {$\vec n$};
    \draw (24.69,86.74) node[anchor=north west] {$\vec n$};
    \draw (31.35,90.63) node[anchor=north west] {$\vec n$};
    \draw (26.38,90.2) node[anchor=north west] {$\vec E$};
    \end{tikzpicture}
\end{center}

Рассматриваем поток сквозь этот цилиндр:
\[\Phi_E = \oiint\limits_S \vec E \cdot \vec n ds = \frac{1}{\epsilon_0} 2 \pi R l \cdot \sigma \]

И разобьем на поверхности соотвественно: на основаниях нормаль будет перпендикурляно заряду, а соответственно нормали н абоковой поверхности будут сонаправлены векторам силового поля, то есть получаем:
%TODO : формула проверить хз 
\[\oiint\limits_S \vec E \cdot \vec n ds = \iint\limits_{S_{бок}} \vec E \cdot \vec n dS + 2\iint\limits_{S_{осн}} \vec E \cdot \vec n dS = E \iint\]

Тогда сопоставля два способа получаем(r - расстояние от оси цилиндра):
%TODO : формула итог 

Формула выше справдлива только для случая когда $r > R$, при $r < R \Rightarrow E = 0$. $r >> R$ в этом случае мы переходим от поверхностной плотности зарядя к линейной.Итак $\lambda = \sigma \cdot 2\pi R$, подставим:

\[ E = \frac{\lambda}{2\pi \epsilon_0 r}\]

\subsection{Сферически однородно заряженная поверхность}
\begin{center}
    \definecolor{ffttqq}{rgb}{1,0.2,0}
    \definecolor{qqwwqq}{rgb}{0,0.4,0}
    \definecolor{fftttt}{rgb}{1,0.2,0.2}
    \definecolor{qqttcc}{rgb}{0,0.2,0.8}
    \definecolor{qqqqff}{rgb}{0,0,1}
    \definecolor{ffqqqq}{rgb}{1,0,0}
    \begin{tikzpicture}[line cap=round,line join=round,>=triangle 45,x=1.0cm,y=1.0cm]
    \clip(22.43,84.42) rectangle (31.77,93.62);
    \draw [color=fftttt,fill=fftttt,fill opacity=0.1] (27,89) circle (2.64cm);
    \draw [line width=1pt,color=ffqqqq] (29.17,45.85) circle (0.28cm);
    \draw [color=ffqqqq](29.03,46.11) node[anchor=north west] {$+$};
    \draw [line width=1.6pt,color=qqqqff] (31.87,45.96) circle (0.28cm);
    \draw [color=qqqqff](31.63,46.24) node[anchor=north west] {$-$};
    \draw [line width=1.6pt,color=qqttcc] (32,67)-- (32,63);
    \draw [->] (29,66.64) -- (30.26,66.62);
    \draw [->] (29.03,64) -- (30.29,63.98);
    \draw [->] (29.04,65.3) -- (30.3,65.28);
    \draw [->] (30.69,66.61) -- (31.96,66.6);
    \draw [->] (30.71,65.31) -- (31.97,65.3);
    \draw [->] (30.74,64.01) -- (32,64);
    \draw [->] (29,66.64) -- (27.87,66.64);
    \draw [->] (28.99,65.31) -- (27.86,65.31);
    \draw [->] (28.96,64) -- (27.83,64);
    \draw [->] (33.08,66.6) -- (31.96,66.6);
    \draw [->] (33.1,65.3) -- (31.97,65.3);
    \draw [->] (33.13,64) -- (32,64);
    \draw (29.1,67.33) node[anchor=north west] {$+ \sigma $};
    \draw (32.12,67.28) node[anchor=north west] {$- \sigma$};
    \draw [line width=1pt,dash pattern=on 1pt off 2pt] (27,89) circle (3.76cm);
    \draw [rotate around={0:(27.01,89.01)},line width=1pt,dash pattern=on 1pt off 2pt] (27.01,89.01) ellipse (3.74cm and 0.63cm);
    \draw [rotate around={-0.28:(27,88.97)},color=fftttt,fill=fftttt,fill opacity=0.15] (27,88.97) ellipse (2.68cm and 0.45cm);
    \draw [->] (27,89) -- (29.68,88.96);
    \draw (28.1,89.41) node[anchor=north west] {$R$};
    \draw (27.75,91.48) node[anchor=north west] {$+ \sigma$};
    \draw [->,color=qqwwqq] (27,89) -- (24.35,91.66);
    \draw [->,color=ffttqq] (28.99,90.74) -- (29.7,91.22);
    \draw [->,color=ffttqq] (27.76,86.47) -- (27.98,85.73);
    \draw [->,color=ffttqq] (24.64,87.81) -- (23.87,87.37);
    \draw (29.06,91.51) node[anchor=north west] {$\vec E$};
    \draw (24,88.16) node[anchor=north west] {$\vec E$};
    \draw (27.97,86.53) node[anchor=north west] {$\vec E$};
    \draw [color=qqwwqq](24.91,91.54) node[anchor=north west] {$\vec r$};
    \end{tikzpicture}
\end{center}

Линии напряженности будут перпендикулярны поверхности сферы.
По аналогии с предыдущими доказательствами, возьмем сферу окружающую нашу сферу и применим теорему Гаусса:
\[\Phi_E = \oiint\]

%TODO : формула

Или можно посчитать с помощью формулы площади поверхности сферы:

%TODO : формула

При этом сфера замкнута и мы можем посчитать общий заряд на сфере и он будет равен:

%TODO : формула
И тогда при $r > R$:

%TODO : формула
Заметим, что напряженность поля точечного заряда совпадает с этой формулой.

\subsection{Работа сил электростатического поля}

Рассмотрим два заряда: заряд создающий поле соотвественно и пробный заряд,произвольно движущийся в этом поле. Нужно посчитать работу по перемещению этого заряда в поле.

Заметим, что действие силы поля напомнинает действие центральной силы, а именно что вектор силы будет на одной линии с неподвижной точкой, к тому же любая центральная сила консервативна и тогда переделы интегрирования будут определяться радиус векторами от начальной $q_0$ до первой точки и до второй соотвественно:

\begin{center}
    \definecolor{qqwwqq}{rgb}{0,0.4,0}
    \definecolor{qqttcc}{rgb}{0,0.2,0.8}
    \definecolor{qqqqff}{rgb}{0,0,1}
    \definecolor{ffqqqq}{rgb}{1,0,0}
    \begin{tikzpicture}[line cap=round,line join=round,>=triangle 45,x=1.0cm,y=1.0cm]
    \clip(22.52,87.6) rectangle (30.35,92.61);
    \draw [line width=1.6pt,color=ffqqqq] (29.17,45.85) circle (0.28cm);
    \draw [color=ffqqqq](29.03,46.11) node[anchor=north west] {$+$};
    \draw [line width=1.6pt,color=qqqqff] (31.87,45.96) circle (0.28cm);
    \draw [color=qqqqff](31.63,46.24) node[anchor=north west] {$-$};
    \draw [line width=1.6pt,color=qqttcc] (32,67)-- (32,63);
    \draw [->] (29,66.64) -- (30.26,66.62);
    \draw [->] (29.03,64) -- (30.29,63.98);
    \draw [->] (29.04,65.3) -- (30.3,65.28);
    \draw [->] (30.69,66.61) -- (31.96,66.6);
    \draw [->] (30.71,65.31) -- (31.97,65.3);
    \draw [->] (30.74,64.01) -- (32,64);
    \draw [->] (29,66.64) -- (27.87,66.64);
    \draw [->] (28.99,65.31) -- (27.86,65.31);
    \draw [->] (28.96,64) -- (27.83,64);
    \draw [->] (33.08,66.6) -- (31.96,66.6);
    \draw [->] (33.1,65.3) -- (31.97,65.3);
    \draw [->] (33.13,64) -- (32,64);
    \draw (29.1,67.33) node[anchor=north west] {$+ \sigma $};
    \draw (32.12,67.28) node[anchor=north west] {$- \sigma$};
    \draw [shift={(25.57,94.32)}] plot[domain=4.11:5.03,variable=\t]({1*3.35*cos(\t r)+0*3.35*sin(\t r)},{0*3.35*cos(\t r)+1*3.35*sin(\t r)});
    \draw [shift={(26.51,87.81)}] plot[domain=0.72:1.54,variable=\t]({1*3.32*cos(\t r)+0*3.32*sin(\t r)},{0*3.32*cos(\t r)+1*3.32*sin(\t r)});
    \draw [->,dash pattern=on 1pt off 2pt] (25,88.53) -- (23.66,91.56);
    \draw [->,dash pattern=on 1pt off 2pt] (25,88.53) -- (29,90);
    \draw (23.48,91.59) node[anchor=north west] {$1$};
    \draw (28.9,89.96) node[anchor=north west] {$2$};
    \draw (23.61,92.07) node[anchor=north west] {$q$};
    \draw [->,color=qqwwqq] (23.66,91.56) -- (23.37,92.32);
    \draw [->,color=qqwwqq] (29,90) -- (29.8,90.29);
    \draw [color=qqwwqq](23.01,91.86) node[anchor=north west] {$\vec F$};
    \draw [color=qqwwqq](29.38,90.21) node[anchor=north west] {$\vec F$};
    \draw (24.82,88.53) node[anchor=north west] {$q_0$};
    \draw (24.32,90.56) node[anchor=north west] {$r_1$};
    \draw (27.13,89.87) node[anchor=north west] {$r_2$};
    \end{tikzpicture}
\end{center}
%TODO : формула
Соответсвенно, если $r_1 = r_2$, то А = 0. При этом не обязательно что это будет одна и та же точка, главное чтобы радиус векторы совпадали по длине.
Рассмотрим замнутую дугу, то есть контур:

%TODO : формула

Так как наш сила консервативная, то можем ввести понятие потенциальной энергии, то есть: 

%TODO : формула

%TODO : формула

Для того чтобы посчитать эту энергию, нужно взять точку в которой потенциальная энергия равна нулю, а именно можно взять точку расстоние от которой приближается к бесконечности.
Тогда для нее $W \to 0$ при $r \to \inf$ получим что $C = 0$: 

%TODO : формула

Потенциальную энергию поле не принято считать характеристикой этого поля, посокльку зависит от заряда помещенного в это поле. 
Тогда рассматривают величину потенциальной энергии деленное на величину заряда помещенное в это поле - она и будет являться характеристикой этого поля.

\define \textbf{Потенциалом электрического поля} называется величина равная отношению потенциальной энергии заряженной частицы к величине этого заряда.
%TODO : формула

Если есть несколько частиц с потенциалами соотвественно $\phi_1, \phi_2 \ldots \phi_n$, то общий потенциал равен сумме этих потенциалов:

%TODO : формула

\define Величину $A = W_1 - W_2 = q\phi_1 - q\phi_2 = q(\phi_1 - \phi_2) = q\Delta \phi$ называют разностью потенциалов.

\end{document}
