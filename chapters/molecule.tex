\chapter{Молекулярно-кинетическая теория и термодинамика}

Молекулярно-кинетическая теория рассматривает состояния тел и переходы между агрегатными состояниями, преобразования различных тел с позиции что они состоят из большого числа молекул.

\vspace{5px}

\textbf{Положения молекулярно-кинетическая теории}
\begin{enumerate}
    \item Все тела состоят из огромного числа молекул
    \item Молекулы непрерывно и хаотично двигаются в телах
    \item Взаимодействие между молекулами различных тел - различно
\end{enumerate}
Важно отметить, что действие огромного числа молекул описывается статистическими законами. Будем рассматривать среднее значение молекул.

\vspace{5px}

\define \textit{\textbf{Термодинамика} - рассматривает различные термодинамические состояния системы и различные переходы между ними, причем рассматривает систему как единое целое. Состояния любой системы описывается с помощью термодинамических параметров : давление P, объем(удельный объем) V, температура Т.То есть любое состояние описывается функцией $F(P, V, T)$. }

\vspace{5px}

\define \textit{\textbf{Давление} - отношение силы к площади, к которой эта сила приложена:} $P = \frac{F}{S}$

\vspace{5px}

Причем, в газообразных и жидких телах - давление, в твердых - напряжение. Также в газообразных и жидких телах давление распределяется одинаково по всем направлениям. а в твердых - может и нет.

\vspace{5px}

За единицу атомной массы(атомный вес) принимают значение равное : $ M = \frac{1}{12}C^{12}$

\vspace{5px}

\define \textit{Количество вещества масса которого выражена в килограммах численно равна атомной массе называется киломоль (если в граммах - моль).}

\define Масса одного киломоля называется молярной массой вещества.

\vspace{5px}

Средняя кинетическая энергия всех молекул, обозначается: $\xi_k$

\vspace{5px}

Энергия взаимодействия между молекулами вещества: $u$

\vspace{5px}

Общая энергия вещества: $U = \xi_k + u$

\newpage

Зависимость между $\xi_k$ и $u$ определяется агрегатным состоянием вещества:
\begin{itemize}
    \item $\xi_k \gg u$ - газообразное
    \item $\xi_k \approx u$ - жидкое
    \item $\xi_k \ll u$ - твердое
\end{itemize}
У температуры нет точного и однозначного определения, но мы ее определили так:

\vspace{5px}

\define \textit{ Если два тела при соприкосновении обмениваются энергией в виде тепла, то значит у этих тел разные температуры, причем тела отдающие тепло, имеет более высокую температуру, а принимающие соответственно более низкую.}

\vspace{5px}

Для измерения используют специальные шкалы, самые популярные из них: \textit{шкала Цельсия}  - мировая практическая шкала, \textit{шкала Кельвина} - термодинамическая шкала.
\begin{itemize}
    \item шкала Цельсия: имеет 2 реперные точки - температура замерзания воды - $0 C^{\circ}$ и температура кипения воды - $100 C^{\circ}$
    \item шкала Кельвина: имеет 1 реперную точку - тройная точка воды(достигается при давлении 611 Па), это состояние когда вода находится сразу в трех агрегатных состояниях.

          Абсолютный ноль - 0К, $T_k = tC^{\circ} + 273,15$
\end{itemize}

\define \textit{\textbf{Идеальный газ} - это газ, который удовлетворяет нескольким условиям:}
\begin{enumerate}
    \item размер молекул пренебрежительно мал по сравнению с размером сосуда где находится газ(то есть размер молекул много меньше чем расстояние между ними)
    \item энергия взаимодействия между молекулами U = 0
    \item Столкновение молекул между собой и стенками сосуда считаются абсолютно упругими(то есть происходят без потери энергии)
\end{enumerate}
\section{Опытные газовые законы}

\subsection{Закон Авокадо}
\textit{\textbf{Формулировка:} киломоли всех идеальных газов при одних и тех температурах и давлениях занимают одинаковый объем. При этом при нормальных условиях($T_o = 273,15^{\circ}, P_o = 1,01 \cdot 10^5$Па) один киломоль идеального газа занимает объем $V_o = 22,4m^3$}.

\vspace{5px}

\textbf{Из этого следует вывод, что киломоли всех веществ содержат одинаковое число молекул, называемое числом Авогадро:
    $N_A = 6,023 \cdot 10^{26}$ в киломолях и соответственно (соответственно в молях: $N_a = 6,023 \cdot 10^{23}$)}

\vspace{5px}

Следующие законы связаны с постоянством одной из величин.В продолжении, масса вещества не изменяется.

\subsection{Закон Бойля-Мариотта}
\textit{\textbf{Формулировка:} При постоянной температуре и массе $m - const, T- const$ произведение $P \cdot V = const$ (произведение давления на объем - постоянная величина)}

\vspace{5px}

--- \textit{Процесс протекающий при $T = const$ называется \textbf{изотермальным}: }
\[ P_1V_1 = P_2V_2\]

--- \textit{Линия изображающая изотермальный процесс называется \textbf{изотермой}}

\begin{center}
    \begin{tikzpicture}[line cap=round,line join=round,>=triangle 45,x=1.0cm,y=1.0cm]
        \clip(13.63,15.36) rectangle (21.33,20.28);
        \draw [->] (15,15) -- (15,20);
        \draw [->] (14,16) -- (21,16);
        \draw [shift={(20.88,22.14)}] plot[domain=3.54:4.73,variable=\t]({1*5.69*cos(\t r)+0*5.69*sin(\t r)},{0*5.69*cos(\t r)+1*5.69*sin(\t r)});
        \draw [shift={(19.84,21.02)}] plot[domain=3.41:4.87,variable=\t]({1*4.92*cos(\t r)+0*4.92*sin(\t r)},{0*4.92*cos(\t r)+1*4.92*sin(\t r)});
        \draw (17.52,18.49) node[anchor=north west] {$T_2$};
        \draw (16.09,17.47) node[anchor=north west] {$T_1$};
        \draw (14.52,20.1) node[anchor=north west] {$P$};
        \draw (20.71,15.94) node[anchor=north west] {$V$};
    \end{tikzpicture}
\end{center}

\subsection{Закон Гей-Люссака}
\textit{\textbf{Формулировка:}}
\begin{enumerate}
    \item \textit{Объем любого газа при $ m-const, P - const$ линейно зависит от температуры($V = V_o(1 + \alpha t^\circ C)$, где $\alpha = \frac{1}{273,15}$)} %TODO: Фиксануть переполнение
    \item \textit{Давление любого газа при $ m-const, V - const$ линейно зависит от температуры $(P = P_o(1 + \alpha t^\circ C)$, где $\alpha = \frac{1}{273,15}$)}
\end{enumerate}
\begin{center}
    \begin{tikzpicture}[line cap=round,line join=round,>=triangle 45,x=1.0cm,y=1.0cm]
        \clip(9.31,14.04) rectangle (24.25,19.78);
        \draw [->] (11,14) -- (11,19);
        \draw [->] (9,15) -- (16,15);
        \draw [->] (19,14) -- (19,19);
        \draw [->] (17,15) -- (24,15);
        \draw [domain=10.0:24.2511344519391] plot(\x,{(--5--1*\x)/1});
        \draw [domain=18.0:24.2511344519391] plot(\x,{(-3--1*\x)/1});
        \draw (10,15)-- (17,19);
        \draw (18,15)-- (25,19);
        \draw (10.49,19.18) node[anchor=north west] {$V$};
        \draw (0.01,0.19) node[anchor=north west] {$V$};
        \draw (18.46,19.25) node[anchor=north west] {$P$};
        \draw (15.71,14.99) node[anchor=north west] {$T$};
        \draw (23.75,15) node[anchor=north west] {$T$};
        \draw (11.45,14.78) node[anchor=north west] {$P_2 > P_1$};
        \draw (19.41,14.83) node[anchor=north west] {$V_2 > V_1$};
        \draw (12.93,18.98) node[anchor=north west] {$P_1$};
        \draw (14.35,18.34) node[anchor=north west] {$P_2$};
        \draw (22.27,18.27) node[anchor=north west] {$V_2$};
        \draw (20.71,18.93) node[anchor=north west] {$V_1$};
        \draw (9.72,16.2) node[anchor=north west] {$- \frac{1}{\alpha}$};
        \draw (17.76,16.32) node[anchor=north west] {$- \frac{1}{\alpha}$};
        \draw (10.29,15.46)-- (10.41,15.34);
        \draw (10.61,15.71)-- (10.73,15.58);
        \draw (10.33,15.23)-- (10.4,15.14);
        \draw (10.56,15.35)-- (10.62,15.26);
        \draw (10.81,15.52)-- (10.89,15.42);
        \draw (18.2,15.43)-- (18.39,15.29);
        \draw (18.58,15.71)-- (18.76,15.56);
        \draw (18.28,15.23)-- (18.35,15.15);
        \draw (18.57,15.37)-- (18.63,15.28);
        \draw (18.82,15.53)-- (18.89,15.42);
    \end{tikzpicture}
\end{center}
\define \textit{Процесс, протекающий при постоянном объеме называется \textbf{изохорическим(изохорным)}, а соответствующая линия \textbf{изохорой.}}

\define \textit{Процесс, протекающий при постоянном давлении называется изобарическим(изобарным), а соответствующая линия изобарой.}

Тогда можно записать:
\[ V =V_o\frac{T}{T_o} \Rightarrow\]
\[
    \begin{cases}
        \frac{V}{V_o} = \frac{T}{T_o} \\
        \frac{P}{P_o} = \frac{T}{T_o}
    \end{cases}
\]

\subsection{Закон Менделеева- Клапейрона}
--- PV-диаграмма -- график, связывающий давление и объем.
\begin{center}
    \definecolor{qqwuqq}{rgb}{0,0.39,0}
    \definecolor{qqttcc}{rgb}{0,0.2,0.8}
    \definecolor{qqqqff}{rgb}{0,0,1}
    \begin{tikzpicture}[line cap=round,line join=round,>=triangle 45,x=1.4842926861455696cm,y=1.9344000062103033cm]
        \clip(13.19,12.42) rectangle (17.32,15.63);
        \draw [shift={(15.5,34)},color=qqwuqq,fill=qqwuqq,fill opacity=0.1] (0,0) -- (53.13:0.19) arc (53.13:161.57:0.19) -- cycle;
        \draw [->] (14,12.57) -- (14,15.5);
        \draw [->] (13.5,13) -- (17,13);
        \draw [shift={(16.15,15.15)},color=qqttcc]  plot[domain=3.23:4.62,variable=\t]({1*1.66*cos(\t r)+0*1.66*sin(\t r)},{0*1.66*cos(\t r)+1*1.66*sin(\t r)});
        \draw [dash pattern=on 1pt off 1pt] (14,15)-- (14.5,15);
        \draw [dash pattern=on 1pt off 1pt] (14.5,15)-- (14.5,13);
        \draw [dash pattern=on 1pt off 1pt] (16,14.5)-- (14,14.5);
        \draw [dash pattern=on 1pt off 1pt] (16,14.5)-- (16,13);
        \draw [rotate around={-54.68:(14.56,25.5)}] (14.56,25.5) ellipse (2.32cm and 2.24cm);
        \draw [->] (14,26) -- (15,26);
        \draw [->] (14.5,25) -- (15,26);
        \draw [->] (14,26) -- (14.5,25);
        \draw (14.33,26.41) node[anchor=north west] {$\vec a$};
        \draw (14.83,25.72) node[anchor=north west] {$\vec b$};
        \draw (13.96,25.63) node[anchor=north west] {$\vec r$};
        \draw (13.69,26.34) node[anchor=north west] {$A$};
        \draw (15.01,26.27) node[anchor=north west] {$M$};
        \draw (14.52,25.05) node[anchor=north west] {$B$};
        \draw [->] (12.5,35) -- (17,36);
        \draw [->] (12.5,35) -- (15.5,34);
        \draw [->] (15.5,34) -- (17,36);
        \draw (14.27,35.88) node[anchor=north west] {$\vec r+d \vec r$};
        \draw (15.78,35.22) node[anchor=north west] {$d\vec r$};
        \draw (14.22,34.84) node[anchor=north west] {$\vec r$};
        \draw [color=qqttcc](14.85,14.33) node[anchor=north west] {$T = const$};
        \draw (14.37,15.23) node[anchor=north west] {$T_1$};
        \draw (16.05,14.65) node[anchor=north west] {$T_2$};
        \draw [color=qqttcc](16.13,13.67) node[anchor=north west] {$P^{\prime}$};
        \draw (13.64,15.2) node[anchor=north west] {$P_1$};
        \draw (13.66,14.68) node[anchor=north west] {$P_2$};
        \draw (13.76,15.58) node[anchor=north west] {$P$};
        \draw (16.87,13.04) node[anchor=north west] {$V$};
        \draw (14.45,12.98) node[anchor=north west] {$V_1$};
        \draw (15.95,12.96) node[anchor=north west] {$V_2$};
        \begin{scriptsize}
            \fill [color=qqqqff] (14.5,15) circle (1.5pt);
            \fill [color=qqqqff] (16,13.5) circle (1.5pt);
            \fill [color=black] (16,14.5) circle (1.5pt);
            \fill [color=black] (12.5,35) circle (1.5pt);
            \fill [color=black] (17,36) circle (1.5pt);
            \fill [color=black] (15.5,34) circle (1.5pt);
        \end{scriptsize}
    \end{tikzpicture}
\end{center}
$P^\prime$ - промежуточное состояние. Пользуясь известными законами,попробуем перевести из состояния 1 в состояние 2.
\begin{enumerate}
    \item Изотермически приведем $V_1 -> V_2$
    \item Изохорически приведем к $T_2$
\end{enumerate}
Применим закон \textit{Бойля- Мариотта}: \[P_1V_1 = P_2V_2 \langle T = const = T_1 \rangle\]
Тогда, в силу: $\langle V_1 = const  = V_2 \rangle$
\[ \frac{P_1}{T_1} = \frac{P_2}{T_2} \] \[ P_1 = \frac{T_1P_2}{T_2} \]
\[ P_1V_1 = \frac{T_1P_2V_2}{T_2}\]
\[\frac{P_1V_1}{T_1} = \frac{P_2V_2}{T_2} \]

В силу произвольности выбранных состояний для данной массы данного газа справедливо:
\[ \frac{P \cdot V}{T} = const \]

\textbf{\textit{- уравнение Клапейрона}}

\vspace{5px}

Пусть $V_m$ - объем одного киломоля. Тогда в силу закона Авогадро:
\[ \frac{P \cdot V_m}{T} = R \]
- \textit{\textbf{уравнение состояний идеального газа,}} где R - одинаковая постоянная для всех газов.

\vspace{5px}

Тогда если $V = V_m \cdot \nu$, где $\nu = \frac{m}{\mu}$ - количество вещества. Получим:
\[ PV = \nu R T \Rightarrow PV = \frac{m}{\mu} R T \]
\textit{\textbf{- уравнение Менделеева-Клайперона}}


\subsection{Основное уравнение молекулярно-кинетической теории газов}

$\xi_i$ - энергия отдельной молекулы, $\bar v_i$ - скорость отдельной молекулы.

\vspace{6px}

\textbf{Внутренняя энергия газа:} \[ E = \sum_{i = 1}^N \frac{m \bar v_i^2}{2}\] где $N$ - количество молекул газа

\vspace{6px}

$<\epsilon> - $ \textbf{средняя энергия одной молекулы}:
\[ <\epsilon> = \frac{E}{N} = \frac{1}{N} \sum_{i = 1}^N \frac{m \bar v_i^2}{2} \]
Если рассматриваем однородный газ:
\[ \frac{1}{N} \cdot \frac{m}{2} \sum_{i = 1}^N {\bar v_i^2} = \frac{m}{2} \sum_{i = 1}^N \frac{\bar v_i^2}{N} = \frac{m}{2} \langle v^2 \rangle\]
где $\sum_{i = 1}^N \frac{\bar v_i^2}{N}$ - средний квадрат скорости

\vspace{6px}

$v_{sr} = \sqrt{\langle v^2 \rangle}$ - \textbf{среднеквадратичная скорость}, $\langle v \rangle$ - средняя величина скорости: \[ \langle v \rangle = \frac{1}{N}\sum_{i = 1}^N {v_i} \]

Заметим, $\langle v \rangle < v_{sr}$

\vspace{6px}
\begin{itemize}
    \item Макропараметры: $P, T ,\nu$
    \item Микропараметры $\epsilon, v_sr$
\end{itemize}

\begin{enumerate}
    \item Будем считать что все молекулы в газе движутся по трем взаимно перпендикулярным прямым
    \item Из всех молекул к стенке будет двигаться лишь $\frac{1}{6}$ часть молекул в объеме, так как лишь $\frac{1}{3}$ прямых перпендикулярна стенке и на ней 2 направления, то есть $\frac{1}{3} \cdot \frac{1}{2}$
    \item Все молекулы имеют скорость $v$
\end{enumerate}
\define \textit{\textbf{Концентрация} - количество молекул в единице объема:} $n. \Delta t$ - изменение времени

\vspace{3px}
\newpage
Пусть имеется стенка сосуда площадью $\Delta S$. Выясним $\Delta N$ - число молекул, которое ударяется о стенку за время $\Delta t$.

\vspace{3px}

Объем из которого молекулы могут долететь до стенки: \[V =  v \cdot \Delta t \cdot \Delta  S\] \begin{flushright}, где $\Delta N = \frac{1}{6} \cdot n \Delta V = \frac{1}{6} \cdot n  v \cdot \Delta t \cdot \Delta S $\end{flushright}


\textit{Импульс обозначим буквой j}: $j = mv$

\[ \Delta j = -mv -(mv) = -2mv \Rightarrow \] сама стенка получила тот же импульс но со знаком +

\vspace{3px}

Тогда общий импульс: \[ \Delta J = \Delta N \cdot \Delta j = \frac{1}{3} n v^2 m \Delta t \cdot \Delta S\]

Тогда используя запись второго закона Ньютона в импульсной форме получим:
\[ \Delta J = F \Delta t \]
\[ \langle P = \frac{F}{\Delta S} \Rightarrow F = P \Delta S \rangle \]
\[\Rightarrow \Delta J = P \Delta S \Delta t\]
Подставляя в $\Delta j = -mv -(mv) = -2mv$ и сокращая $\Delta t, \Delta S$ получим
\[ P = \frac{1}{3}nmv^2\]

Так как $n = \frac{N}{V}$ и $v^2 = \frac{1}{N}\sum_{i = 1}^N {v_i^2} \Rightarrow$
\[ m \cdot v^2 = \frac{2}{N}\sum_{i = 1}^N \frac{mv_i^2}{2} = 2<\epsilon> \Rightarrow \]
\[ P = \frac{2}{3}n<\epsilon>\]
- \textit{\textbf{основное уравнения МКТ}}

\vspace{5px}

Домножим последнее уравнение на объем:
\[ PV = \frac{2}{3}V \cdot n <\epsilon> \Rightarrow PV = \frac{2}{3}U \]
где $V \cdot n $ - общее число молекул N, а $U$ - внутренняя энергия молекул газа. По уравнению Менделеева-Клайперона получим:
\[ \nu R T = \frac{2}{3} U \Rightarrow U = \frac{3}{2} \nu R T\]
Рассмотрим эту формулу для одного киломоля: $ E = \frac{3}{2}RT$, здесь можно заметить что \underline{температура есть мера внутренней энергии молекул газа.}

В таком виде можем применить закон Авогадро:
\[ N_A \cdot <\epsilon> = \frac{3}{2}RT \Rightarrow <\epsilon> = \frac{3}{2}kT \]
\[ k = \frac{R}{N_A} \]
\[ k = 1,38 \cdot 10^{-23} \frac{\text{Дж}}{\text{К}} \text{   --- \textbf{\textit{постоянная Больцмана.}}}\]

\vspace{5px}

В общем, $kT$ определяет хаотическое движение молекул, электронов и других частиц. Подставив это соотношение в основное уравнение МКТ получим: $P = nkT$, то есть получили связь давления с температурой и концентрацией молекул в данном газе. Причем, эта формула верна для любых типов молекул.

Предположим что у нас имеется смесь газов, то есть $n = n_1 + n_2 + \ldots + n_S$, подставив в формулу выше получим
\[ P = n_1kT + n_2kT + \ldots + n_SkT\]
где $P_i = n_ikT$ - \textbf{\textit{парциальное давление}}
\subsection{Закон Дальтона}
\textit{\textbf{Формулировка:} Давление смеси газов складывается из суммы парциальных давлений: $P = P_1 +P_2 + \ldots + P_S$}
\section{Первое начало термодинамики}
Внутренняя энергия: $ U = \frac{3}{2} \nu R T$

\vspace{5px}

Причем $ \Delta U $ зависит от:

\vspace{5px}

--- \textit{тепла поступающего и уходящего из тела}

--- \textit{совершенной газом над телом работы или телом над газом}

\vspace{5px}

Будем обозначать $Q$ - \textbf{положительная тепловая энергия, измеряемая в Джоулях.}
\begin{itemize}
    \item $Q$ будет положительной если работа совершается газом над телом и если газу сообщают некоторое количество тепла
    \item $Q$ будет отрицательной если тело совершает работу над газом и сам газ сообщает телу некоторое количество тепла.
\end{itemize}
\textbf{\textit{Так \[ \Delta U = Q - A \Rightarrow Q = A + \Delta U\]
        - выражение первого начала термодинамики}}

\vspace{6px}

\textit{\textbf{Формулировка}: Тепло поступающее в термодинамическую систему расходуется на изменение ее внутренней энергии и на совершение работы системы над внешними телами}

\subsection{Теплоемкость}
\define \textit{\textbf{Теплоемкость} тела называется количество тепла, которое надо передать телу для того чтобы поднять его температуру на один градус: \[ C = \frac{Q}{\Delta T} \] }
\begin{enumerate}
    \item \textbf{Определение:} \textit{\textbf{Удельная теплоемкость }с - теплоемкость одного килограмма вещества: \[ c = \frac{Q}{\Delta T m}\]}
    \item \textbf{Определение:} \textit{\textbf{Молярная теплоемкость }$\mathbb{C}$ - теплоемкость одного киломоля вещества }
          \[ \mathbb{C} = \frac{Q}{\Delta T \nu} = \frac{m}{\nu}c = \mu c\]
\end{enumerate}
Важно отметить, что при различных процессах теплоемкость различается существенно.
\begin{itemize}
    \item \textbf{Изохорический процесс} $ V = const , \nu  = 1$ Киломоль. Так как $V = const \Rightarrow A = const$. Тогда получаем, что $U$ состоит из внутренней тепловой энергии, то есть $\Delta U = Q$, тогда для 1 киломоля: $\Delta U = \frac{3}{2}R \Delta T$

          Тогда согласно формуле молярной теплоемкости получим:
          \[ \mathbb{C}_V = \frac{Q}{\Delta T} = \frac{3}{2}R \approx 12,5 \frac{\text{Дж}}{\text{К}} \]

    \item \textbf{Изобарический процесс} $P = const$
          В таком процессе может присутствовать работа, то есть $Q = \Delta U + A$, тогда вычислим теплоемкость:
          \begin{equation}\label{eq:isobara}
              \frac{Q}{\Delta T} = \mathbb{C}_P = \frac{\Delta U}{\Delta T} + \frac{A}{\Delta T} = \frac{3}{2}R + \frac{A}{\Delta T}
          \end{equation}

          Очевидно, $\mathbb{C}_P > \mathbb{C}_V$, найдем $ \frac{A}{\Delta T}$

          Пусть есть некоторый сосуд, в котором есть поршень:
          \[
              \begin{cases}
                  PV_1 = \nu R T_2 \\
                  PV_2 = \nu R T_2
              \end{cases}
              \Rightarrow \]
          \[ P(V_2 - V_1) = \nu R (T_2 - T_1) \Rightarrow \langle \nu = 1 \rangle P \Delta V = R \Delta T\]
          Сила с которой газ давит на поршень выражается как :
          \[ F = const = P \cdot S\]

          Тогда работа будет равна:
          \[ A = F \cdot \Delta x = PS\Delta x = P \Delta V \Rightarrow A = R \Delta T\]
          Подставим в \hyperref[eq:isobara]{(\ref{eq:isobara})}:
          \[ \mathbb{C}_P = \mathbb{C}_V + \frac{A}{\Delta T} = \mathbb{C}_V + R\]
          - \textit{\textbf{формула Майера:} Теплоемкость при постоянном давлении равна сумме теплоемкости при постоянном объеме и газовой постоянной}
    \item \textbf{Изотермический процесс} $T = const$
          \[ \mathbb{C}_T = \frac{Q}{\Delta T} \to \infty \]
          При таком процессе можно сделать вывод, что при передаче тепло газ совершает большую работу. К тому, же сама теплоемкость может быть так положительной, так и отрицательной.
\end{itemize}
\subsection{Теплоемкость одноатомных и многоатомных газов}
Мы рассчитали что молярная теплоемкость газа равна: $\mathbb{C}_V \approx 12,5 \frac{\text{Дж}}{\text{К}}$. Однако экспериментально оказалось, что не у всех газов такая молярная теплоемкость, а лишь у инертных и одноатомных газов.

\vspace{4px}

Рассмотрим молекулы газов, точнее их степени свободы. Экспериментально было доказано что у многоатомных газов молярная теплоемкость было больше чем у одноатомных: $\approx 20,5 \frac{\text{Дж}}{\text{К}}$ - для двухатомных и $\approx 25,5 \frac{\text{Дж}}{\text{К}}$ - для трех и более атомных.

\vspace{4px}

Рассмотрим среднюю энергию молекулы:
\[ <\epsilon> = \frac{3}{2}kT \]
\[<\epsilon> = <\epsilon_x>+<\epsilon_y>+<\epsilon_z> = \frac{1}{2}kT + \frac{1}{2}kT+ \frac{1}{2}kT = \frac{3}{2}kT\]

Заметим, что \textit{поступательное движение может перейти во вращательное}, тем самым следует что на каждую степень \textit{свободы приходится одно и то же количество энергии}. Так, у двухатомной молекулы 5 степеней свободы( по Х,У,Z, и еще 2 вращательных направления), получаем
\[<\epsilon> = 3<\epsilon_{post}>+ 2<\epsilon_{vr}> = \frac{5}{2}kT\]
Соответственно для трехатомного газа:
\[<\epsilon> = 3<\epsilon_{post}>+ 3<\epsilon_{vr}> = \frac{6}{2}kT\]

\vspace{5px}

\textbf{\textit{То есть общая формула для молярной теплоемкости:}}
\[ \mathbb{C}_V = \frac{i}{2}R\]

\section{Термодинамические процессы в газах}
\subsection{Изохорнический процесс}
Заметим, что при постоянном объеме работа газом не совершается, то есть тепловая энергия по первому началу термодинамики есть изменение внутренней энергии :
\[Q = \Delta U = \nu \mathbb{C}_V \Delta T\].
\begin{center}
    \definecolor{qqqqff}{rgb}{0,0,1}
    \begin{tikzpicture}[line cap=round,line join=round,>=triangle 45,x=1.2976798350034187cm,y=1.0cm]
        \clip(17.72,15.73) rectangle (24.44,18.7);
        \draw [->] (20,16.5) -- (20,18.5);
        \draw [->] (20,16.5) -- (22.5,16.5);
        \draw (21,17)-- (21,18);
        \draw [dash pattern=on 1pt off 1pt] (20,17)-- (21,17);
        \draw [dash pattern=on 1pt off 1pt] (20,18)-- (21,18);
        \draw [dash pattern=on 1pt off 1pt] (21,17)-- (21,16.5);
        \draw (19.47,18.16) node[anchor=north west] {$P_1$};
        \draw (19.51,17.24) node[anchor=north west] {$P_2$};
        \draw (21.09,18.28) node[anchor=north west] {$1$};
        \draw (21.13,17.31) node[anchor=north west] {$2$};
        \draw (20.95,16.43) node[anchor=north west] {$V$};
        \draw [->] (36,34.5) -- (36,36.5);
        \draw [->] (36,34.5) -- (39.5,34.5);
        \begin{scriptsize}
            \fill [color=black] (21,17) circle (1.5pt);
            \fill [color=black] (21,18) circle (1.5pt);
            \fill [color=qqqqff] (36,34.5) circle (1.5pt);
            \draw[color=qqqqff] (15.55,20.89) node {$I$};
            \fill [color=qqqqff] (36,36.5) circle (1.5pt);
            \draw[color=qqqqff] (15.55,20.89) node {$J$};
            \fill [color=qqqqff] (39.5,34.5) circle (1.5pt);
            \draw[color=qqqqff] (15.57,20.89) node {$K$};
        \end{scriptsize}
    \end{tikzpicture}
\end{center}
\textit{Соответственно, $V = const$ есть уравнение этого процесса.}

\subsection{Изобарический процесс}
\begin{center}
    \begin{tikzpicture}[line cap=round,line join=round,>=triangle 45,x=1.5156326056017628cm,y=1.0cm]
        \clip(25.44,57.42) rectangle (31.26,60.31);
        \draw [->] (20,16.5) -- (20,18.5);
        \draw [->] (20,16.5) -- (22.5,16.5);
        \draw (21,17)-- (21,18);
        \draw [dash pattern=on 1pt off 1pt] (20,17)-- (21,17);
        \draw [dash pattern=on 1pt off 1pt] (20,18)-- (21,18);
        \draw [dash pattern=on 1pt off 1pt] (21,17)-- (21,16.5);
        \draw (19.47,18.16) node[anchor=north west] {$P_1$};
        \draw (19.51,17.24) node[anchor=north west] {$P_2$};
        \draw (21.09,18.28) node[anchor=north west] {$1$};
        \draw (21.12,17.31) node[anchor=north west] {$2$};
        \draw (20.95,16.43) node[anchor=north west] {$V$};
        \draw [->] (36,34.5) -- (36,36.5);
        \draw [->] (36,34.5) -- (39.5,34.5);
        \draw [->] (27,58) -- (27,60);
        \draw [->] (27,58) -- (30.5,58);
        \draw [dash pattern=on 1pt off 1pt] (27.5,58)-- (27.5,59);
        \draw (27.5,59)-- (29,59);
        \draw [dash pattern=on 1pt off 1pt] (29,59)-- (29,58);
        \draw [dash pattern=on 1pt off 1pt] (27.5,59)-- (27,59);
        \draw (26.43,59.25) node[anchor=north west] {$P$};
        \draw (27.4,57.83) node[anchor=north west] {$V_1$};
        \draw (28.91,57.88) node[anchor=north west] {$V_2$};
        \draw (27.97,58.78) node[anchor=north west] {$A$};
        \draw (27.43,59.4) node[anchor=north west] {$1$};
        \draw (28.92,59.4) node[anchor=north west] {$2$};
        \begin{scriptsize}
            \fill [color=black] (21,17) circle (1.5pt);
            \fill [color=black] (21,18) circle (1.5pt);
            \fill [color=black] (27.5,59) circle (1.5pt);
            \fill [color=black] (29,59) circle (1.5pt);
        \end{scriptsize}
    \end{tikzpicture}
\end{center}
\[ P = const \]
\[ A = P \Delta V, Q = \Delta U + A\]
Отсюда явно видим, что по сути \textit{значение работы совпадает со значением площади прямоугольника со сторонами $<P,V_1V_2>$ соответственно.}
\begin{center}
    \begin{tikzpicture}[line cap=round,line join=round,>=triangle 45,x=1.5577860350557382cm,y=1.7832740066740247cm]
        \clip(32.2,69.57) rectangle (37.58,72.13);
        \draw [->] (20,16.5) -- (20,18.5);
        \draw [->] (20,16.5) -- (22.5,16.5);
        \draw (21,17)-- (21,18);
        \draw [dash pattern=on 1pt off 1pt] (20,17)-- (21,17);
        \draw [dash pattern=on 1pt off 1pt] (20,18)-- (21,18);
        \draw [dash pattern=on 1pt off 1pt] (21,17)-- (21,16.5);
        \draw (19.47,18.13) node[anchor=north west] {$P_1$};
        \draw (19.51,17.21) node[anchor=north west] {$P_2$};
        \draw (21.09,18.25) node[anchor=north west] {$1$};
        \draw (21.13,17.28) node[anchor=north west] {$2$};
        \draw (20.95,16.4) node[anchor=north west] {$V$};
        \draw [->] (36,34.5) -- (36,36.5);
        \draw [->] (36,34.5) -- (39.5,34.5);
        \draw [->] (27,58) -- (27,60);
        \draw [->] (27,58) -- (30.5,58);
        \draw [dash pattern=on 1pt off 1pt] (27.5,58)-- (27.5,59);
        \draw (27.5,59)-- (29,59);
        \draw [dash pattern=on 1pt off 1pt] (29,59)-- (29,58);
        \draw [dash pattern=on 1pt off 1pt] (27.5,59)-- (27,59);
        \draw (26.43,59.22) node[anchor=north west] {$P$};
        \draw (27.4,57.8) node[anchor=north west] {$V_1$};
        \draw (28.92,57.85) node[anchor=north west] {$V_2$};
        \draw (27.97,58.74) node[anchor=north west] {$A$};
        \draw (27.44,59.37) node[anchor=north west] {$1$};
        \draw (28.92,59.37) node[anchor=north west] {$2$};
        \draw [->] (33,70) -- (33,72);
        \draw [->] (33,70) -- (37,70);
        \draw (33.5,71.5)-- (33.98,71.16)-- (34.5,71)-- (34.88,71.25)-- (35.24,71.01)-- (35.5,71)-- (35.8,71.01)-- (35.92,70.83)-- (36.5,70.71);
        \draw [dash pattern=on 1pt off 1pt] (33.5,71.5)-- (33.5,70);
        \draw [dash pattern=on 1pt off 1pt] (33.61,71.42)-- (33.64,70);
        \draw [dash pattern=on 1pt off 1pt] (33.72,71.34)-- (33.74,69.99);
        \draw [dash pattern=on 1pt off 1pt] (33.81,71.28)-- (33.87,70);
        \draw [dash pattern=on 1pt off 1pt] (36.5,70.71)-- (33,70.73);
        \draw [dash pattern=on 1pt off 1pt] (33.5,71.5)-- (33,71.5);
        \draw [dash pattern=on 1pt off 1pt] (36.5,70.71)-- (36.5,70);
        \draw (33.45,69.91) node[anchor=north west] {$V_1$};
        \draw (36.43,69.93) node[anchor=north west] {$V_2$};
        \draw (32.67,70.9) node[anchor=north west] {$P_1$};
        \draw (32.66,71.68) node[anchor=north west] {$P_2$};
        \begin{scriptsize}
            \fill [color=black] (33.5,71.5) circle (1.5pt);
            \fill [color=black] (36.5,70.71) circle (1.5pt);
        \end{scriptsize}
    \end{tikzpicture}
\end{center}
Рассмотрим диаграмму для произвольного процесса, учитывая этот "геометрический" смысл разобьем эту диаграмму на бесконечно малые прямоугольники, где $P = const$. Тогда, $\Delta A_i = P_i \Delta V_i$, где $A = \sum_i A_i$. При $\Delta V \to 0$ получим интеграл, с помощью которого может быть вычислена работа в любом процессе:
\[ A = \int_{V_1}^{V_2}P(V)dV\]

\subsection{Изотермический процесс}
При постоянной температуре, учитывая первое начало термодинамики, не изменяется внутренняя энергия  и все тепло уходит на совершение работы: $Q = A$.

Рассмотрим закон Менделеева-Клапейрона для этого процесса:
\[ PV = \nu RT \Rightarrow P = \frac{\nu R T}{V} \langle T = const \rangle\]

Тогда диаграмма для этого процесса выглядит так:
\begin{center}
    \begin{tikzpicture}[line cap=round,line join=round,>=triangle 45,x=1.8726252637316438cm,y=1.664810443587885cm]
        \clip(32.76,69.59) rectangle (37.34,72.6);
        \draw [->] (20,16.5) -- (20,18.5);
        \draw [->] (20,16.5) -- (22.5,16.5);
        \draw (21,17)-- (21,18);
        \draw [dash pattern=on 1pt off 1pt] (20,17)-- (21,17);
        \draw [dash pattern=on 1pt off 1pt] (20,18)-- (21,18);
        \draw [dash pattern=on 1pt off 1pt] (21,17)-- (21,16.5);
        \draw (19.47,18.13) node[anchor=north west] {$P_1$};
        \draw (19.51,17.21) node[anchor=north west] {$P_2$};
        \draw (21.09,18.25) node[anchor=north west] {$1$};
        \draw (21.13,17.28) node[anchor=north west] {$2$};
        \draw (20.95,16.4) node[anchor=north west] {$V$};
        \draw [->] (36,34.5) -- (36,36.5);
        \draw [->] (36,34.5) -- (39.5,34.5);
        \draw [->] (27,58) -- (27,60);
        \draw [->] (27,58) -- (30.5,58);
        \draw [dash pattern=on 1pt off 1pt] (27.5,58)-- (27.5,59);
        \draw (27.5,59)-- (29,59);
        \draw [dash pattern=on 1pt off 1pt] (29,59)-- (29,58);
        \draw [dash pattern=on 1pt off 1pt] (27.5,59)-- (27,59);
        \draw (26.43,59.22) node[anchor=north west] {$P$};
        \draw (27.4,57.8) node[anchor=north west] {$V_1$};
        \draw (28.92,57.85) node[anchor=north west] {$V_2$};
        \draw (27.97,58.74) node[anchor=north west] {$A$};
        \draw (27.44,59.37) node[anchor=north west] {$1$};
        \draw (28.92,59.37) node[anchor=north west] {$2$};
        \draw [->] (33.5,70) -- (33.5,72.5);
        \draw [->] (33.5,70) -- (37,70);
        \draw [shift={(36.13,72.68)}] plot[domain=3.32:4.65,variable=\t]({1*2.18*cos(\t r)+0*2.18*sin(\t r)},{0*2.18*cos(\t r)+1*2.18*sin(\t r)});
        \draw (36,70.5)-- (36.7,70.5);
        \draw [dash pattern=on 1pt off 1pt] (33.47,72.28)-- (33.99,72.28);
        \draw [dash pattern=on 1pt off 1pt] (36.7,70.5)-- (36.71,70);
        \draw [dash pattern=on 1pt off 1pt] (36.7,70.5)-- (33.5,70.5);
        \draw [dash pattern=on 1pt off 1pt] (33.99,72.28)-- (34,70);
        \draw (33.15,72.47) node[anchor=north west] {$P_1$};
        \draw (33.18,70.71) node[anchor=north west] {$P_2$};
        \draw (33.94,69.92) node[anchor=north west] {$V_1$};
        \draw (36.75,69.89) node[anchor=north west] {$V_2$};
        \begin{scriptsize}
            \fill [color=black] (33.99,72.28) circle (1.5pt);
            \fill [color=black] (36.7,70.5) circle (1.5pt);
        \end{scriptsize}
    \end{tikzpicture}
\end{center}
Вычислим работу:
\[ A = \int_{V_1}^{V_2}P(V)dV = \int_{V_1}^{V_2}\frac{\nu R T}{V}dV = \nu RT \int_{V_1}^{V_2} \frac{dV}{V}  = \nu RT \ln{\frac{V_2}{V_1}}\]
\begin{itemize}
    \item $V_2 > V_1 \Rightarrow \ln{\frac{V_2}{V_1}} > 0$, то есть работа будет положительной
    \item $V_2 < V_1 \Rightarrow \ln{\frac{V_2}{V_1}} < 0$, то есть работа будет отрицательной
\end{itemize}

\subsection{Адиабатический процесс}
\define \textit{\textbf{Адиабатический процесс} - процесс протекающий без теплообмена с внешней средой, то есть по определению $Q = 0$, тогда из первого начала термодинамики получаем $\Delta U + A = 0$. }

\textit{Работа в таком процессе выполняется за счет изменения внутренней энергии.}

\vspace{4px}

Теплоемкость в этом процессе: \[ \mathbb{C}_Q = \frac{Q}{\Delta T} = \frac{0}{\Delta T} = 0 \]

Запишем уравнение этого процесса. Рассмотрим:
\[PV = \nu RT \Rightarrow d(PV) = d(\nu RT) \Rightarrow \]
\[ PdV + VdP = \nu R dT \]
Преобразуем формулу $\Delta U + A = 0$ как ($\delta A$ - элементарная работа):
\[ dU + \delta A = 0 \Rightarrow d(\nu \mathbb{C}_V T) + pdV \]
\[ \nu \mathbb{C}_VdT +pdV = 0\]
Итого:
\begin{equation*}
    \begin{cases}
        PdV + VdP = \nu R dT |\cdot \mathbb{C}_V
        \\
        \nu \mathbb{C}_VdT = -pdV | \cdot R
    \end{cases}
\end{equation*}
Домножим эти уравнения на соответствующие множители и сложим:
\[(\mathbb{C}_V(PdV + VdP) + RPdV = 0 \]
\[\mathbb{C}_V + R)PdV + \mathbb{C}_VVdP = 0 \]
\[ \langle \mathbb{C}_V + R = \mathbb{C}_P, \gamma = \frac{\mathbb{C}_P}{\mathbb{C}_V} \rangle\]
\[VdP = -\gamma PdV\]
Решим это дифференциальное уравнение:
\[V \frac{dP}{dV} = -\gamma P \]
\[\ln{P} = -\gamma \ln{V} + \ln{C} \]
\[ P = V^{- \gamma} \cdot C\]
\textbf{Окончательное уравнение для адиабатического процесса:
    \[ PV^{\gamma} = const\]}
Из формулы Майера получим $\mathbb{C}_V = \frac{i}{2}R, \mathbb{C}_P = \frac{i+2}{2}R$, тогда из $\gamma = \frac{\mathbb{C}_P}{\mathbb{C}_V} \Rightarrow $
\[\gamma = \frac{i+2}{i}\]

Очевидно, что показатель для адиабатического процесса $\gamma > 1$

Построим график этого процесса
\[ P = \frac{B}{V^{\gamma}}\]
\begin{center}
    \definecolor{qqqqzz}{rgb}{0,0,0.6}
    \definecolor{qqqqcc}{rgb}{0,0,0.8}
    \begin{tikzpicture}[line cap=round,line join=round,>=triangle 45,x=1.5848527792423721cm,y=1.5541752222939598cm]
        \clip(34.29,69.05) rectangle (39.42,72.66);
        \draw [->] (20,16.5) -- (20,18.5);
        \draw [->] (20,16.5) -- (22.5,16.5);
        \draw (21,17)-- (21,18);
        \draw [dash pattern=on 1pt off 1pt] (20,17)-- (21,17);
        \draw [dash pattern=on 1pt off 1pt] (20,18)-- (21,18);
        \draw [dash pattern=on 1pt off 1pt] (21,17)-- (21,16.5);
        \draw (19.48,18.16) node[anchor=north west] {$P_1$};
        \draw (19.51,17.24) node[anchor=north west] {$P_2$};
        \draw (21.09,18.27) node[anchor=north west] {$1$};
        \draw (21.13,17.31) node[anchor=north west] {$2$};
        \draw (20.95,16.43) node[anchor=north west] {$V$};
        \draw [->] (36,34.5) -- (36,36.5);
        \draw [->] (36,34.5) -- (39.5,34.5);
        \draw [->] (27,58) -- (27,60);
        \draw [->] (27,58) -- (30.5,58);
        \draw [dash pattern=on 1pt off 1pt] (27.5,58)-- (27.5,59);
        \draw (27.5,59)-- (29,59);
        \draw [dash pattern=on 1pt off 1pt] (29,59)-- (29,58);
        \draw [dash pattern=on 1pt off 1pt] (27.5,59)-- (27,59);
        \draw (26.44,59.24) node[anchor=north west] {$P$};
        \draw (27.41,57.83) node[anchor=north west] {$V_1$};
        \draw (28.91,57.88) node[anchor=north west] {$V_2$};
        \draw (27.97,58.77) node[anchor=north west] {$A$};
        \draw (27.44,59.4) node[anchor=north west] {$1$};
        \draw (28.92,59.4) node[anchor=north west] {$2$};
        \draw [->] (35,69.5) -- (35,72.5);
        \draw [->] (35,69.5) -- (39,69.5);
        \draw [shift={(37.57,72.07)}] plot[domain=3.06:4.76,variable=\t]({1*2.27*cos(\t r)+0*2.27*sin(\t r)},{0*2.27*cos(\t r)+1*2.27*sin(\t r)});
        \draw (37.68,69.79)-- (39.03,69.88);
        \draw [shift={(39.16,74.89)},dash pattern=on 1pt off 1pt,color=qqqqcc]  plot[domain=3.65:4.6,variable=\t]({1*4.77*cos(\t r)+0*4.77*sin(\t r)},{0*4.77*cos(\t r)+1*4.77*sin(\t r)});
        \draw (34.38,72.3) node[anchor=north west] {$P_1$};
        \draw (35.29,72.1)-- (35,72.11);
        \draw (35.29,72.1)-- (35.31,69.5);
        \draw (35.22,69.45) node[anchor=north west] {$V_1$};
        \draw (36.43,70.5) node[anchor=north west] {$P = \frac{B}{V^\gamma}$};
        \draw [color=qqqqzz](37.26,70.85) node[anchor=north west] {$\textit{изобара}$};
        \draw (36.88,71.32) node[anchor=north west] {$\textit{капибара}$};
        \draw (37.18,70.11) node[anchor=north west] {$\textit{адиабата}$};
    \end{tikzpicture}
\end{center}
\begin{enumerate}
    \item Чисто(идеальный) изотермический процесс можно осуществить только за бесконечно большое время из-за колебаний температуры
    \item Аналогично, идеально адиабатический процесс также невозможен в естественных условиях и протекает только за бесконечное время.
\end{enumerate}

\section{Циклы в газах}
\define \textit{\textbf{Циклом в газе} называется процесс при котором газ, проходя ряд термодинамических состояний возвращается в исходное состояние.}
\begin{center}
    \definecolor{qqqqzz}{rgb}{0,0,0.6}
    \definecolor{qqqqcc}{rgb}{0,0,0.8}
    \begin{tikzpicture}[line cap=round,line join=round,>=triangle 45,x=1.510947825859174cm,y=1.505428327740326cm]
        \clip(46.41,88.63) rectangle (52.28,92.58);
        \draw [->] (20,16.5) -- (20,18.5);
        \draw [->] (20,16.5) -- (22.5,16.5);
        \draw (21,17)-- (21,18);
        \draw [dash pattern=on 2pt off 2pt] (20,17)-- (21,17);
        \draw [dash pattern=on 2pt off 2pt] (20,18)-- (21,18);
        \draw [dash pattern=on 2pt off 2pt] (21,17)-- (21,16.5);
        \draw (19.48,18.16) node[anchor=north west] {$P_1$};
        \draw (19.51,17.24) node[anchor=north west] {$P_2$};
        \draw (21.09,18.27) node[anchor=north west] {$1$};
        \draw (21.13,17.31) node[anchor=north west] {$2$};
        \draw (20.95,16.43) node[anchor=north west] {$V$};
        \draw [->] (36,34.5) -- (36,36.5);
        \draw [->] (36,34.5) -- (39.5,34.5);
        \draw [->] (27,58) -- (27,60);
        \draw [->] (27,58) -- (30.5,58);
        \draw [dash pattern=on 2pt off 2pt] (27.5,58)-- (27.5,59);
        \draw (27.5,59)-- (29,59);
        \draw [dash pattern=on 2pt off 2pt] (29,59)-- (29,58);
        \draw [dash pattern=on 2pt off 2pt] (27.5,59)-- (27,59);
        \draw (26.44,59.24) node[anchor=north west] {$P$};
        \draw (27.41,57.83) node[anchor=north west] {$V_1$};
        \draw (28.91,57.88) node[anchor=north west] {$V_2$};
        \draw (27.97,58.77) node[anchor=north west] {$A$};
        \draw (27.44,59.4) node[anchor=north west] {$1$};
        \draw (28.92,59.4) node[anchor=north west] {$2$};
        \draw [->] (35,69.5) -- (35,72.5);
        \draw [->] (35,69.5) -- (39,69.5);
        \draw [shift={(37.57,72.07)}] plot[domain=3.06:4.76,variable=\t]({1*2.27*cos(\t r)+0*2.27*sin(\t r)},{0*2.27*cos(\t r)+1*2.27*sin(\t r)});
        \draw (37.68,69.79)-- (39.03,69.88);
        \draw [shift={(39.16,74.89)},dash pattern=on 2pt off 2pt,color=qqqqcc]  plot[domain=3.65:4.6,variable=\t]({1*4.77*cos(\t r)+0*4.77*sin(\t r)},{0*4.77*cos(\t r)+1*4.77*sin(\t r)});
        \draw (34.38,72.3) node[anchor=north west] {$P_1$};
        \draw (35.29,72.1)-- (35,72.11);
        \draw (35.29,72.1)-- (35.31,69.5);
        \draw (35.22,69.45) node[anchor=north west] {$V_1$};
        \draw (36.43,70.5) node[anchor=north west] {$P = \frac{B}{V^\gamma}$};
        \draw [color=qqqqzz](37.26,70.85) node[anchor=north west] {$\textit{изобара}$};
        \draw (36.88,71.32) node[anchor=north west] {$\textit{капибара}$};
        \draw (37.18,70.11) node[anchor=north west] {$\textit{адиабата}$};
        \draw [->] (47.5,89) -- (47.5,92.5);
        \draw [->] (47.5,89) -- (52,89);
        \draw [rotate around={158.46:(49.46,90.98)}] (49.46,90.98) ellipse (2.35cm and 1.11cm);
        \draw [->] (49.15,90.44) -- (48.48,90.88);
        \draw [dash pattern=on 2pt off 2pt] (48,91.5)-- (47.5,91.5);
        \draw [dash pattern=on 2pt off 2pt] (50.77,90.23)-- (50.78,88.94);
        \draw (48.32,89.81) node[anchor=north west] {$A_{12} > 0 , A_{21} < 0$};
        \draw (47.78,92.11) node[anchor=north west] {$1$};
        \draw (50.56,90.82) node[anchor=north west] {$2$};
        \draw (47.09,91.74) node[anchor=north west] {$P_1$};
        \draw (50.76,88.96) node[anchor=north west] {$V_2$};
        \draw [dash pattern=on 2pt off 2pt] (48,91.5)-- (48,89);
        \draw (47.89,88.97) node[anchor=north west] {$V_1$};
        \begin{scriptsize}
            \fill [color=black] (48,91.5) circle (1.5pt);
            \fill [color=black] (50.77,90.23) circle (1.5pt);
        \end{scriptsize}
    \end{tikzpicture}
\end{center}
Разбивая цикл 1-1 точкой 2, получаем разбиение участка и совершаемой работы: $A_{1,2} > 0, A_{2,1} < 0 \Rightarrow A = A_{1,2}+A_{2,1}$.

\vspace{4px}

Заметим, что площадь этой фигуры есть работа: $S = A$

\vspace{4px}

--- при переходе $1 \to 2$ газ получает тепло $+Q_1$ - расширяется

\vspace{4px}

--- при переходе из $2 \to 1$ отдает тепло $-Q_2$ - сжимается

\vspace{4px}

\textbf{Работа любой тепловой машины находится как $A = Q_1 - Q_2$}, где $Q_1$ - газ берет тепло от нагревателя, $Q_2$ - газ отдает тепло холодильнику. Тогда у машины есть \textit{коэффициент полезного действия}, при котором работа будет минимальна:
\[ \eta = \frac{A}{Q_1} = \frac{Q_1 - Q_2}{Q_1} = 1- \frac{Q_2}{Q_1}\]

\textit{Циклы тепловых машин идет по часовой стрелке, холодильников против часовой стрелки.} Существуют и холодильные машины, которые за счет работы не повышают температуру, а наоборот снижают.
\section{Цикл Карно, машина Карно}
\begin{center}
    \definecolor{qqqqcc}{rgb}{0,0,0.8}
    \begin{tikzpicture}[line cap=round,line join=round,>=triangle 45,x=6.584933167851849cm,y=6.769158393874768cm]
        \clip(15.6,11.44) rectangle (17.21,12.66);
        \draw [->] (15.8,11.6) -- (15.8,12.6);
        \draw [->] (15.8,11.6) -- (17.2,11.6);
        \draw [shift={(16.45,12.81)},color=qqqqcc]  plot[domain=3.77:5.02,variable=\t]({1*0.54*cos(\t r)+0*0.54*sin(\t r)},{0*0.54*cos(\t r)+1*0.54*sin(\t r)});
        \draw [shift={(17.18,12.41)}] plot[domain=3.34:4.27,variable=\t]({1*0.58*cos(\t r)+0*0.58*sin(\t r)},{0*0.58*cos(\t r)+1*0.58*sin(\t r)});
        \draw [shift={(17.05,15.3)},color=qqqqcc]  plot[domain=4.46:4.68,variable=\t]({1*3.41*cos(\t r)+0*3.41*sin(\t r)},{0*3.41*cos(\t r)+1*3.41*sin(\t r)});
        \draw [shift={(16.71,12.48)}] plot[domain=3.13:3.9,variable=\t]({1*0.71*cos(\t r)+0*0.71*sin(\t r)},{0*0.71*cos(\t r)+1*0.71*sin(\t r)});
        \draw (15.65,12.66) node[anchor=north west] {$P$};
        \draw (17.14,11.57) node[anchor=north west] {$V$};
        \draw (15.93,12.64) node[anchor=north west] {$T_1$};
        \draw (16.61,12.46) node[anchor=north west] {$T_1$};
        \draw (16.98,11.98) node[anchor=north west] {$T_2$};
        \draw (16.11,12.01) node[anchor=north west] {$T_2$};
        \draw (16.26,12.53) node[anchor=north west] {$\downarrow \downarrow \downarrow Q_1$};
        \draw (16.33,11.92) node[anchor=north west] {$\downarrow \downarrow \downarrow Q_2$};
        \draw [dash pattern=on 4pt off 4pt] (16.01,12.49)-- (15.8,12.5);
        \draw [dash pattern=on 4pt off 4pt] (16.61,12.3)-- (15.8,12.29);
        \draw [dash pattern=on 4pt off 4pt] (16.93,11.89)-- (15.8,11.89);
        \draw [dash pattern=on 4pt off 4pt] (16.2,12)-- (15.8,12);
        \draw [dash pattern=on 4pt off 4pt] (16.01,12.49)-- (16,11.6);
        \draw [dash pattern=on 4pt off 4pt] (16.61,12.3)-- (16.6,11.6);
        \draw [dash pattern=on 4pt off 4pt] (16.93,11.89)-- (16.93,11.6);
        \draw [dash pattern=on 4pt off 4pt] (16.2,12)-- (16.2,11.6);
        \draw (15.68,12.54) node[anchor=north west] {$P_1$};
        \draw (15.66,12.35) node[anchor=north west] {$P_2$};
        \draw (15.66,12.08) node[anchor=north west] {$P_4$};
        \draw (15.68,11.96) node[anchor=north west] {$P_3$};
        \draw (15.97,11.57) node[anchor=north west] {$V_1$};
        \draw (16.17,11.59) node[anchor=north west] {$V_4$};
        \draw (16.56,11.59) node[anchor=north west] {$V_2$};
        \draw (16.91,11.58) node[anchor=north west] {$V_3$};
        \draw [color=qqqqcc](16.17,12.45) node[anchor=north west] {$\text{изотерма}$};
        \draw (16.7,12.21) node[anchor=north west] {$\text{адиабата}$};
        \begin{scriptsize}
            \fill [color=black] (16.01,12.49) circle (1.5pt);
            \fill [color=black] (16.61,12.3) circle (1.5pt);
            \fill [color=black] (16.93,11.89) circle (1.5pt);
            \fill [color=black] (16.2,12) circle (1.5pt);
        \end{scriptsize}
    \end{tikzpicture}
\end{center}
\define \textit{\textbf{Обратный процесс} называется процесс, который можно провести в обратном направлении без изменений в окружающей среде. Такие процессы только изотермические или адиабатические(условно обратимые)}
\begin{itemize}
    \item $1 \to 2$ \textbf{изотермический} процесс $T_1 = const$ забирает тепло $Q_1$ от нагревателя, газ есть рабочее тело
    \item $2 \to 3$ \textbf{адиабатический} процесс $ Q = 0 ; T_1 \to T_2(T_2 < T_1)$
    \item $3 \to 4$ \textbf{изотермический} процесс $T_2 = const$, газ отдает тепло $Q_2$
    \item $4 \to 1$ \textbf{адиабатический} процесс $Q = 0$, так что $T_2 \to T_1$
\end{itemize}
Посчитаем \textit{коэффициент полезного действия}:
\[ \eta = \frac{Q_1 - Q_2}{Q_1}\]
\[A = Q_1 - Q_2 \]

Выведем формулу \textit{коэффициента полезного действия} для цикла Карно:
\begin{equation*}
    \begin{cases}
        1 \to 2: Q_1 = \nu RT_1 ln{\frac{V_2}{V_1}}
        \\
        3 \to 4: Q_2 = \nu RT_2 ln{\frac{V_3}{V_4}}
    \end{cases}
\end{equation*}

Запишем
\[ \eta  = \frac{\nu RT_1 ln{\frac{V_2}{V_1}} - \nu RT_2 ln{\frac{V_3}{V_4}}}{\nu RT_1 ln{\frac{V_2}{V_1}}} =  \frac{T_1ln{\frac{V_2}{V_1}} - T_2ln{\frac{V_3}{V_4}}}{T_1ln{\frac{V_2}{V_1}}}\]
Рассмотрим адиабатический процесс:
\[ P_1V_1^{\gamma} = P_2V_2^{\gamma}\]
Рассмотрим \textbf{уравнение Менделеева-Клайперона }$ PV = \nu RT$ и подставим в формулу:
\[ \frac{\nu RT_1}{V_1}V_1^{\gamma} = \frac{\nu RT_2}{V_2}V_2^{\gamma} \Rightarrow \]
\[ T_1V_1^{\gamma - 1} = T_2V_2^{\gamma - 1}\]
\textit{Перейдем к процессу $2 \to 3$  и к $ 4 \to 1$ :}
\begin{equation*}
    \begin{cases}
        2 \to 3: T_1V_2^{\gamma - 1} = T_2V_3^{\gamma - 1}
        \\
        4 \to 1: T_2V_4^{\gamma - 1} = T_1V_1^{\gamma - 1}
    \end{cases}
\end{equation*}
\begin{equation*}
    \begin{cases}
        \frac{T_1}{T_2} = (\frac{V_3}{V_2}) ^ {\gamma - 1}
        \\
        \frac{T_1}{T_2} = (\frac{V_4}{V_2}) ^ {\gamma - 1}
    \end{cases}
\end{equation*}

Из этих выражений получим:
\[ \frac{V_3}{V_2} = \frac{V_4}{V_1} \Rightarrow \frac{V_2}{V_1} = \frac{V_3}{V_4} \Rightarrow\]
\[ \eta = \frac{T_1ln{\frac{V_2}{V_1}} - T_2ln{\frac{V_3}{V_4}}}{T_1ln{\frac{V_2}{V_1}}} = \frac{T_1 - T_2}{T_1}\]

\textit{\textbf{Важно: }КПД идеальной тепловой машины всегда больше КПД другой машины работающей с одним и тем же нагревателем и холодильником.} Рассмотрим это утверждение:
\[ \frac{T_1 - T_2}{T_1} \geq\ \frac{Q_1 - Q_2}{Q_1} \Rightarrow \]
\[1 - \frac{T_2}{T_1} \geq 1 - \frac{Q_2}{Q_1} \Rightarrow \]
\[ \frac{Q_2}{Q_1} \geq \frac{T_2}{T_1} \Rightarrow \frac{Q_2}{T_2} \geq \frac{Q_1}{T_1} \Rightarrow \]
\[\frac{Q_1}{T_1} - \frac{Q_2}{T_2}  \leq 0\]
Заметим что $Q_1$ - количество тепла \textit{поступающее} газу, $Q_2$ - количество газа \textit{уходящего}, тогда очевидно что $Q_2 < 0$, тогда:
\[\frac{Q_2}{T_2} + \frac{Q_1}{T_1} \leq 0\]
\textbf{\textit{ --- неравенство Клазиуса, запишем для нескольких машин:}}
\[ \sum_{i = 1}^{n} \frac{Q_i}{T_i} \leq 0\]

\section{Второе начало термодинамики}
\textit{\textbf{Формулировка 1:} Невозможны процессы \underline{единственным} результатом которых было бы передача тепла от тела менее нагретого к более нагретому телу.Такая формулировка указывает направление движения процесса.}

\vspace{5px}

\textit{\textbf{Формулировка 2:} Невозможны те процессы в которых все тепло взятое у нагревателя полностью превращалось бы в работу.}

\vspace{5px}

Покажем эквивалентность\textit{ Формулировки 1 и Формулировки 2}.

\vspace{4px}
\begin{itemize}
    \item Возьмем два тела: заберем тепло $Q_2$ у тела 1 и передадим это тепло совершив работу $A = Q_2$ более нагретому телу 2. Тогда получается, что мы передали полностью тепло от менее нагретого тела к более нагретому : $T_1 \to T_2$, что невозможно по первой формулировке то есть тепло не полностью превращается в работу, что и требовалось доказать.
\end{itemize}
\subsection{Энтропия}
Заметим, что любое макросостояние состоит из множества микросостояний .

\vspace{4px}

\define \textit{\textbf{Термодинамическая вероятность \texttt{W}} - количество микросостояний систем соответствующее данному макросостоянию. Измерим общую термодинамическую вероятность $W_1, W_2$, где каждое состояние $W_1$ соответствует состоянию $W_2$ : $W = W_1 \cdot W_2$}

\vspace{4px}

Поскольку таких состояний очень много, величиной W очень неудобно оперировать. Больцманом было предложено использовать логарифм этой величины и с помощью него было введено понятие энтропии.

\vspace{5px}

\textit{\textbf{Энтропия : $S = k \cdot ln{W}, k - $ постоянная Больцмана.}}

\vspace{5px}

\textit{\textbf{Формулировка 3: }Энтропия системы не убывает($\Delta S \geq  0 $)}
Такая формулировка относится не ко всем системам, но к любой замкнутой, применяется только для систем с огромным числом степеней свободы.

\section{Явление переноса в термодинамических неравновесных системах(в газах)}
К ним относятся:
\begin{enumerate}
    \item Внутреннее трение(вязкость) - перенос импульса
    \item Теплопроводность - перенос энергии
    \item Диффузия - перенос массы
\end{enumerate}

\subsection{Вязкость. Внутреннее трение}
Рассмотрим газ:
\begin{center}
    \begin{tikzpicture}[line cap=round,line join=round,>=triangle 45,x=1.0cm,y=1.0cm]
        \clip(20.08,18.84) rectangle (28.32,23.38);
        \draw [->] (21,19) -- (21,23);
        \draw (21,21)-- (28,21);
        \draw [->] (22,22) -- (24,22);
        \draw [->] (22,20) -- (24,20);
        \draw (22.42,23.2) node[anchor=north west] {$v_1$};
        \draw (22.72,20.8) node[anchor=north west] {$v_2$};
        \draw (27.54,21.88) node[anchor=north west] {$V$};
        \draw (24.46,21.12) node[anchor=north west] {\textit{разделение слоев}};
        \begin{scriptsize}
            \fill [color=black] (22,22) circle (1.5pt);
            \fill [color=black] (22,20) circle (1.5pt);
        \end{scriptsize}
    \end{tikzpicture}
\end{center}
\[ f = \eta \frac{dV}{dz}S\]
- сила на единицу площади, где $\eta$ - коэффициент вязкости, $S$ - площадь соприкосновения слоев, $\frac{dV}{dz}$ - скорость движения вдоль z.

\vspace{4px}

Попробуем обосновать и вывести эту формулу:

\vspace{4px}

Пусть $v_1 > v_2$, $U$ - средняя скорость хаотического(теплового) движения молекул, $m_0$ - масса молекул. Пусть общие импульсы слоев соответственно $J_1$ и $J_2$. $v_1, v_2$ - скорость соответственно упорядоченного движения молекул.

\vspace{5px}

Обозначим за число молекул переходящих из первого слоя во второй слой $\Delta N_{12}$ и наоборот: $\Delta N_{21}$, тогда очевидно: $\Delta N_{12} = \Delta N_{21} = \Delta N$, посчитаем это число
\[\Delta N = \frac{1}{6} n U S \Delta t\]
Посчитаем переход импульса $\Delta J_1 = \Delta N m_0V_1 , \Delta J_2 = \Delta N m_0V_2$:
\[\Delta J^{I} = \Delta J_2 - \Delta J_1 = \Delta N m_0(v_2 - v_1) < 0\]
для первого слоя
\newpage
\[\Delta J^{II} = \Delta J_1 - \Delta J_2 = \Delta N m_0(v_1 - v_2) > 0\]
для второго слоя

\vspace{5px}

--- \textit{Основная мысль в том, что  весь потерянный импульс первого слоя получает второй слой}, то есть:
\[\Delta J^{II} = -\Delta J^{I}\]
Выразим импульс через силу:
\[f_1 = \frac{\Delta J_1}{\Delta t} = \frac{1}{6}nUm_0(v_2 - v_1)S\]

Обозначим длину свободного пробега молекулы(средняя): $<l>$

\vspace{5px}

Тогда $v_2 = v(z - <l>), v_1 = v(z + <l>)$, учитывая, что $<l>$ - крайне мала:
\[v_2 = v(z) - <l>\frac{dV}{dz} + O(<l>^2)\]
\[v_1 = v(z) + <l>\frac{dV}{dz} + O(<l>^2)\]
Подставляя в $f_1 = \frac{\Delta J_1}{\Delta t} = \frac{1}{6}nUm_0(v_2 - v_1)S$, получим:
\[f_1 = \frac{1}{6}nUm_0 \cdot 2 \cdot <l> \cdot \frac{dV}{dz} \cdot S\]
для первого слоя.

\vspace{5px}

Аналогично для второго, только с противоположным знаком:
\[f_1 = \frac{1}{6}nUm_0 \cdot -2 \cdot <l> \cdot \frac{dV}{dz} \cdot S\]

\subsection{Теплопроводность}
Пусть $T_1 > T_2$, выразим $T_1, T_2$ как:
\[T_1 = T(x - <l>)\]
\[T_2 = T(x+<l>)\]
Изменения количества молекул соответственно:
\[\Delta N = \frac{1}{6}nUS\Delta t\]
Примем то, что U - значение не значащее и следовательно не меняющееся. Как известно из \textit{основного уравнения МКТ} следует:
\[<\epsilon_1> = \frac{3}{2}kT_1 \Rightarrow \Delta Q_1 = \Delta N \cdot <\epsilon_1>\]
\[<\epsilon_2> = \frac{3}{2}kT_2 \Rightarrow \Delta Q_2 = \Delta N \cdot <\epsilon_2>\]

Тогда вычислим изменения количества теплоты в первом слое:
\[ \Delta Q ^{I} = -\Delta Q_1 + \Delta Q_2 = \Delta N(<\epsilon_2> - <\epsilon_1>)\]

Подставляя $\Delta N = \frac{1}{6}nUS\Delta t,<\epsilon> = \frac{3}{2}kT $ получим:
\[\Delta Q ^{I} = \frac{1}{6}nUS\Delta t\frac{3}{2}k(T_2 - T_1)\]
\[T_1 =T(x) -<l>\frac{dT}{dx}\]
\[T_2 = T(x) +<l>\frac{dT}{dx}\]

Обозначим $q$ - поток тепла через единичную поверхность в единицу времени: \[ q = \frac{\Delta Q}{S \Delta t}\]
Разделим $\Delta Q ^{I} = \frac{1}{6}nUS\Delta t\frac{3}{2}k(T_2 - T_1)$ на $S\Delta t$, получим:
\[\frac{\Delta Q}{\Delta t S} = q =  \frac{1}{6}nU\frac{3}{2}k\cdot 2<l>\frac{dT}{dx}\]
Тогда:
\[q = \frac{1}{3}nU\frac{3}{2}k<l>\frac{dT}{dx}\]
\begin{flushright} , где $\frac{1}{3}nU\frac{3}{2}k<l>$ - коэффициент теплопроводности - $\lambda$ \end{flushright}

\vspace{5px}

Поскольку $T_1 > T_2 \Rightarrow \frac{dT}{dx} < 0$
\[q = -\lambda\frac{dT}{dx}\]
- \textbf{\textit{общее уравнение переноса тепла}}
\subsection{Диффузия}
Имеем как минимум два различны вещества с различной концентрацией
\begin{center}
    \begin{tikzpicture}[line cap=round,line join=round,>=triangle 45,x=1.0cm,y=1.0cm]
        \clip(31.16,21.72) rectangle (46.98,28.33);
        \draw [->] (36,22) -- (36,28);
        \draw [->] (34,23) -- (39,23);
        \draw [->] (43,22) -- (43,28);
        \draw [->] (41,23) -- (46,23);
        \draw [dash pattern=on 3pt off 3pt] (35,23)-- (35,28);
        \draw [dash pattern=on 3pt off 3pt] (37,23)-- (37,28);
        \draw [shift={(45.86,28.29)}] plot[domain=3.23:4.75,variable=\t]({1*4.8*cos(\t r)+0*4.8*sin(\t r)},{0*4.8*cos(\t r)+1*4.8*sin(\t r)});
        \draw [shift={(39.56,28.58)}] plot[domain=4.98:6.15,variable=\t]({1*5.39*cos(\t r)+0*5.39*sin(\t r)},{0*5.39*cos(\t r)+1*5.39*sin(\t r)});
        \draw (44.95,27.65) node[anchor=north west] {$n_1$};
        \draw (41.35,28.02) node[anchor=north west] {$n_2$};
        \draw (37.16,27.7) node[anchor=north west] {$n_1(x+<l>)$};
        \draw (37.18,24.8) node[anchor=north west] {$n_2(x+<l>)$};
        \draw (32.25,27.63) node[anchor=north west] {$n_1(x - <l>)$};
        \draw (32.27,24.8) node[anchor=north west] {$n_2(x-<l>)$};
        \draw (35.37,22.89) node[anchor=north west] {<l>};
        \draw (38.59,23.76) node[anchor=north west] {$x$};
    \end{tikzpicture}
\end{center}
При рассмотрении одного вещества получаем перенос массы.

\vspace{4px}

Пусть G - \textit{количество молекул, прошедшие через единичную площадку в единицу времени:} $G = \frac{\Delta N}{S \Delta t}$. Тогда:
\[\Delta N_1 = \frac{1}{6}US\Delta t(n_1(x+<l>) - n_1(x-<l>)\]
Распишем концентрации $n_1, n_2$ как:
\[n_1(x+<l>) = n_1(x) + <l>\frac{dn_1}{dx}\]
\[n_2(x-<l>) = n_1(x) - <l>\frac{dn_1}{dx}\]

Тогда подставив получим:
\[ \Delta N = \frac{1}{6}US\Delta t <l> \cdot 2 \frac{dn_1}{dx}\]
Вычислим G. Подставим полученные значения:
\[G = \frac{1}{3}U <l> \frac{dn_1}{dx}\]

Обозначим $ D = \frac{1}{3}U <l>$ - \textit{коэффициент диффузии}. Отметим, что изначальный поток идет против оси x, в сторону уменьшения концентрации.

\[G = -D \frac{dn}{dx}\]
--- \textbf{\textit{уравнение диффузии}}
