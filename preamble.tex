\usepackage[english,russian]{babel}
\usepackage[T2A]{fontenc}
\usepackage[utf8]{inputenc}
\usepackage{amsmath, amssymb, amsfonts, amsthm, mathtools}
\usepackage{bookmark}
\usepackage{geometry}
\usepackage{graphicx}
\usepackage{fancyhdr}
\usepackage{subfiles}
\usepackage{tikz}
\usepackage{pgf,tikz}
\usepackage{esint}
\usepackage{mathrsfs}

\usetikzlibrary{arrows}
\geometry{a4paper, margin=1in}
\usepackage{hyperref}
\hypersetup{
    colorlinks=true,
    linkcolor=blue,
    urlcolor=blue,
    pdftitle={Физика},
    pdfauthor={},
    pdfsubject={},
    pdfkeywords={Определение, формула, законы, процессы},
}

\usepackage{indentfirst} % Отступ для первого абзаца
\usepackage{titlesec}
\titleformat{\section}{\Large\bfseries}{\thesection}{1em}{}
\titleformat{\subsection}{\large\bfseries}{\thesubsection}{1em}{}


\setcounter{secnumdepth}{4}
\setcounter{tocdepth}{2}
\newcommand {\define}{--- \textbf{Определение:  }}

\renewcommand{\thechapter}{\Roman{chapter}}
\renewcommand{\thesection}{\arabic{section}}
\renewcommand{\thesubsection}{\thesection.\arabic{subsection}}
\renewcommand{\thesubsubsection}{\thesubsection.\arabic{subsubsection}}

\renewcommand{\chaptermark}[1]{\markboth{#1}{}}
\renewcommand{\sectionmark}[1]{\markright{#1}}

\newcommand{\deriv}[2]{\frac{\partial #1}{\partial #2}} % Производная
\renewcommand{\phi}{\varphi} % Переопределение символа phi
\renewcommand{\epsilon}{\varepsilon} % Переопределение символа epsilon

% Настройка верхнего колонтитула
\pagestyle{fancy}
\fancyhf{}
\fancyhead[L]{\rightmark} % Левый верхний угол: название главы
\fancyhead[R]{\thepage} % Правый верхний угол: номер страницы
\renewcommand{\headrulewidth}{0.2pt} % Линия под верхним колонтитулом

%настройка теорем и прочего
\theoremstyle{plain}
\newtheorem{theorem}{Теорема}[section]
\newtheorem*{definition}{Определение}
\newtheorem{example}{Пример}
\newtheorem{proposition}{Утверждение}
\newtheorem{corollary}{Следствие}[theorem]
\newtheorem{lemma}{Лемма}
